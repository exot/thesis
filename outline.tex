\documentclass[english,fleqn]{scrartcl}
\usepackage[T1]{fontenc}
\usepackage[utf8]{inputenc}
\usepackage{babel}

\usepackage[nobackrefs]{preamble}
\usepackage[subtle]{savetrees}

\pagestyle{empty}

\begin{document}

\medskip
\begin{center}
  \normalsize Abstract of the thesis\\
  \LARGE\textbf{Learning Terminological Knowledge\\ with High Confidence from Erroneous
    Data}\\
  \bigskip%
  \large Dipl.-Math.\ Daniel Borchmann
\end{center}
\bigskip
\bigskip

\noindent
Knowledge is a necessary requirement for intelligent behavior.  For computers, in
particular, it is indispensable to obtain relevant knowledge before they can act
intelligently in certain situations.  The problem of obtaining this knowledge, commonly
referred to as the problem of \emph{knowledge acquisition} or just \emph{learning}, is
hard to address, because computers, in contrast to human beings, are not able to learn
autonomously.  Moreover, for computers it is relevant that the knowledge they obtain is
represented in a suitable way for them to work with.  Thus, the problem of \emph{knowledge
  representation} has also to be tackled to enable intelligent behavior of machines.

A popular choice to represent knowledge for computers is to use logic-based formalisms,
because they yield predictable and reliable behavior.  In the area of logic-based
knowledge representation formalisms, \emph{description logics}~\cite{DLhandbook} are one of
the most successful approaches.  By the use of \emph{description logic knowledge bases} it
is possible to represent \emph{assertional} and \emph{terminological} knowledge.
Assertional knowledge expresses facts about one or two individuals of the domain.
Examples for this kind of knowledge are \enquote{Abraham Lincoln was a president}, or
\enquote{John McCarthy was a professor at Stanford University}.  In other words,
assertional knowledge expresses \emph{assertions} (\emph{facts}) about certain
individuals.

Terminological knowledge represents common knowledge about all individuals of the domain
of interest.  Examples for this kind of knowledge are \enquote{all humans are mortal} or
\enquote{every human has a head}.  Thus, terminological knowledge represents dependencies
between different \emph{terms}, called \emph{concept descriptions} in description logics.

If one chooses to represent knowledge using description logics, one is left with the
problem to obtain all relevant assertional and terminological knowledge about the domain
of interest.  Obtaining assertional knowledge is relatively easy: assuming that one knows
all individuals of the domain, one can gather all facts that are known about them.  The
difficulty here mainly lies in the problem of automating this process, in particular, if
one tries to gather the facts from unstructured data that is hard to process for
computers, like text.  On the other hand, obtaining terminological knowledge cannot be
achieved this way, since one is looking for knowledge that affects \emph{all} individuals
of the domain.  Moreover, it is not clear at all how to find the concept descriptions that
appear in this kind of knowledge.

Recent works by Baader and Distel~\cite{Diss-Felix} propose an approach to learning
terminological knowledge, using ideas from the area of \emph{formal concept
  analysis}~\cite{fca-book}.  This approach assumes that the domain of interest is
representable as a \emph{finite interpretation}.  Interpretations are relational
structures used to define the semantics of description logics, and can intuitively be seen
as directed edge- and node-labeled graphs.  Baader and Distel then propose to learn
terminological knowledge by computing \emph{finite bases} of all \emph{general concept
  inclusions} (GCIs) that are valid in a given finite interpretation and are expressible
in the description logic \ELbot.  GCIs are the most general means in description logics to
express terminological knowledge.  Using GCIs, the examples given above could be
reformulated as
\begin{equation*}
  \mathsf{Human} \sqsubseteq \mathsf{Mortal}, \;\;\; \mathsf{Human} \sqsubseteq \exists
  \mathsf{has}. \mathsf{Head}.
\end{equation*}
Then these two GCIs are valid in an interpretation if and only if every \emph{individual}
(node) that is labeled with \enquote{\textsf{Human}} is also labeled with
\enquote{\textsf{Mortal}}, and has an outgoing edge labeled with \enquote{\textsf{has}}
that points to a node labeled with \enquote{\textsf{Head}}.  Finally, finite bases
$\mathcal{B}$ are finite sets of valid GCIs that are \emph{complete}, in the sense that
all GCIs that are valid in the given interpretation are \emph{entailed} by $\mathcal{B}$.

The approach of Baader and Distel is interesting for practical applications.  This is
because finite interpretations can be seen as a different form of \emph{linked data}, the
data-format used by the \emph{Semantic Web}.  The first contribution of this work is to
evaluate the practicability of the approach by Baader and Distel by applying it to linked
data obtained by the DBpedia project~\cite{DBpedia}.  This evaluation shows that this
approach indeed allows us to learn terminological knowledge from data.

These experiments also reveal a disadvantage of the algorithm: considering only valid GCIs
leads to a high sensitivity of the extracted knowledge to \emph{errors} in the data.  In
other words, as soon as a single error is contained in the data, all valid GCIs affected
by this error are not extracted anymore.  Since one cannot assume that data, and linked
data in particular, is free of errors, this sensitivity severely impairs the usefulness of
Baader and Distel's approach.

One can assume, however, that the given data represents the domain of interest
\enquote{sufficiently well}, in the sense that it does not contain too many errors.  Based
on this assumption, one can think of alleviating the sensitivity of Baader and Distel's
approach to errors in the data by considering GCIs that are \enquote{almost valid}
instead.  This way, GCIs that are erroneously invalidated by sporadic errors are still
retrieved from the data.

It is the main goal of this thesis to generalize the results by Baader and Distel
accordingly.  For this, the notion of \emph{confidence} as used in
data-mining~\cite{arules:agrawal:association-rules} is transferred to GCIs.  Intuitively,
the confidence of a GCI $C \sqsubseteq D$ in a finite interpretation $\mathcal{I}$
quantifies how often this GCI is correct, in the sense that if an individual in
$\mathcal{I}$ satisfies the concept description $C$, then it also satisfies the concept
description $D$.  For example, the confidence of $\mathsf{Human} \sqsubseteq
\mathsf{Mortal}$ in a finite interpretation would be the number of individuals labeled
with both $\mathsf{Human}$ and $\mathsf{Mortal}$, divided by the number of individuals
just labeled with $\mathsf{Human}$.

The notion of confidence of GCIs gives a measure of \enquote{how valid} a GCI is in a
finite interpretation $\mathcal{I}$, and it can be used to formalize the notion of being
\enquote{almost valid}: a GCI is \emph{almost valid} in $\mathcal{I}$ if and only if its
confidence in $\mathcal{I}$ is above a chosen \emph{confidence threshold} $c \in [0,1]$.
Such an almost valid GCI is also called a GCI \emph{with high confidence}.  The task is
then to find finite bases for such GCIs with high confidence, \ie to find finite sets
$\mathcal{B}$ of GCIs with high confidence that are \emph{complete}, in the sense that all
GCIs with high confidence are entailed by $\mathcal{B}$.

First results on computing such bases are achieved by exploiting the close connection
between description logics and formal concept analysis that has already been used in the
approach by Baader and Distel.  Based on this connection as well as on results by
Luxenburger on \emph{partial implications} in formal contexts~\cite{diss:Luxenburger},
this thesis discusses methods to explicitly describe bases of GCIs with high confidence.
Furthermore, results are obtained that allow to compute such bases from bases of
implications with \emph{high confidence} in formal contexts.  All these findings are
applied to the data set used for the experiments above, and are evaluated for their
practicability with respect to handling sporadic errors.

A drawback of this approach is that the consideration of GCIs with high confidence is
purely heuristic, in the sense that counterexamples to GCIs are ignored if they are just
not frequent enough.  In particular, if the initial interpretation contains rare but valid
counterexamples of a given GCI, then these are ignored as well.  To remedy this, an
external source of information that is able to distinguish between these rare
counterexamples and errors in the data is necessary.  An example for such a source would
be a human expert for the domain of interest.

Given such an expert and a finite base of GCIs with high confidence, one could address
this problem by letting an expert verify every GCI contained in this base.  All GCIs
rejected by the expert are then removed.  However, this naive approach has the
disadvantage that as soon as a GCI is removed, other GCIs may have to be added to ensure
completeness of the base, and it is by far not clear how this can be done without
recomputing the base completely.

In this thesis, the more systematic approach of \emph{model exploration by confidence} is
introduced.  This is an interactive learning algorithm based on \emph{model exploration},
an algorithm devised by Baader and Distel to address the problem of \emph{incomplete
  data}, and is itself based on \emph{attribute exploration}, an interactive knowledge
acquisition algorithm from formal concept analysis.

Model exploration by confidence is essentially an interactive algorithm to compute finite
bases of GCIs with high confidence.  During the run of this algorithm, every computed GCI
is given to the expert for verification.  If the expert confirms this GCI, it is added to
the base.  If it is rejected, the expert is required to provide \emph{valid}
counterexamples for this GCI, which are then added to the data.  When the algorithm
terminates, the set of confirmed GCIs constitutes a base.  Moreover, counterexamples
provided by the expert are not subject to the confidence heuristics, \ie as soon as a
counterexample invalidates a GCI, it is not considered any further, even if its confidence
is high enough.  In this way, rare counterexamples can be distinguished from errors.

\printbibliography{}

\end{document}

%%% Local Variables: 
%%% mode: latex
%%% TeX-master: t
%%% End: 
