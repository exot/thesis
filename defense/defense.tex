\documentclass{beamer}
\usepackage{etex}
\usepackage[beamer]{mydefs}

%%%

\title{Lernen Terminologischen Wissens\\ mit hoher Konfidenz aus fehlerhaften Daten}
\author{Daniel Borchmann}
\date{9.\,September 2014}

%%%

\setbeamertemplate{headline}{}

%%%

\begin{document}

\begin{frame}[plain]
  \maketitle
\end{frame}

\begin{frame}
  Motivation: faktisches Wissen da, terminologisches Wissen nicht (kleines Beispiel)
\end{frame}

\begin{frame}
  Fragestellungen: wie Wissen repräsentieren? (→ DL) wie Wissen erlangen? (→ FCA);
  Beispiele für GCIs
\end{frame}

\begin{frame}
  EL Syntax und Semantik
\end{frame}

\begin{frame}
  GCIs
\end{frame}

\begin{frame}
  FCA
\end{frame}

\begin{frame}
  Ähnlichkeiten FCA - DL
\end{frame}

\begin{frame}
  Übersicht
\end{frame}

\begin{frame}
  DBpedia-Beispiel (auch mit sinnvollem Wissen?)
\end{frame}

\begin{frame}
  Beobachtung, was nicht gefunden wird
\end{frame}

\begin{frame}
  Konfidenz einführen, Basen mit hoher Konfidenz einführen
\end{frame}

\begin{frame}
  Neue Übersicht, mit Konfidenz
\end{frame}

\begin{frame}
  Details 1: Model-Based Most-Specific Concept Description

  (ELgfp erwähnen)
\end{frame}

\begin{frame}
  Details 2: Kontextkonstruktion, Baader-Distel Resultat
\end{frame}

\begin{frame}
  Details 3: Idee von Luxenburger, Eigenes Resultat
\end{frame}

\begin{frame}
  DBpedia: was kommt dazu?
\end{frame}

\begin{frame}
  Diagramme
\end{frame}

\begin{frame}
  Was gibt es noch?
\end{frame}

\begin{frame}
  Exploration
\end{frame}

\begin{frame}
  Ende und Schluss
\end{frame}

\end{document}

%%% Local Variables: 
%%% mode: latex
%%% TeX-master: t
%%% ispell-local-dictionary: "de_DE"
%%% TeX-engine: luatex
%%% End: 
