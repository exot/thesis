\documentclass[ngerman]{beamer}
\usepackage{etex}
\usepackage[beamer]{mydefs}
\usepackage{calc}

%%%

\title{Lernen Terminologischen Wissens\\ mit hoher Konfidenz aus fehlerhaften Daten}
\author{Daniel Borchmann}
\date{9.\,September 2014}

%%%

\setbeamertemplate{headline}{}

%%%

\begin{document}

\begin{frame}[plain]
  \maketitle
\end{frame}

\begin{frame}
  Motivation: faktisches Wissen da, terminologisches Wissen nicht (kleines Beispiel)
\end{frame}

\begin{frame}
  Fragestellungen: wie Wissen repräsentieren? (→ DL) wie Wissen erlangen? (→ FCA);
  Beispiele für GCIs
\end{frame}

\begin{frame}
  EL Syntax und Semantik
\end{frame}

\begin{frame}
  GCIs
\end{frame}

\begin{frame}
  FCA (formale Kontexte, Ableitungsoperatoren)
\end{frame}

\begin{frame}
  FCA (Implikationen, Basen)
\end{frame}

\begin{frame}
  Ähnlichkeiten FCA - DL
\end{frame}

\begin{frame}
  Übersicht
\end{frame}

\begin{frame}
  \frametitle{Extracting Knowledge from DBpedia}

  \onslide<+->
  
  \begin{block}{Experiment}
    \begin{itemize}
    \item<+-> DBpedia: aus der Wikipedia halbautomatisch gewonnenes Wissen
    \item<+-> betrachte nur \textsf{child}-Relation ${} \leadsto \Idbpedia$
    \item<+-> $\Delta^{\Idbpedia} = 5626$
    \end{itemize}
  \end{block}

  \onslide<+->

  \begin{block}{Einige Ergebnisse}
    \vspace*{-3ex}
    \begin{gather*}
      \onslide<+->{\sf MemberOfParliament \sqsubseteq Person \sqcap Politician\\}
      \onslide<+->{\sf \exists child. Person \sqsubseteq Person\\}
      \onslide<+->{\sf FictionalCharacter \sqcap \exists child. Person \sqsubseteq \exists
        child. FicitionalCharacter\\}
    \end{gather*}
  \end{block}

  \onslide<+->

  \vspace*{-3ex}
  \begin{block}{Fragwürdige Ergebnisse}
    \vspace*{-3ex}
    \begin{gather*}
      \onslide<+->{\sf Person \sqcap \exists child. Book \sqsubseteq
        FicitionalCharacter\\}
      \onslide<+->{\sf Criminal \sqcap \exists child. Politician \sqsubseteq \bot\\}
      \onslide<+->{\sf Person \sqcap \exists child. Criminal \sqsubseteq Criminal}
    \end{gather*}
  \end{block}

\end{frame}

\begin{frame}
  
  \onslide<1-8>

  \begin{block}{Beobachtung}
    \begin{equation*}
      \sf \exists child. \top \sqsubseteq Person
    \end{equation*}
    \onslide<2->%
    \emph{gilt nicht} in $\Idbpedia$\onslide<3->, denn es gibt 4 \alt<7->{\alert{falsche
        Gegenbeispiele:}}{Gegenbeispiele, \dh}
    \begin{overlayarea}{\textwidth}{5ex}
      \only<4-6>{
        \begin{equation*}
          \abs{ (\sf \exists child.top)^{\Idbpedia} \setminus Person^{\Idbpedia} } = 4.
        \end{equation*}
      }
      \only<8->{
        \begin{equation*}
          \text{\texttt{Teresa\_Carpio}, \texttt{Charles\_Heung},
            \texttt{Adam\_Cheng}, \texttt{Lydia\_Shum}.}
        \end{equation*}
      }
    \end{overlayarea}

    \onslide<5->

    \bigskip{}

    Andererseits: 2547 Elemente in $\Idbpedia$ erfüllen $\sf \exists child. \top
    \sqsubseteq \sf Person$, \dh
    \begin{equation*}
      \abs{ (\sf \exists child.\top \sqcap Person)^{\Idbpedia} } = 2547.
    \end{equation*}

    \onslide<6->

    $\sf \exists child. \top \sqsubseteq Person$ ist also
    \alt<7->{\alert{\enquote{\sout{fast}}}}{\enquote{\makebox[\widthof{\sout{fast}}]{fast}}} richtig.
  \end{block}

\end{frame}

\begin{frame}

  \onslide<+->

  \begin{block}{Ansatz}
    Betrachte auch GCIs, die \enquote{fast} richtig sind.
  \end{block}

  \onslide<+->

  Konfidenz einführen, Basen mit hoher Konfidenz einführen
\end{frame}

\begin{frame}
  Neue Übersicht, mit Konfidenz
\end{frame}

\begin{frame}
  Details 1: Model-Based Most-Specific Concept Description

  (ELgfp erwähnen)
\end{frame}

\begin{frame}
  Details 2: Kontextkonstruktion, Baader-Distel Resultat
\end{frame}

\begin{frame}
  Details 3: Idee von Luxenburger, Eigenes Resultat
\end{frame}

\begin{frame}
  DBpedia: was kommt dazu?
\end{frame}

\begin{frame}
  Diagramme
\end{frame}

\begin{frame}
  Was gibt es noch?
\end{frame}

\begin{frame}
  Exploration Übersicht
\end{frame}

\begin{frame}
  ? Exploration im Detail
\end{frame}

\begin{frame}
  Ende und Schluss
\end{frame}

\end{document}

%%% Local Variables: 
%%% mode: latex
%%% TeX-master: t
%%% ispell-local-dictionary: "de_DE"
%%% TeX-engine: xetex
%%% End: 
