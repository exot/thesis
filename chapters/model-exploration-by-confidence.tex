\chapter[Model Exploration by Confidence]{Model Exploration by Confidence with Completely Specified Counterexamples}
\label{cha:model-expl-conf}

We have seen how we can extend attribute exploration to explore implications which enjoy a
high confidence in some given formal context.  In this chapter we want to extend this
generalization of attribute exploration even further to be able to explore GCIs with high
confidence in finite interpretations.  The basis for this extension will be the
\emph{model exploration} algorithm from~\cite{Diss-Felix}, an extension of attribute
exploration to explore valid GCIs of finite interpretations.

Model exploration, similarly to attribute exploration, assumes that a certain domain of
interest is representable by a finite interpretation $\mathcal{I}_{\mathsf{back}}$, which
we shall call the \emph{background interpretation} of the exploration process.  However,
we assume that this interpretation is not directly accessible, but instead we have given
an expert that allows us to decide whether certain GCIs are valid in
$\mathcal{I}_{\mathsf{back}}$.  In addition, if a given GCI $C \sqsubseteq D$ is not valid
in $\mathcal{I}_{\mathsf{back}}$, then the expert can provide counterexamples for $C
\sqsubseteq D$ in a suitable way.

The principal way how model exploration works is again very akin to attribute exploration.
Given a finite \emph{connected subinterpretation} $\mathcal{I}$ of
$\mathcal{I}_{\mathsf{back}}$ and a set $\mathcal{B}$ of valid GCIs of
$\mathcal{I}_{\mathsf{back}}$, the algorithm successively generates valid GCIs $C
\sqsubseteq D$ of $\mathcal{I}$ which do not follow from $\mathcal{B}$, and presents them
to some expert.  If the expert confirms $C \sqsubseteq D$, then it is added to
$\mathcal{B}$.  If the expert rejects $C \sqsubseteq D$, then she provides a
counterexample to it in the form of a connected subinterpretation of
$\mathcal{I}_{\mathsf{back}}$, which is added to $\mathcal{I}$.  Since $\mathcal{I}$ and
$\mathcal{B}$ play the same role as the working context and the set of known implications
during attribute exploration, we shall refer to them as the \emph{working interpretation}
and the \emph{set of known GCIs}, respectively.  If no more GCIs can be generated to be
asked to the expert, the algorithm stops.  It can be shown that at this point, the set
$\mathcal{B}$ is a finite base of $\mathcal{I}_{\mathsf{back}}$.

The foregone description already suggests that there are some difficulties in transferring
attribute exploration to the setting of GCIs and finite interpretations.  The most
apparent is that the set of GCIs we potentially have to cover is infinite, as the set of
valid GCIs of $\mathcal{I}_{\mathsf{back}}$ is infinite.

Another problem is that validity of GCIs in interpretations deploy a \emph{closed-world}
semantics: if an element $x \in \Delta^{\mathcal{I}}$ of a (finite) interpretation
$\mathcal{I} = (\Delta^{\mathcal{I}}, \cdot^{\mathcal{I}})$ does not have an $r$-successor
in $\mathcal{I}$, for some $r \in N_{R}$, then it is assumed that $x$ does not have
$r$-successors $\mathcal{I}_{\mathsf{back}}$.  Therefore, if we add an element $x$ of
$\mathcal{I}_{\mathsf{back}}$ as a counterexample for a GCI to our working interpretation
$\mathcal{I}$, then we also have to include \emph{all} its role successors (and their role
successors, and so on) in $\mathcal{I}_{\mathsf{back}}$, even so they may not be necessary
as part of the counterexample; otherwise, elements in $\mathcal{I}$ may serve as
counterexamples to GCIs which are valid in $\mathcal{I}_{\mathsf{back}}$, because missing
information is considered as false information.  The approach followed by Baader and
Distel to account for this problem is to let the expert provide \emph{connected
  subinterpretations} of $\mathcal{I}_{\mathsf{back}}$ as counterexamples for GCIs.

We shall discuss the details of model exploration in \Cref{sec:model-expl-with}.  Based on
this discussion, we shall develop a model exploration algorithm that also includes GCIs
with high confidence among those proposed to the expert.  This algorithm, which we shall
call \emph{model exploration by confidence}, will be introduced in
\Cref{sec:model-expl-with-1}, and its construction will mimic the argumentation used by
Baader and Distel for his model exploration algorithm.

The results presented in this section have been published previously in
\cite{Borc-LTCS-13-11}.

\section{Model Exploration with Valid GCIs}
\label{sec:model-expl-with}

In this section we shall review the argumentation used to develop model exploration, as
given in~\cite[Chapter~6]{Diss-Felix}.  In the next section, we shall then use this
argumentation presented here and generalize it to the setting of GCIs with high confidence
in finite interpretations.

Model exploration is based on the result that bases of finite interpretations
$\mathcal{I}$ can be obtained from bases of their corresponding induced formal context
$\con K_{\mathcal{I}}$ (\Cref{thm:Felix-base-B3}).  Since attribute exploration arises
from the computation of the canonical base by adding suitable expert interaction (see
\Cref{sec:attr-expl}), one could think of obtaining an algorithm for model exploration by
adding suitable expert interaction during the computation of bases of $\con
K_{\mathcal{I}}$.  It shall turn out that this is indeed correct.

There is a technical problem, however, which does not arise in attribute exploration: when
we add counterexamples to our current working interpretation $\mathcal{I}$ during model
exploration, then the attribute set $M_{\mathcal{I}}$ of the corresponding induced context
$\con K_{\mathcal{I}}$ may change, since it depends on the elements of $\mathcal{I}$.
Recall that $M_{\mathcal{I}}$ was defined as
\begin{equation*}
  M_{\mathcal{I}} = N_{C} \cup \set{ \bot } \cup \set{ \exists r. X^{\mathcal{I}} \mid r
    \in N_{R}, X \subseteq \Delta^{\mathcal{I}}, X \neq \emptyset }.
\end{equation*}

Thus, to allow to use attribute exploration as a basis for model exploration we need to
fix the attribute set, and the best way for this would be to use
$M_{\mathcal{I}_{\mathsf{back}}}$.  However, since we cannot access the background
interpretation $\mathcal{I}_{\mathsf{back}}$ directly, we cannot compute this set
completely.  On the other hand, it can be shown that we can compute the set
$M_{\mathcal{I}_{\mathsf{back}}}$ \emph{incrementally}, using the fact that the expert
confirms certain types of GCIs, and then use the parts of
$M_{\mathcal{I}_{\mathsf{back}}}$ we already know for the exploration process.

To explain how this can be done, we shall first discuss in \Cref{sec:grow-sets-attr} how
we can compute bases of formal contexts where the set of attributes is allowed to grow
during the computation.  Thereafter, we shall see in \Cref{sec:comp-bases-given} how we
can transfer this algorithm to the setting of computing finite bases of finite
interpretations $\mathcal{I}$, thus allowing the set $M_{\mathcal{I}}$ to be computed
successively during the computation.  Finally, we shall see in \Cref{sec:an-algor-expl}
how we can add expert interaction to avoid direct access to the underlying interpretation,
thus obtaining the model exploration algorithm.

\subsection{Growing Sets of Attributes}
\label{sec:grow-sets-attr}

We want to find an algorithm that allows us to compute bases of formal contexts where the
attribute set is allowed to grow during the computation.  We can think of this situation
as follows: we want to compute a base of a formal context, which however we cannot access
completely, in the sense that some of the attributes in this formal context are
\enquote{hidden}.  However, during the computation of the base, hidden attributes are
uncovered incrementally.  The goal is then to find an algorithm which allows us to compute
bases in such a setting.

Obtaining such an algorithm is actually not that difficult.  For this let us consider how
\Cref{alg:canonical-base} computes the canonical base.  There, we use the Next-Closure
algorithm to enumerate the premises of the canonical base of a given formal context $\con
K = (G, M, I)$, using some linear order $\leq_{M}$ on $M$.  If $M = \set{ m_{1}, \dots,
  m_{n} }$ and
\begin{equation*}
  m_{n} \leq_{M} m_{n-1} \leq_{M} \dots \leq_{M} m_{1},
\end{equation*}
then the Next Closure algorithm firstly enumerates all premises which are subsets of
$\emptyset$, then those which are subsets of $\set{ m_{1} }$, then those of $\set{ m_{1},
  m_{2} }$, and so on.  In particular, it will not consider an element $m_{k} \in M$
before it has enumerated all premises which are subsets of $\set{ m_{1}, \dots, m_{k-1}
}$.

We can exploit this idea for our purpose of computing bases with growing sets of
attributes: if the attribute set in iteration $i$ is $M_{i}$, ordered by $\leq_{M_{i}}$,
and we are about to add some new attributes $m_{1}, \dots, m_{n}$ to $M_{i}$ to obtain
\begin{equation*}
  M_{i+1} := M_{i} \cup \set{ m_{1}, \dots, m_{n} },
\end{equation*}
then we define the linear order $\leq_{M_{i+1}}$ on $M_{i+1}$ by ordering the elements in
$M_{i} \subseteq M_{i+1}$ as before, \ie
\begin{equation}
  \label{eq:48}
  \leq_{M_{i}} = {\leq_{M_{i+1}}} \cap M_{i} \times M_{i},
\end{equation}
and requiring in addition that
\begin{equation}
  \label{eq:49}
  m_{j} \leq_{M_{i+1}} x
\end{equation}
is true for all $j \in \set{ 1, \dots, n }$ and $x \in M_{i}$.  In other words, we just
put the new elements \emph{before} the old elements.  In that way, the Next-Closure
behaves as if the elements would have been there from the start, and computes the base as
desired.

\addfunctionname{base/growing-set-of-attributes}

\begin{figure}[tp]
  \begin{Algorithm}[Algorithm 8 from~\cite{Diss-Felix}]~ Computing a Base of a Formal
    Context with Growing Sets of Attributes and Background Knowledge%
    \label{alg:base/growing-set-of-attributes}
\begin{lstlisting}
define base/growing-set-of-attributes($\con K = (G, M, I)$, $\leq_{M}$, $\mathcal{S} \subseteq \Th(\con K)$)
  $i$ := 0
  $P_i$ := $\emptyset$
  $\mathcal{K}_i$ := $\emptyset$
  $\con K_i$ := $\con K$
  $M_i$ := $M$
  $\mathcal{S}_i$ := $\mathcal{S}$
  $\leq_{M_i}$ := $\leq_{M}$

  forever do
    read $\con K_{i+1} = (G, M_{i+1}, I_{i+1}) \text{ such that } M_i \subseteq M_{i+1}
\text{ and } I_i = I_{i+1} \cap M_i \times M_i$
    read $\mathcal{S}_{i+1} \text{ such that } \mathcal{S}_i \subseteq \mathcal{S}_{i+1} \subseteq \Th(\con K_{i+1})$
    choose ${\leq_{M_{i+1}}} \text{ such that (\ref{eq:48}) and (\ref{eq:49}) hold.}$

    $\mathcal{K}_{i+1}$ := $\set{ P_r \to (P_r)_{\con K_{i+1}}'' \mid P_r \neq
      (P_r)_{\con K_{i+1}}'', r \in \set{ 0, \ldots, i } }$

    $P_{i+1}$ := next-closure($M_{i+1}$, $\leq_{M_{i+1}}$, $P_{i}$, $\mathcal{K}_{i+1} \cup \mathcal{S}_{i+1})$
    if $P_{i+1} =$ nil exit

    $i$ := $i + 1$  
  end

  return $\mathcal{K}_i$  
end    
\end{lstlisting}  
  \end{Algorithm}
\end{figure}

An implementation of this idea is shown in \Cref{alg:base/growing-set-of-attributes}.
There we start with some initial formal context $\con K_{0} = \con K = (G, M, I)$ and some
background knowledge $\mathcal{S}_{0} = \mathcal{S} \subseteq \Th(\con K)$.  Then, in
every iteration we allow to extend the current context $\con K_{i} = (G, M_{i}, I_{i})$ by
providing a new set $M_{i+1} \supseteq M_{i}$ of attributes and a new incidence relation
$I_{i+1} \subseteq M_{i+1} \times M_{i+1}$ which satisfies
\begin{equation*}
  I_{i} = I_{i+1} \cap M_{i} \times M_{i}.
\end{equation*}
This corresponds to our perception that at the beginning, some of the attributes are
hidden, and are uncovered during the run of the algorithm.

For this algorithm to make sense, we of course need to require that at a certain point all
of the formal context has been uncovered, \ie that for some $\ell \in \NN_{\geq 0}$ it is
true that $M_{\ell} = M_{i}$ for all $i \geq \ell$.  From this point on,
\Cref{alg:base/growing-set-of-attributes} behaves as \Cref{alg:canonical-base} for
computing the canonical base of a given formal context.

\begin{Theorem}[Theorems~6.2 and~6.3 from~\cite{Diss-Felix}]
  \label{thm:base-with-growing-set-of-attributes}
  Let $\con K = (G, M, I)$ be a finite formal context, $\leq_{M}$ a linear order on $M$,
  and $\mathcal{S} \subseteq \Th(\con K)$.  Then in a run of
  \Cref{alg:base/growing-set-of-attributes}, let $\ell \in \NN_{\ge 0}$ be such that
  $M_{\ell} = M_{i}$ for all $i \geq \ell$.  Then this run terminates.  If $n$ is the last
  iteration of this run, then $\mathcal{K}_{n}$ is a base of $\con K_{n}$ with background
  knowledge $\mathcal{S}_{n}$.
\end{Theorem}

A difference to the classical computation of the canonical base as shown in
\Cref{alg:canonical-base} is that in the latter we only consider sets $P$ as premises for
implications which are closed under the currently known implications, but are not intents
of the given formal context.  In contrast to this,
\Cref{alg:base/growing-set-of-attributes} considers all sets $P_{i}$ which are closed
under the currently known implications, no matter whether they are intents of $\con
K_{i}$.  The reason for this that even if $P_{i}$ is an intent of $\con K_{i}$, it could
very well be that $P_{i}$ is not an intent of $\con K_{n}$ (where $n$ is the number of
iterations of the algorithm) because of attributes which have been introduced in $\con
K_{n}$, but were not present in $\con K_{i}$.  Since we cannot know whether $P_{i}$ will
be an intent of $\con K_{n}$ or not when we compute it, we have to consider it as well.
Otherwise, we cannot guarantee that $\mathcal{K}_{n}$ will be a base of $\con K_{n}$.

Unfortunately, the fact that we have to keep all those sets $P_{i}$ may lead to
$\mathcal{K}_{n}$ not being irredundant anymore.  This has been illustrated
in~\cite{Diss-Felix} by the following example.

\begin{Example}[Example~6.1 from~\cite{Diss-Felix}]
  \label{exp:non-redundant-bases}
  %
  \begin{figure}[tp]
    \centering
    \begin{equation*}
      \con K_0 = \con K_1 =
      \begin{array}{c | c}
        ~ & \mathsf{A} \\
        \midrule
        1 & \times \\
        2 & 
      \end{array}
      \qquad
      \con K_2 = \con K_3 =
      \begin{array}{c|cc}
        ~ & \mathsf{A} & \mathsf{B} \\
        \midrule
        1 & \times & \\
        2 &        & \times
      \end{array}
      \qquad
      \con K_3 = \con K_4 = \con K_5 =
      \begin{array}{c|ccc}
        ~ & \mathsf{A} & \mathsf{B} & \mathsf{C}\\
        \midrule
        1 & \times & & \times \\
        2 & & \times &
      \end{array}
    \end{equation*}
    \caption{Formal Contexts for \Cref{exp:non-redundant-bases}}
    \label{fig:example-context-1}
  \end{figure}
  %
  We consider the following run of \Cref{alg:base/growing-set-of-attributes} with input
  $\con K = \con K_0$ as shown in \Cref{fig:example-context-1}, and $\mathcal{S} =
  \mathcal{S}_0 = \emptyset = \mathcal{S}_1 = \ldots = \mathcal{S}_6$:
  \begin{equation*}
    \begin{array}{c|ccc}
      k & M_{k+1} \setminus M_k & \mathcal{L}_k & P_k \\
      \midrule
      0 & \emptyset      & \emptyset & \emptyset         \\
      1 & \emptyset      & \emptyset & \set{\mathsf{A}}    \\
      2 & \set{\mathsf{B}} & \emptyset & \set{\mathsf{B}}    \\
      3 & \emptyset      & \emptyset & \set{\mathsf{A}, \mathsf{B}} \\
      4 & \set{\mathsf{C}} & \set{\set{\mathsf{A}} \to \set{\mathsf{A},\mathsf{C}},
        \set{\mathsf{A},\mathsf{B}} \to \set{\mathsf{A}, \mathsf{B}, \mathsf{C}}} & \set{\mathsf{C}}\\
      5 & \emptyset      & \set{\set{\mathsf{A}} \to \set{\mathsf{A},\mathsf{C}},
        \set{\mathsf{A},\mathsf{B}} \to \set{\mathsf{A}, \mathsf{B}, \mathsf{C}}, \set{\mathsf{C}} \to
        \set{\mathsf{A},\mathsf{C}}} & \set{\mathsf{A}, \mathsf{B}, \mathsf{C}}\\
      6 & \emptyset & \set{\set{\mathsf{A}} \to \set{\mathsf{A},\mathsf{C}},
        \set{\mathsf{A},\mathsf{B}} \to \set{\mathsf{A}, \mathsf{B}, \mathsf{C}}, \set{\mathsf{C}} \to
        \set{\mathsf{A},\mathsf{C}}} & \text{\textbf{nil}}
    \end{array}
  \end{equation*}
  In iterations 2 and 4, the new attributes B and C are added, as shown in
  Figure~\ref{fig:example-context-1}.  The algorithm terminates in iteration 6 with output
  $\mathcal{L}_6$, which is clearly non-redundant: the implication $\set{\mathsf{A},
    \mathsf{B}} \to \set{\mathsf{A}, \mathsf{B}, \mathsf{C}}$ is entailed by
  $\set{\mathsf{A}} \to \set{\mathsf{A}, \mathsf{C}}$.
\end{Example}

\subsection{Computing Bases of Given Finite Interpretations}
\label{sec:comp-bases-given}

We now want to use \Cref{alg:base/growing-set-of-attributes} to devise an algorithm that
allows us to compute bases of finite interpretations $\mathcal{I}$ without computing
$M_{\mathcal{I}}$ first.  Instead, we want that the elements of the set $M_{\mathcal{I}}$
are computed successively during the run of the algorithm.  In this way, we can
immediately start with computing valid GCIs of $\mathcal{I}$, and do not have to wait for
$M_{\mathcal{I}}$ to be computed completely.

The successive computation of the elements of $M_{\mathcal{I}}$ is achieved in
\Cref{alg:base/growing-set-of-attributes} by defining the sets $M_{i}$ as follows.  For $i
= 0$, we define
\begin{equation*}
  M_{0} = N_{C} \cup \set{ \bot }.
\end{equation*}
Then, during the run of the algorithm, we add elements of the form $\exists
r. X^{\mathcal{I}}$ for $r \in N_{R}$ and $X \subseteq \Delta^{\mathcal{I}}, X \neq
\emptyset$.  More precisely, whenever we compute a set $P_{i}$ in
\Cref{alg:base/growing-set-of-attributes}, we define
\begin{equation*}
  M_{i+1} := M_{i} \cup \set{ \exists r. (\bigsqcap P_{i})^{\mathcal{I}\mathcal{I}} \mid r
    \in N_{R} },
\end{equation*}
where the union is only up to equivalence, \ie if for some $C:= \exists r. (\bigsqcap
P_{i})^{\mathcal{I}\mathcal{I}}$ there already exists $D \in M_{i}$ such that $C \equiv
D$, then we do not add $C$ in the definition of $M_{i+1}$.

We also need to specify how we define the formal contexts $\con K_{i}$ and the sets
$\mathcal{S}_{i}$ of background knowledge.  We set $\con K_{i}$ to be the induced formal
context of $M_{i}$ and $\mathcal{I}$, and we define
\begin{equation*}
  \mathcal{S}_{i+1} = \set{ \set{A} \to \set{B} \mid A, B \in M_{i}, A \sqsubseteq B }.
\end{equation*}
The resulting algorithm is shown in \Cref{alg:base-of-interpretation}.

\addfunctionname{base-of-interpretation,induced-context}

\begin{figure}[tp]
  \begin{Algorithm}[Algorithm 9 from~\cite{Diss-Felix}]~ Computing a Base of a Given
    Interpretation with Incremental Computation of $M_{\mathcal{I}}$%
    \label{alg:base-of-interpretation}
    \begin{lstlisting}
define base-of-interpretation($\mathcal{I} = (\Delta^{\mathcal{I}}, \cdot^{\mathcal{I}})
\text{ over } N_{C} \text{ and } N_{R}$)
  $i$ := 0
  $P_i$ := $\emptyset$
  $M_i$ := $N_C \cup \set{\bot}$
  $\mathcal{K}_i$ := $\emptyset$
  $\mathcal{S}_i$ := $\set{\set{\bot} \to \set{A} \mid A \in N_C}$
  choose ${\leq_{M_{i}}} \text{ as a linear order on } M_{i}$

  forever do
    $M_{i+1}$ := $M_i \cup \set{ \exists r.(\bigsqcap P_i)^{\mathcal{I}\mathcal{I}} \mid r \in N_R}\label{lst:base-of-interpretation-1}$
    $\con K_{i+1}$ := induced-context($\mathcal{I}$, $M_{i+1}$)$\label{lst:base-of-interpretation-2}$
    $\mathcal{K}_{i+1}$ := $\set{ P_r \to (P_r)_{\con K_{i+1}}'' \mid P_r \neq
      (P_r)_{\con K_{i+1}}'', r \in \set{ 0, \ldots, i} }$
    $\mathcal{S}_{i+1}$ := $\set{ \set{A} \to \set{B} \mid A, B \in M_{i+1}, A \sqsubseteq
      B }$
    choose ${\leq_{M_{i+1}}} \text{ such that (\ref{eq:48}) and (\ref{eq:49}) hold.}$

    $P_{i+1}$ := next-closure($M_{i+1}$, $\leq_{M_{i+1}}$, $P_i$, $\mathcal{K}_{i+1} \cup \mathcal{S}_{i+1}$)
    if $P_{i+1} =$ nil exit

    $i$ := $i + 1$  
  end

  return $\set{ \bigsqcap P \sqsubseteq (\bigsqcap P)^{\mathcal{I}\mathcal{I}} \mid (P \to
    P_{\con K_{i+1}}'') \in \mathcal{K}_{i+1} }$
end    
    \end{lstlisting}  
  \end{Algorithm}
\end{figure}

Note that \Cref{alg:base-of-interpretation} has the form of
\Cref{alg:base/growing-set-of-attributes}, and thus we can argue that a run of
\Cref{alg:base-of-interpretation} terminates with finite sets $N_{C}, N_{R}$ and a finite
interpretation $\mathcal{I}$ as input.  In this case, all concept descriptions which are
added to the set of attributes during the run of the algorithm are, up to equivalence,
elements of $M_{\mathcal{I}}$, which is finite.  Thus, there exists a number $\ell \in
\NN_{\ge 0}$ such that for all $i \geq \ell$ it is true that $M_{i} = M_{\ell}$.  Then, by
\Cref{thm:base-with-growing-set-of-attributes}, \Cref{alg:base-of-interpretation} has to
terminate.

To see that the definition of $M_{i}$ will eventually yield all elements of
$M_{\mathcal{I}}$, up to equivalence, we first observe that $M_{i} \subseteq
M_{\mathcal{I}}$ is true up to equivalence for all iterations $i$ of
\Cref{alg:base-of-interpretation}.  On the other hand, if $\exists r. X^{\mathcal{I}} \in
M_{\mathcal{I}}$, then $X^{\mathcal{I}} \equiv X^{\mathcal{I}\mathcal{I}\mathcal{I}} =
(X^{\mathcal{I}})^{\mathcal{I}\mathcal{I}}$, and $X^{\mathcal{I}}$ is expressible in terms
of $M_{\mathcal{I}}$ by \Cref{lem:mmsc-are-expressible-in-terms-of-M_I}.  Therefore, there
exists $U \subseteq M_{\mathcal{I}}$ such that
\begin{equation*}
  X^{\mathcal{I}} \equiv \bigsqcap U.
\end{equation*}
If $n$ is the number of iterations of the algorithm, and if $\con K_{n}$ denotes the
induced context of $M_{n}$ and $\mathcal{I}$, then we find
\begin{equation*}
  (\bigsqcap U_{\con K_{n}}'')^{\mathcal{I}} = U_{\con K_{n}}''' = U_{\con K_{n}}' =
  (\bigsqcap U)^{\mathcal{I}}
\end{equation*}
using \Cref{prop:connection-I-prime-2}.  Then
\begin{equation*}
  \exists r. X^{\mathcal{I}} \equiv \exists r. (X^{\mathcal{I}})^{\mathcal{I}\mathcal{I}}
  \equiv \exists r.(\bigsqcap U_{\con K_{n}}'')^{\mathcal{I}\mathcal{I}}.
\end{equation*}
Thus, it suffices to consider only intents of the final context $\con K_{n}$.  The
following result shows that these intents are among the sets $P_{i}$.

\begin{Lemma}[Partly Lemma~6.3 from~\cite{Diss-Felix}]
  \label{lem:Felix-6.3}
  Consider a terminating run of \Cref{alg:base-of-interpretation} with $n$ iterations, and
  let $Q \subseteq M_{n}$.  Then if $Q = Q_{\con K_{n}}''$, then $Q = P_{i}$ for some $i
  \in \set{ 0, \dots, n }$.
\end{Lemma}

Using this lemma we can show that $M_{n}$ is, up to equivalence, equal to
$M_{\mathcal{I}}$.  It can then be shown that a base of the induced context of
$\mathcal{I}$ and $M_{n}$ yields a base of $\mathcal{I}$ as
well~\cite[Corollary~5.14]{Diss-Felix}.  From this, we immediately obtain the correctness
of \Cref{alg:base-of-interpretation}.

\begin{Theorem}[Theorem~6.9 from~\cite{Diss-Felix}]
  \label{thm:Felix-6.9}
  Let $\mathcal{I}$ be a finite interpretation over $N_{C}$ and $N_{R}$.  Then the set
  \begin{equation*}
    \mathcal{K} = \text{\lstinline{base-of-interpretation}}(\mathcal{I})
  \end{equation*}
  is a finite base of $\mathcal{I}$.
\end{Theorem}

\subsection{An Algorithm for Exploring Interpretations}
\label{sec:an-algor-expl}

Based on \Cref{alg:base-of-interpretation}, we now want to discuss an algorithm for model
exploration.  For this, recall that during model exploration we suppose that our domain of
interest can be represented by a finite interpretation $\mathcal{I}_{\mathsf{back}}$, the
background interpretation of the exploration.  If we could access
$\mathcal{I}_{\mathsf{back}}$ directly, then to explore $\mathcal{I}_{\mathsf{back}}$
would just mean to compute a base of it, which we could achieve by using
\Cref{alg:base-of-interpretation}.  However, as already discussed, we assume that
$\mathcal{I}_{\mathsf{back}}$ cannot be accessed directly, but instead is represented by
an expert.

To turn \Cref{alg:base-of-interpretation} into an algorithm that allows us to compute a
base of $\mathcal{I}_{\mathsf{back}}$ using only the expert as a means to access this
interpretation, we want to replace every direct access to $\mathcal{I}_{\mathsf{back}}$ in
\Cref{alg:base-of-interpretation} by a suitable expert interaction.  For this we observe
that there are two places in \Cref{alg:base-of-interpretation} that directly access the
given interpretation:
\begin{enumerate}[i. ]
\item when computing concept descriptions of the form $\exists r.(\bigsqcap
  P_{i})^{\mathcal{I}_{\mathsf{back}}}$ (line~\ref{lst:base-of-interpretation-1} of
  \Cref{alg:base-of-interpretation}),
\item when computing $\con K_{i}$ as induced context of $M_{i}$ and
  $\mathcal{I}_{\mathsf{back}}$ (line~\ref{lst:base-of-interpretation-2} of
  \Cref{alg:base-of-interpretation}).
\end{enumerate}
The computation of $\con K_{i}$ we can fix easily by instead of computing the induced
context of $M_{i}$ and $\mathcal{I}_{\mathsf{back}}$, we just compute the induced context
$M_{i}$ and the current working interpretation of the exploration process.  For computing
$\exists r. (\bigsqcap P_{i})^{\mathcal{I}_{\mathsf{back}}}$, however, we need to use the
expert.

For this, we need to consider another issue first, which we have already talked about in
the introduction, namely the way the experts specifies counterexamples during the
exploration.  We had argued that if the expert gives an element $x \in
\Delta^{\mathcal{I}_{\mathsf{back}}}$ from the background interpretation as a
counterexample, then she also has to include all corresponding concept names and role
successors $x$ has in $\mathcal{I}_{\mathsf{back}}$.  Otherwise, there may exist the
danger that the provided counterexamples invalidate GCIs which are actually valid in the
background interpretation $\mathcal{I}_{\mathsf{back}}$.

To makes this more formal, Distel introduced the notion of a \emph{connected
  subinterpretation}.

\begin{Definition}[Connected Subinterpretations; Definition~6.1 from~\cite{Diss-Felix}]
  \label{def:connected-subinterpretations}
  Let $\mathcal{I} = (\Delta^{\mathcal{I}}, \cdot^{\mathcal{I}})$ be a finite
  interpretation over $N_{C}$ and $N_{R}$.  Define%
  \def\succop{\operatorname{succ}}%
  \def\nameop{\operatorname{names}}%
  \begin{align*}
    \nameop_{\mathcal{I}}(x) &:= \set{ C \in N_{C} \mid x \in C^{\mathcal{I}} },\\
    \succop_{\mathcal{I}}(x, r) &:= \set{ y \in \Delta^{\mathcal{I}} \mid (x, y) \in
      r^{\mathcal{I}} },
  \end{align*}
  for $x \in \Delta^{\mathcal{I}}$ and $r \in N_{R}$.  An interpretation $\mathcal{J} =
  (\Delta^{\mathcal{J}}, \cdot^{\mathcal{J}})$ over $N_{C}$ and $N_{R}$ is called a
  \emph{subinterpretation} of $\mathcal{I}$ if and only if
  \begin{enumerate}[i. ]
  \item $\Delta^{\mathcal{J}} \subseteq \Delta^{\mathcal{I}}$,
  \item $\nameop_{\mathcal{I}}(x) = \nameop_{\mathcal{J}}(x)$ for all $x \in
    \Delta^{\mathcal{J}}$, and
  \item $\succop_{\mathcal{J}}(x,r) \subseteq \succop_{\mathcal{I}}(x,r)$ for all $x \in
    \Delta^{\mathcal{J}}, r \in N_{R}$.
  \end{enumerate}
  $\mathcal{J}$ is called a \emph{connected subinterpretation} of $\mathcal{I}$ if
  $\mathcal{J}$ is a subinterpretation of $\mathcal{I}$, and in addition it is true that
  \begin{equation*}
    \succop_{\mathcal{J}}(x, r) = \succop_{\mathcal{I}}(x, r)
  \end{equation*}
  is true for all $x \in \Delta^{\mathcal{J}}, r \in N_{R}$.  In this case we shall say
  that $\mathcal{I}$ \emph{extends} $\mathcal{J}$.
\end{Definition}

If we now ensure that during the exploration process the current working interpretation is
a connected subinterpretation of the background interpretation
$\mathcal{I}_{\mathsf{back}}$, then we can guarantee that counterexamples provided by the
expert do not accidentally invalidate valid GCIs.  This can be achieved by adding
counterexamples only as connected subinterpretations of $\mathcal{I}_{\mathsf{back}}$ to
our current working interpretation.

\begin{Lemma}[Lemma~6.12 from~\cite{Diss-Felix}]
  \label{lem:Felix-6.12}
  Let $\mathcal{J} = (\Delta^{\mathcal{J}}, \cdot^{\mathcal{J}})$ be an interpretation
  over $N_{C}$ and $N_{R}$ which is a connected subinterpretation of the interpretation
  $\mathcal{I}$.  Then for all $\ELgfpbot$ concept descriptions $C$ over $N_{C}$ and
  $N_{R}$ it is true that
  \begin{equation*}
    C^{\mathcal{J}} = C^{\mathcal{I}} \cap \Delta^{\mathcal{J}}.
  \end{equation*}
\end{Lemma}

\begin{Theorem}[Corollary~6.13 from~\cite{Diss-Felix}]
  \label{thm:GCIs-valid-in-interpretations-are-also-valid-in-connected-subinterpretations}
  Let $\mathcal{I} = (\Delta^{\mathcal{I}}, \cdot^{\mathcal{I}})$ be a finite
  interpretation over $N_{C}$ and $N_{R}$, and let $\mathcal{J} = (\Delta^{\mathcal{J}},
  \cdot^{\mathcal{J}})$ be a connected subinterpretation of $\mathcal{I}$.  Let $C, D$ be
  two $\ELgfpbot$ concept descriptions over $N_{C}$ and $N_{R}$.  Then if $C \sqsubseteq
  D$ is valid in $\mathcal{I}$, then $C \sqsubseteq D$ is also valid in $\mathcal{J}$.
\end{Theorem}
\begin{Proof}
  Since $C \sqsubseteq D$ holds in $\mathcal{I}$, it is true that $C^{\mathcal{I}}
  \subseteq D^{\mathcal{I}}$.  Using \Cref{lem:Felix-6.12} we thus obtain
  \begin{equation*}
    C^{\mathcal{J}} = C^{\mathcal{I}} \cap \Delta^{\mathcal{J}} \subseteq D^{\mathcal{J}}
    \cap \Delta^{\mathcal{J}} = D^{\mathcal{J}},
  \end{equation*}
  \ie $C \sqsubseteq D$ holds in $\mathcal{J}$, as it was claimed.
\end{Proof}

Now that we know how the expert should provide counterexamples to proposed GCIs, let us
reconsider the question of how to compute concept descriptions of the form $\exists
r. (\bigsqcap P_{i})^{\mathcal{I}_{\mathsf{back}}\mathcal{I}_{\mathsf{back}}}$.  Recall
that since we cannot $\mathcal{I}_{\mathsf{back}}$, we cannot compute this concept
description directly.  What we can compute is the concept description $\exists
r. (\bigsqcap P_{i})^{\mathcal{I}_{\ell}\mathcal{I}_{\ell}}$, where $\mathcal{I}_{\ell}$
is the currently known interpretation.  The good thing is that the expert can ensure that
$\exists r. (\bigsqcap P_{i})^{\mathcal{I}_{\mathsf{back}}\mathcal{I}_{\mathsf{back}}}$
and $\exists r. (\bigsqcap P_{i})^{\mathcal{I}_{\ell}\mathcal{I}_{\ell}}$ are equivalent.

\begin{Lemma}[Lemma~6.14 from~\cite{Diss-Felix}]
  \label{lem:Felix-6.14}
  Let $\mathcal{I}$ be a finite interpretation over $N_{C}$ and $N_{R}$, and let
  $\mathcal{J}$ be a connected subinterpretation of $\mathcal{I}$.  Then for all
  $\ELgfpbot$ concept descriptions $C$ over $N_{C}$ and $N_{R}$, it is true that if $C
  \sqsubseteq C^{\mathcal{J}\mathcal{J}}$ is valid in $\mathcal{I}$, then
  $C^{\mathcal{I}\mathcal{I}} \equiv C^{\mathcal{J}\mathcal{J}}$.
\end{Lemma}

If we choose $\mathcal{I} = \mathcal{I}_{\mathsf{back}}$, $\mathcal{J} =
\mathcal{I}_{\ell}$ and $C = \bigsqcap P_{i}$ in the previous lemma we see that if the
expert confirms the GCI
\begin{equation*}
  \bigsqcap P_{i} \sqsubseteq (\bigsqcap P_{i})^{\mathcal{I}_{\ell}\mathcal{I}_{\ell}},
\end{equation*}
then $(\bigsqcap P_{i})^{\mathcal{I}_{\ell}\mathcal{I}_{\ell}} \equiv (\bigsqcap
P_{i})^{\mathcal{I}_{\mathsf{back}}\mathcal{I}_{\mathsf{back}}}$, just as we need it.

On the other hand, if the expert rejects $\bigsqcap P_{i} \sqsubseteq (\bigsqcap
P_{i})^{\mathcal{I}_{\ell}\mathcal{I}_{\ell}}$, then she adds counterexamples to the
current working interpretation $\mathcal{I}_{\ell}$ to yield a new working interpretation
$\mathcal{I}_{\ell+1}$.  If $\bigsqcap P_{i} \not\sqsubseteq (\bigsqcap
P_{i})^{\mathcal{I}_{\ell+1}\mathcal{I}_{\ell+1}}$, then the GCI
\begin{equation*}
  \bigsqcap P_{i} \sqsubseteq (\bigsqcap P_{i})^{\mathcal{I}_{\ell+1}\mathcal{I}_{\ell+1}}
\end{equation*}
is again proposed to the expert.  If $\bigsqcap P_{i} \sqsubseteq (\bigsqcap
P_{i})^{\mathcal{I}_{\ell+1}\mathcal{I}_{\ell+1}}$, \ie $\bigsqcap P_{i} \equiv (\bigsqcap
P_{i})^{\mathcal{I}_{\ell+1}\mathcal{I}_{\ell+1}}$, then the next set $P_{i+1}$ is
considered.

\addfunctionname{model-exploration}

\begin{figure}[tp]
  \begin{Algorithm}[Algorithm 11 from~\cite{Diss-Felix}]~ A Model Exploration Algorithm%
    \label{alg:model-exploration}
    \begin{lstlisting}
define model-exploration($\mathcal{I} = (\Delta^{\mathcal{I}}, \cdot^{\mathcal{I}}) \text{ over } N_{C} \text{ and } N_{R}$)
  $i$ := 0
  $P_i$ := $\emptyset$
  $M_i$ := $N_C \cup \set{\bot}$
  $\mathcal{K}_i$ := $\emptyset$
  $\mathcal{S}_i$ := $\set{\set{\bot} \to \set{A} \mid A \in N_C}$
  choose ${\leq_{M_{i}}} \text{ as a linear order on } M_{i}$
  $\ell$ := 0
  $\mathcal{I}_{\ell}$ := $\mathcal{I}$
  
  forever do
    ;; expert interaction
    while $\text{expert refutes } \bigsqcap P_i \sqsubseteq (\bigsqcap P_i)^{\mathcal{I}_\ell\mathcal{I}_\ell}$ do
      $\mathcal{I}_{\ell+1}$ := $\text{new interpretation such that }$
        - $\mathcal{I}_{\ell+1} \text{ extends } \mathcal{I}_\ell$
        - $\mathcal{I}_{\ell+1} \text{ contains counterexamples for } \bigsqcap P_i \sqsubseteq
        (\bigsqcap P_i)^{\mathcal{I}_\ell\mathcal{I}_\ell}$
      $\ell$ := $\ell + 1$
    end

    ;; add new attributes (up to equivalence)
    $M_{i+1}$ := $M_i \cup \set{ \exists r.(\bigsqcap P_i)^{\mathcal{I}_{\ell}\mathcal{I}_{\ell}} \mid r \in N_R}$

    ;; update $\con K_{i+1}, \mathcal{S}_{i+1}$ and $\mathcal{L}_{i+1}$
    $\con K_{i+1}$ := induced-context($\mathcal{I}_{\ell}$, $M_{i+1}$)
    $\mathcal{K}_{i+1}$ := $\set{ P_r \to (P_r)_{\con K_{i+1}}'' \mid P_r \neq
      (P_r)_{\con K_{i+1}}'', r \in \set{ 0, \ldots, i} }$
    $\mathcal{S}_{i+1}$ := $\set{ \set{A} \to \set{B} \mid A, B \in M_{i+1}, A \sqsubseteq
      B }$
    choose ${\leq_{M_{i+1}}} \text{ such that (\ref{eq:48}) and (\ref{eq:49}) hold}$

    ;; next closed set
    $P_{i+1}$ := next-closure($M_{i+1}$, $\leq_{M_{i+1}}$, $P_i$, $\mathcal{K}_{i+1} \cup \mathcal{S}_{i+1}$)
    if $P_{i+1} =$ nil exit

    $i$ := $i + 1$
  end

  return $\set{ \bigsqcap P \sqsubseteq (\bigsqcap P)^{\mathcal{I}_{\ell}\mathcal{I}_{\ell}} \mid (P \to
    P_{\con K_{i+1}}'') \in \mathcal{K}_{i+1} }$
end    
    \end{lstlisting}  
  \end{Algorithm}
\end{figure}

We are now able to adapt \Cref{alg:base-of-interpretation} by replacing all references to
the background interpretation by expert interactions.  The result is shown in
\Cref{alg:model-exploration}.  From our previous discussion we now easily obtain the
following result.

\begin{Theorem}[Theorem~6.16 from~\cite{Diss-Felix}]
  \label{thm:Felix-6.16}
  Let $\mathcal{I}_{\mathsf{back}}$ be a finite interpretation, and let $\mathcal{I}$ be a
  connected subinterpretation of $\mathcal{I}_{\mathsf{back}}$.  Then
  \Cref{alg:model-exploration} applied to $\mathcal{I}$, using
  $\mathcal{I}_{\mathsf{back}}$ as background interpretation, terminates after finitely
  many steps.  If $n$ is the number of iterations in this run, and if $\mathcal{I}_{\ell}$
  is the final working interpretation, then the set
  \begin{equation*}
    \set{ \bigsqcap P \sqsubseteq (\bigsqcap P)^{\mathcal{I}_{\ell}\mathcal{I}_{\ell}}
      \mid (P \to P_{\con K_{n+1}}'') \in \mathcal{K}_{n+1} }
  \end{equation*}
  is a finite base of $\mathcal{I}_{\mathsf{back}}$.
\end{Theorem}

\section{Model Exploration with Confident GCIs}
\label{sec:model-expl-with-1}

In the previous section we have seen how we can obtain an algorithm for model exploration
by extending Distel's results on computing finite bases of finite interpretations.  In
this section we want to generalize this argumentation to the setting of GCIs with high
confidence, \ie we want to obtain an algorithm for model exploration which not only asks
GCIs which are valid in the current working interpretation, but which is also allowed to
ask GCIs whose confidence in the original data is just high enough.  This process we shall
call \emph{model exploration by confidence}.

Distel's argumentation for his model exploration algorithm is based on his method to
obtain finite bases of finite interpretations.  However, in model exploration by
confidence we do not compute bases of the current working interpretation
$\mathcal{I}_{\ell}$, but instead compute bases of $\Th_{c}(\mathcal{I}_{\ell})$, for some
chosen threshold $c \in [0,1]$.  Moreover, the interpretation $\mathcal{I}_{\ell}$
contains a connected subinterpretation consisting of the counterexamples given by the
expert, and all GCIs which are not valid within this subinterpretation should not be
considered further, even if they have a confidence above $c$ in the initial working
interpretation.

We can thus think of $\mathcal{I}_{\ell}$ as consisting of two parts: the initial working
interpretation $\mathcal{I}$, which may contain errors and where we apply our confidence
heuristics, and a connected subinterpretation $\mathcal{I}_{\ell} \setminus \mathcal{I}$,
consisting of the counterexamples given by the expert, where we only consider valid GCIs.
We can think of all elements of $\mathcal{I}$ as \emph{untrusted}, and of all elements of
$\mathcal{I}_{\ell} \setminus \mathcal{I}$ as \emph{trusted}.

To generalize Distel's argumentation for model exploration to GCIs with high confidence,
we shall thus start by devising an algorithm that allows us to compute finite
interpretation containing trusted and untrusted elements, \ie that computes bases of
\begin{equation*}
  \Th_{c}(\mathcal{I}) \cap \Th(\mathcal{I}_{\ell} \setminus \mathcal{I}).
\end{equation*}
We shall do this in \Cref{sec:trust-untr-indiv}.

Thereafter, we shall follow the argumentation of the previous section.  This means that in
\Cref{sec:grow-sets-attr-1} we shall discuss an algorithm that computes bases of formal
contexts containing trusted and untrusted objects, and where the attribute set is allowed
to grow during the computation.  Thereafter, we shall discuss in
\Cref{sec:grow-sets-attr-1} how we can adapt this algorithm to compute bases of finite
interpretation that contain trusted and untrusted individuals, and where the set
$M_{\mathcal{I}}$ is computed on the fly during the run of the algorithm.  Finally, we
shall see in \Cref{sec:expl-conf-gcis} how we can introduce suitable expert interaction
into this last algorithm to obtain an algorithm for model exploration by confidence.

\subsection{Bases of Finite Interpretations with Untrusted Elements}
\label{sec:trust-untr-indiv}

Let $\mathcal{J}$ be a finite interpretation over $N_{C}$ and $N_{F}$, and let
$\mathcal{I}$ be a subinterpretation of $\mathcal{J}$.  As already discussed, we want to
think of $\mathcal{I}$ as the interpretation of \emph{untrusted} elements, and of the
interpretation
\begin{equation*}
  \mathcal{J} \setminus \mathcal{I} := (\Delta^{\mathcal{J}} \setminus
  \Delta^{\mathcal{I}}, \cdot^{\mathcal{J} \setminus \mathcal{I}})
\end{equation*}
as the interpretation of \emph{trusted} elements (provided by the expert), where we define
\begin{align*}
  A^{\mathcal{J} \setminus \mathcal{I}} &:= A^{\mathcal{J}} \cap \Delta^{\mathcal{J}}
  \setminus \Delta^{\mathcal{I}} = A^{\mathcal{J}} \setminus \Delta^{\mathcal{I}},\\
  r^{\mathcal{J} \setminus \mathcal{I}} &:= r^{\mathcal{J}} \cap (\Delta^{\mathcal{J}}
  \setminus \Delta^{\mathcal{I}}) \times (\Delta^{\mathcal{J}} \setminus \Delta^{\mathcal{I}})
\end{align*}
for $A \in N_{C}$ and $r \in N_{R}$.

The aim of this section is to obtain a method to find finite bases of $\mathcal{J}$ with
untrusted elements $\mathcal{I}$.  More precisely, let us define for $c \in [0,1]$ the set
\begin{multline*}
  \Th_{c}(\mathcal{J}, \mathcal{I}) := \{\, C \sqsubseteq D
    \mid C, D \in \ELgfpbot(N_{C}, N_{R}), \\C^{\mathcal{J}} \setminus \Delta^{\mathcal{I}}
    \subseteq D^{\mathcal{J}} \setminus \Delta^{\mathcal{I}}, \abs{ (C \sqcap
      D)^{\mathcal{J}} \cap \Delta^{\mathcal{I}} } \ge c \cdot \abs{ C^{\mathcal{J}} \cap
      \Delta^{\mathcal{I}} } \,\}.
\end{multline*}
We then want to describe a finite base of $\Th_{c}(\mathcal{J}, \mathcal{I})$.  The
results of this section have been published previously in~\cite{conf/dlog/Borchmann13}.

Note that we have given the confidence constraint in the form of
\begin{equation}
  \label{eq:50}
  \abs{ (C \sqsubseteq D)^{\mathcal{J}} \cap \Delta^{\mathcal{I}} } \ge c \cdot \abs{
    C^{\mathcal{J}} \cap \Delta^{\mathcal{I}} },
\end{equation}
which is the suitable formulation for our setting of $\mathcal{I}$ being a
subinterpretation of $\mathcal{J}$.  On the other hand, in our later considerations, both
$\mathcal{I}$ and $\mathcal{J} \setminus \mathcal{I}$ will be connected subinterpretations
of $\mathcal{J}$, and in this case the definition of $\Th_{c}(\mathcal{J}, \mathcal{I})$
can be simplified as follows: recall that in the case that $\mathcal{I}$ is a connected
subinterpretation of $\mathcal{J}$, \Cref{lem:Felix-6.12} yields that for all $C, D \in
\ELgfpbot(N_{C}, N_{R})$
\begin{align*}
  C^{\mathcal{J}} \cap \Delta^{\mathcal{I}} &= C^{\mathcal{I}}, \\
  (C \sqcap D)^{\mathcal{J}} \cap \Delta^{\mathcal{I}} &= (C \sqcap D)^{\mathcal{I}}.
\end{align*}
Thus, \Cref{eq:50} simplifies to
\begin{equation*}
  \abs{ (C \sqcap D)^{\mathcal{I}} } \ge c \cdot \abs{ C^{\mathcal{I}} },
\end{equation*}
which is equivalent to $\conf_{\mathcal{I}}( C \sqsubseteq D ) \ge c$.  Furthermore, since
$\mathcal{J} \setminus \mathcal{I}$ is a connected subinterpretation of $\mathcal{J}$, we
obtain again by \Cref{lem:Felix-6.12} that
\begin{equation*}
  C^{\mathcal{J} \setminus \mathcal{I}} = C^{\mathcal{J}} \cap (\Delta^{\mathcal{J}}
  \setminus \Delta^{\mathcal{I}}) = C^{\mathcal{J}} \setminus \Delta^{\mathcal{J}}
\end{equation*}
is true for all $C \in \ELgfpbot(N_{C}, N_{R})$.  Therefore, $C^{\mathcal{J}} \setminus
\Delta^{\mathcal{I}} \subseteq D^{\mathcal{J}} \setminus \Delta^{\mathcal{I}}$ is
equivalent to $C^{\mathcal{J}\setminus\mathcal{I}} \subseteq
D^{\mathcal{J}\setminus\mathcal{I}}$, \ie if $(C \sqsubseteq D) \in \Th(\mathcal{J}
\setminus \mathcal{I})$.

Thus, the definition of $\Th_{c}(\mathcal{J}, \mathcal{I})$ can now be written as
\begin{align*}
  \Th_{c}(\mathcal{J}, \mathcal{I})
  &= \{\, C \sqsubseteq D \mid C, D \in \ELgfpbot(N_{C}, N_{R}),
  C^{\mathcal{J}} \setminus \Delta^{\mathcal{I}} \subseteq D^{\mathcal{J}}
  \setminus \Delta^{\mathcal{I}}, \conf_{\mathcal{I}}( C \sqsubseteq D ) \ge c \,\} \\
  &= \set{ (C \sqsubseteq D) \in \Th(\mathcal{J} \setminus \mathcal{I}) \mid
    \conf_{\mathcal{I}} (C \sqsubseteq D) \ge c } \\
  &= \Th(\mathcal{J} \setminus \mathcal{I}) \cap \Th_{c}(\mathcal{I}),
\end{align*}
which corresponds to our intention of finding a finite base of all GCIs which are valid in
$\mathcal{J} \setminus \mathcal{I}$ and have high confidence in $\mathcal{I}$.

Let us return to the general case that $\mathcal{I}$ is just a subinterpretation of
$\mathcal{J}$.  To find a base for the set $\Th_{c}(\mathcal{J} \setminus \mathcal{I})$,
we make use of the ideas we have already used to find finite confident bases of
$\Th_{c}(\mathcal{I})$.  More precisely, we first observe that
\begin{equation*}
  \Th(\mathcal{J}) \subseteq \Th_{c}(\mathcal{J}, \mathcal{I}).
\end{equation*}
Since we can find bases of $\Th(\mathcal{J})$ using the results of Distel, we again
concentrate on finding bases of the set $\Th_{c}(\mathcal{J}, \mathcal{I}) \setminus
\Th(\mathcal{J})$.  In other words, if $\mathcal{B}$ is a base of $\Th(\mathcal{J})$, then
we seek a set $\mathcal{C} \subseteq \Th_{c}(\mathcal{J}, \mathcal{I}) \setminus
\Th(\mathcal{J})$ which is complete for $\Th_{c}(\mathcal{J}, \mathcal{I}) \setminus
\Th(\mathcal{J})$.  In this case, $\mathcal{B} \cup \mathcal{C}$ is a base of
$\Th_{c}(\mathcal{J}, \mathcal{I})$.

To find such a set $\mathcal{C}$ we first observe that
\begin{equation*}
  (C \sqsubseteq D) \in \Th_{c}(\mathcal{J}, \mathcal{I}) \iff (C^{\mathcal{J}\mathcal{J}}
  \sqsubseteq D^{\mathcal{J}\mathcal{J}}) \in \Th_{c}(\mathcal{J}, \mathcal{I}).
\end{equation*}
This is because
\begin{align*}
  C^{\mathcal{J}} &= C^{\mathcal{J}\mathcal{J}\mathcal{J}},\\
  (C \sqcap D)^{\mathcal{J}} &= (C \sqcap D)^{\mathcal{J}\mathcal{J}\mathcal{J}}
\end{align*}
is true, and thus
\begin{equation*}
  C^{\mathcal{J}} \setminus \Delta^{\mathcal{I}} \subseteq D^{\mathcal{J}} \setminus
  \Delta^{\mathcal{I}} \iff C^{\mathcal{J}\mathcal{J}\mathcal{J}} \setminus
  \Delta^{\mathcal{I}} \subseteq D^{\mathcal{J}\mathcal{J}\mathcal{J}} \setminus \Delta^{\mathcal{I}}
\end{equation*}
and \Cref{eq:50} is true if and only if
\begin{equation*}
  \abs{ (C \sqsubseteq D)^{\mathcal{J}\mathcal{J}\mathcal{J}} \cap \Delta^{\mathcal{I}} }
  \ge c \cdot \abs{ C^{\mathcal{J}\mathcal{J}\mathcal{J}} \cap \Delta^{\mathcal{I}} }.
\end{equation*}

If now $\mathcal{B}$ is a base of $\Th(\mathcal{J})$, then it is true that
\begin{equation*}
  \mathcal{B} \cup \set{ C^{\mathcal{J}\mathcal{J}} \sqsubseteq D^{\mathcal{J}\mathcal{J}}
  } \models (C \sqsubseteq D).
\end{equation*}
This is because $\mathcal{B} \models (C \sqsubseteq C^{\mathcal{J}\mathcal{J}})$, since $C
\sqsubseteq C^{\mathcal{J}\mathcal{J}}$ is valid in $\mathcal{J}$.  Furthermore,
$D^{\mathcal{J}\mathcal{J}} \sqsubseteq D$, and thus
\begin{equation*}
  \mathcal{B} \cup \set{ C^{\mathcal{J}\mathcal{J}} \sqsubseteq D^{\mathcal{J}\mathcal{J}}
  } \models (C \sqsubseteq C^{\mathcal{J}\mathcal{J}} \sqsubseteq
  D^{\mathcal{J}\mathcal{J}} \sqsubseteq D).
\end{equation*}

Having these two considerations in mind we define
\begin{equation*}
  \Conf(\mathcal{J}, c, \mathcal{I}) := \set{ X^{\mathcal{J}} \sqsubseteq
    Y^{\mathcal{J}} \mid Y \subseteq X \subseteq \Delta^{\mathcal{J}}, (X^{\mathcal{J}}
\sqsubseteq Y^{\mathcal{J}}) \in \Th_{c}(\mathcal{J}, \mathcal{I}) }.
\end{equation*}
Since $\mathcal{J}$ is a finite interpretation, $\Delta^{\mathcal{J}}$ is finite.  We
therefore obtain the following result.

\begin{Theorem}
  \label{thm:finite-base-of-interpretation-with-untrusted-individuals}
  Let $\mathcal{J}$ be a finite interpretation, let $\mathcal{I}$ be a
  subinterpretation of $\mathcal{J}$, and let $c \in [0,1]$.  Then if $\mathcal{B}$ is a
  finite base of $\mathcal{J}$, then the set
  \begin{equation*}
    \mathcal{B} \cup \Conf(\mathcal{J}, c, \mathcal{I})
  \end{equation*}
  is a finite base of $\Th_{c}(\mathcal{J}, \mathcal{I})$.
\end{Theorem}

This result already solves our initial problem of finding a finite base of
$\Th_{c}(\mathcal{J} \setminus \mathcal{I})$.  In the following, we want to extend this
result in the direction of computing finite bases of $\Th_{c}(\mathcal{J}, \mathcal{I})$
by computing suitable bases in the corresponding induced contexts.  This will be helpful
later when we develop our algorithm for model exploration by confidence.

\def\restricted{\mathord{\upharpoonright}}

For this theorem we need to introduce some extra notation first.  Let $X \subseteq
\Delta^{\mathcal{J}}$.  Then we shall denote with $\con K_{\mathcal{J}}\restricted_{X}$
the formal context whose set of objects is restricted to $X$, \ie
\begin{equation*}
  \con K_{\mathcal{J}}\restricted_{X} := (X, M_{\mathcal{J}}, \nabla),
\end{equation*}
where $(x, C) \in \nabla \iff x \in C^{\mathcal{J}}$ for $x \in X, C \in M_{\mathcal{J}}$
as before.

We can now formulate a result that allows to find bases of interpretations with untrusted
element from bases of corresponding induced contexts.

\begin{Theorem}
  \label{thm:bases-of-untrusted-interpretations-from-bases-of-contexts}
  Let $\mathcal{J}$ be a finite interpretation over $N_{C}$ and $N_{R}$, and let
  $\mathcal{I}$ be a subinterpretation of $\mathcal{J}$.  Let $c \in [0,1]$, and define
  \begin{equation*}
    \mathcal{T} := \Th_{c}(\con K_{\mathcal{J}}\restricted_{\Delta^{\mathcal{I}}}) \cap
    \Th(\con K_{\mathcal{J}}\restricted_{\Delta^{\mathcal{J}}\setminus\Delta^{\mathcal{I}}}).
  \end{equation*}
  Let $\mathcal{L} \subseteq \mathcal{T}$ be complete for $\mathcal{T}$.  Then $\bigsqcap
  \mathcal{L} \subseteq \Th_{c}(\mathcal{J}, \mathcal{I})$ and $\bigsqcap \mathcal{L}$ is
  complete for $\Th_{c}(\mathcal{J}, \mathcal{I})$.
\end{Theorem}

\begin{Proof}
  We first show $\bigsqcap \mathcal{L} \subseteq \Th_{c}(\mathcal{J}, \mathcal{I})$.  For
  this we need to show that for each $(\bigsqcap X \sqsubseteq \bigsqcap Y) \in \bigsqcap
  \mathcal{L}$ it is true that
  \begin{enumerate}[i. ]
  \item $\abs{ (\bigsqcap X \sqcap \bigsqcap Y)^{\mathcal{J}} \cap \Delta^{\mathcal{I}} }
    \ge c \cdot \abs{ (\bigsqcap X)^{\mathcal{J}} \cap \Delta^{\mathcal{I}} }$, and
  \item $(\bigsqcap X)^{\mathcal{J}} \setminus \Delta^{\mathcal{I}} \subseteq (\bigsqcap
    Y)^{\mathcal{J}} \setminus \Delta^{\mathcal{I}}$.
  \end{enumerate}
  For the first subclaim we observe that $\conf_{\con
    K_{\mathcal{J}}\restricted_{\Delta^{\mathcal{I}}}}(X \to Y) \ge c$, \ie
  \begin{equation*}
    \abs{ (X \cup Y)' \cap \Delta^{\mathcal{I}} } \ge c \cdot \abs{ X' \cap
      \Delta^{\mathcal{I}} }.
  \end{equation*}
  Since $X' = (\bigsqcap X)^{\mathcal{J}}$ and $Y' = (\bigsqcap Y)^{\mathcal{J}}$ by
  \Cref{prop:connection-I-prime-2}, we obtain
  \begin{equation*}
    \abs{ (\bigsqcap (X \cup Y))^{\mathcal{J}} \cap \Delta^{\mathcal{I}} } \ge c \cdot
    \abs{ (\bigsqcap X)^{\mathcal{J}} \cap \Delta^{\mathcal{I}} },
  \end{equation*}
  and since $\bigsqcap (X \cup Y) = \bigsqcap X \sqcap \bigsqcap Y$ we finally get
  \begin{equation*}
    \abs{ (\bigsqcap X \sqcap \bigsqcap Y))^{\mathcal{J}} \cap \Delta^{\mathcal{I}} } \ge
    c \cdot \abs{ (\bigsqcap X)^{\mathcal{J}} \cap \Delta^{\mathcal{I}} },
  \end{equation*}
  as required.

  For the second subclaim we observe that $X' \setminus \Delta^{\mathcal{I}} \subseteq Y'
  \setminus \Delta^{\mathcal{I}}$, because $X \to Y$ is valid in $\con
  K_{\mathcal{J}}\restricted_{\Delta^{\mathcal{J}} \setminus \Delta^{\mathcal{I}}}$.
  Since $X' = (\bigsqcap X)^{\mathcal{J}}$ and $Y' = (\bigsqcap Y)^{\mathcal{J}}$, we
  obtain
  \begin{equation*}
    (\bigsqcap X)^{\mathcal{J}} \setminus \Delta^{\mathcal{I}} \subseteq (\bigsqcap
    Y)^{\mathcal{J}} \setminus \Delta^{\mathcal{I}},
  \end{equation*}
  as required.

  We have thus shown that $\bigsqcap \mathcal{L} \subseteq \Th_{c}(\mathcal{J},
  \mathcal{I})$.  We shall now consider the completeness of $\bigsqcap \mathcal{L}$ for
  $\Th_{c}(\mathcal{J}, \mathcal{I})$.

  To this end, we shall show the following two subclaims
  \begin{enumerate}[i. ]
  \item $\bigsqcap \mathcal{L} \models (\bigsqcap U \sqsubseteq (\bigsqcap
    U)^{\mathcal{J}\mathcal{J}})$ for all $U \subseteq M_{\mathcal{J}}$, and
  \item $\bigsqcap \mathcal{L} \models \Conf(\mathcal{J}, c, \mathcal{I})$.
  \end{enumerate}
  Since
  \begin{equation*}
    \set{ \bigsqcap U \sqsubseteq (\bigsqcap U)^{\mathcal{J}\mathcal{J}} \mid U \subseteq
      M_{\mathcal{J}} }
  \end{equation*}
  is a base of $\mathcal{J}$, the completeness of $\bigsqcap \mathcal{L}$ for
  $\Th_{c}(\mathcal{J}, \mathcal{I})$ then follows immediately from
  \Cref{thm:finite-base-of-interpretation-with-untrusted-individuals}.

  For the first subclaim let $U \subseteq M_{\mathcal{J}}$.  Because
  \begin{equation*}
    \Th(\con K_{\mathcal{J}}) \subseteq \Th_{c}(\con K_{\mathcal{J}}
    \restricted_{\Delta^{\mathcal{I}}}) \cap \Th(\con K_{\mathcal{J}}
    \restricted_{\Delta^{\mathcal{J}} \setminus \Delta^{\mathcal{I}}})
  \end{equation*}
  it follows that $\mathcal{L}$ is complete for $\con K_{\mathcal{J}}$.  Therefore,
  \begin{equation*}
    \mathcal{L} \models (U \to U'').
  \end{equation*}
  By \Cref{lem:implicational-entailment-implies-gci-entailment} we obtain
  \begin{equation*}
    \bigsqcap \mathcal{L} \models (\bigsqcap U \sqsubseteq \bigsqcap U''),
  \end{equation*}
  and since $\bigsqcap U'' \equiv (\bigsqcap U)^{\mathcal{J}\mathcal{J}}$,  we obtain
  \begin{equation*}
    \bigsqcap \mathcal{L} \models (\bigsqcap U \sqsubseteq (\bigsqcap U)^{\mathcal{J}\mathcal{J}})
  \end{equation*}
  as required.

  For the second subclaim let $(X^{\mathcal{J}} \sqsubseteq Y^{\mathcal{J}}) \in
  \Conf(\mathcal{J}, c, \mathcal{I})$.  Then by \Cref{prop:connection-I-prime-2} it is
  true that
  \begin{equation*}
    X^{\mathcal{J}} \equiv \bigsqcap X', Y^{\mathcal{J}} \equiv \bigsqcap Y'.
  \end{equation*}
  Therefore,
  \begin{equation}
    \label{eq:51}
    \bigsqcap L \models (X^{\mathcal{J}} \sqsubseteq Y^{\mathcal{J}}) \iff \bigsqcap
    \mathcal{L} \models (\bigsqcap X' \sqsubseteq \bigsqcap Y').
  \end{equation}
  Therefore, to show $\bigsqcap \mathcal{L} \models (X^{\mathcal{J}} \sqsubseteq
  Y^{\mathcal{J}})$ it suffices to show $\mathcal{L} \models (X' \to Y')$.

  Recall that since $(X^{\mathcal{J}} \sqsubseteq Y^{\mathcal{J}}) \in \Conf(\mathcal{J},
  c, \mathcal{I})$, it is true that
  \begin{equation*}
    \abs{ (X^{\mathcal{J}} \sqcap Y^{\mathcal{J}})^{\mathcal{J}} \cap \Delta^{\mathcal{I}}
      } \ge c \cdot \abs{ X^{\mathcal{J}\mathcal{J}} \cap \Delta^{\mathcal{I}} }.
  \end{equation*}
  This implies
  \begin{equation*}
    \abs{ (\bigsqcap (X' \cup Y'))^{\mathcal{J}} \cap \Delta^{\mathcal{I}}
      } \ge c \cdot \abs{ (\bigsqcap X')^{\mathcal{J}} \cap \Delta^{\mathcal{I}} }.
  \end{equation*}
  and thus
  \begin{equation*}
    \abs{ ((X' \sqcap Y')' \cap \Delta^{\mathcal{I}} } \ge c \cdot \abs{ (\bigsqcap X')'
      \cap \Delta^{\mathcal{I}}},
  \end{equation*}
  \ie $(X' \to Y') \in \Th_{c}(\con K_{\mathcal{J}} \restricted_{\Delta^{\mathcal{I}}})$.

  Furthermore, it is true that
  \begin{equation*}
    X^{\mathcal{J}\mathcal{J}} \setminus \Delta^{\mathcal{I}} \subseteq Y^{\mathcal{J}\mathcal{J}} \setminus
    \Delta^{\mathcal{I}},
  \end{equation*}
  and by \Cref{prop:connection-I-prime-2} we have $X^{\mathcal{J}\mathcal{J}} = X'',
  Y^{\mathcal{J}\mathcal{J}} = Y''$, thus
  \begin{equation*}
    X'' \setminus \Delta^{\mathcal{I}} \subseteq Y'' \setminus \Delta^{\mathcal{I}},
  \end{equation*}
  \ie $(X' \to Y') \in \Th(\con K_{\mathcal{J}} \restricted_{\Delta^{\mathcal{J}} \setminus
    \Delta^{\mathcal{I}}})$.

  Since $\mathcal{L}$ is complete for $\mathcal{T}$, we thus obtain that
  \begin{equation*}
    \mathcal{L} \models (X' \to Y'),
  \end{equation*}
  and thus $\bigsqcap \mathcal{L} \models (\bigsqcap X' \sqsubseteq \bigsqcap Y')$ by
  \Cref{lem:implicational-entailment-implies-gci-entailment}, and $\bigsqcap \mathcal{L}
  \models (X^{\mathcal{J}} \sqsubseteq Y^{\mathcal{J}})$ by \Cref{eq:51}.
\end{Proof}

\subsection{Computing Bases of Formal Contexts with Growing Sets of Attributes}
\label{sec:grow-sets-attr-1}

We have seen how we can obtain finite bases of interpretation that contain untrusted
elements, for which we apply our confidence heuristics.  In the following two sections we
want to devise an algorithm that allows us to compute this base in a manner which is
suitable for being adapted towards model exploration by confidence.  In particular, we
shall see in this section how we can compute bases of
\begin{equation}
  \label{eq:52}
  \Th_{c}(\con K_{\mathcal{J}} \restricted_{\Delta^{\mathcal{I}}}) \cap \Th(\con
  K_{\mathcal{J}} \restricted_{\Delta^{\mathcal{J}} \setminus \Delta^{\mathcal{I}}}),
\end{equation}
where we compute the set $M_{\mathcal{J}}$ incrementally during the run of the algorithm.
Then, in the next section we shall see how we can use this algorithm and
\Cref{thm:bases-of-untrusted-interpretations-from-bases-of-contexts} to compute finite
bases of interpretation which contain untrusted elements.

Let us consider the problem of finding an algorithm that allows us to compute bases of the
set given in \Cref{eq:52} from a more abstract point of view.  More precisely, let us
consider two formal contexts $\con K_{1}$ and $\con K_{2}$ with the same attribute set
$M$.  Then we want to find an algorithm that computes a base of
\begin{equation}
  \label{eq:53}
  \Th_{c}(\con K_{1}) \cap \Th(\con K_{2}),
\end{equation}
and which allows to us incrementally supply the elements of $M$ as the computation
proceeds.

As a special case of \Cref{eq:53} we first consider the case that $\con K_{2} =
(\emptyset, M, \emptyset)$, \ie we want to devise the algorithm such that it computes a
base of $\Th_{c}(\con K_{1})$.  As in \Cref{sec:grow-sets-attr}, we want to obtain such an
algorithm by adapting the classical algorithm for computing the canonical base of a formal
context.  Indeed, we could simply obtain such an algorithm if we would replace in
\Cref{alg:base/growing-set-of-attributes} every occurrence of $(\cdot)_{\con K_{i+1}}''$
by a call to the closure operator induced by $\Th_{c}(\con K_{i+1})$.  However, as we had
already argued in \Cref{sec:expl-conf-1}, computing closures under $\Th_{c}(\con K_{i+1})$
may be infeasible.

To avoid this, we shall make use of the ideas we have developed in
\Cref{sec:poss-fast-expl}, when we devised an algorithm for exploration by confidence that
avoids computing closures under $\Th_{c}(\con K)$.  Indeed, we can just take
\Cref{alg:exploration-by-confidence-without-Th_c(K)-closures}, and instantiate it with an
expert that confirms all implications.

Recall that in this algorithm there were two cases of implications asked to the expert:
implications where either of the form $P_{i+1} \to \set{m}$, where $\conf_{\con K}(P_{i+1}
\to \set{m}) \ge c$ and $c \notin \mathcal{K}_{i}(P_{i+1})$, or $P_{i+1} \to
(P_{i+1})_{\con K \div \con L_{i}}''$.  We can simplify these two cases into one case by
defining for $P \subseteq M$ and $c \in [0,1]$
\begin{equation*}
  P^{\con K, c} := \set{ m \in M \mid \conf_{\con K}(P \to \set{m}) \ge c }.
\end{equation*}
Then we only ask implications of the form
\begin{equation*}
  P_{i+1} \to (P_{i+1})^{\con K, c},
\end{equation*}
and this then covers both cases.

To make this algorithm into an algorithm that allows the set of attributes to grow during
the computation, we use the ideas of \Cref{sec:grow-sets-attr}: whenever there are new
elements to be added to the current set of attributes, we add them as the smallest
elements.  In this way, the underlying Next Closure algorithm behaves as if those elements
would have been present right from the start of the run, and thus behaves as desired.

\addfunctionname{confident-base}

\begin{figure}[tp]
  \begin{Algorithm} Axiomatize Confident Implications with Growing Sets of Attributes
    \hspace*{0cm}
    \label{alg:confident-base/growing-attributes}
    \begin{lstlisting}
define confident-base($\con K = (G, M, I)$, $c \in [0,1]$)
  $i$ := $0$
  $M_i$ := $M$
  $I_i$ := $I$
  $\mathcal{S}_i$ := $P_i$ := $\mathcal{K}_i$ := $\emptyset$
  choose ${\leq_{M_{i}}} \text{ as a linear order on } M_{i}$
  
  forever do
    read $\con K_{i+1} = (G, M_{i+1}, I_{i+1}) \text{ such that } M_i \subseteq M_{i+1}
\text{ and } I_i = (G \times M_i) \cap I_{i+1}$
    read $\mathcal{S}_{i+1} \text{ such that } \mathcal{S}_i \subseteq \mathcal{S}_{i+1}
\subseteq \Th_{c}(\con K_{i+1})$
    choose ${\leq_{M_{i+1}}} \text{ such that (\ref{eq:48}) and (\ref{eq:49}) hold}$

    $\mathcal{K}_{i+1}$ := $\set{ P_k \to P_k^{\con K_{i+1}, c} \mid k \in \set{0, \ldots,
        i}, P_k \neq P_k^{\con K_{i+1}, c} }$

    $P_{i+1}^1$ := next-closure($M_{i+1}$, $\leq_{M_{i+1}}$, $P_i$, $\con K_{i+1}$)
    $P_{i+1}^2$ := next-closure($M_{i+1}$, $\leq_{M_{i+1}}$, $P_i$, $\mathcal{S}_{i+1} \cup \mathcal{K}_{i+1}$)

    $P_{i+1}$ := $\min\nolimits_{\preceq}(P_{i+1}^1, P_{i+1}^2)$.
    if $P_{i+1} =$ nil exit

    $i$ := $i + 1$
  end

  return $\mathcal{K}_{i+1}$  
end
    \end{lstlisting}
  \end{Algorithm}
\end{figure}

The algorithm that results from these considerations is shown in
\Cref{alg:confident-base/growing-attributes}.  Note that as in the case of
\Cref{alg:base/growing-set-of-attributes}, we cannot discard sets $P_{i}$ which are closed
under $(\cdot)^{\con K_{i+1}, c}$, \ie which satisfy $P_{i} = P_{i}^{\con K_{i+1}, c}$, as
$P_{i}$ may not be closed under $(\cdot)^{\con K_{j}, c}$ for some later iteration $j$.

We can argue termination of \Cref{alg:confident-base/growing-attributes} as we did before
for \Cref{alg:base/growing-set-of-attributes}: if at a certain iteration $\ell$ it is true
that for all iterations $k \geq \ell$ that $M_{k} = M_{\ell}$, then
\Cref{alg:confident-base/growing-attributes} must terminate.  This is in particular the
case if we want to compute a base of $\Th_{c}(\con K)$ where $\con K$ is a finite formal
context, and where the attributes of $\con K$ are supplied incrementally during the run of
the algorithm.

To show that upon termination the set $\mathcal{K}_{n}$ of implications, where $n$ is the
number of iterations of \Cref{alg:confident-base/growing-attributes}, is indeed a base of
$\Th_{c}(\con K_{n})$, we adapt the argumentation of \Cref{sec:grow-sets-attr} and
\Cref{sec:comp-bases-given} accordingly.

The following result, and its proof, are a generalization of~\cite[Lemma~6.3]{Diss-Felix}.

\begin{Proposition}
  \label{prop:property-of-confident-bases-with-growing-sets-of-attributes}
  Consider a terminating run of \Cref{alg:confident-base/growing-attributes}, and let $n$
  be the number of iterations of this run.  Let $Q \subseteq M_{n}$.   Then the following
  statements hold.
  \begin{enumerate}[i. ]
  \item If $Q \neq Q_{\con K_{n}}''$, then $Q$ is not $(\mathcal{S}_{n} \cup
    \mathcal{K}_{n})$-closed.
  \item If $Q = Q_{\con K_{n}}''$, then $Q = P_{k}$ for some $k \in \set{ 0, \dots, n }$.
  \end{enumerate}
\end{Proposition}
\begin{Proof}
  The case $Q = \emptyset = P_{0}$ can be handled quite easily: if $Q \neq Q_{\con
    K_{n}}''$, then $Q \neq Q^{\con K_{n}, c}$, since $Q_{\con K_{n}}'' \subseteq Q^{\con
    K_{n}, c}$.  Therefore, $(Q \to Q^{\con K_{n}, c}) \in \mathcal{K}_{n}$ and thus $Q$
  is not $(\mathcal{S}_{n} \cup \mathcal{K}_{n})$-closed.  If on the other hand $Q =
  Q_{\con K_{n}}''$, then $Q = P_{0}$ shows the claim.

  Now suppose that $Q \neq \emptyset$.  Then there exists $k \in \set{0, \dots, n}$ such that
  \begin{equation*}
    P_{k-1} \precneq Q \preceq P_{k}.
  \end{equation*}
  We first argue that $Q \subseteq M_{k}$.  To this end, suppose by contradiction that
  this is not the case, and let $m \in Q \setminus M_{k}$.  Since $m \notin M_{k}$, it is
  smaller than every element of $M_{k}$, by construction of the linear order
  $\leq_{M_{n}}$ on $M_{n}$.  Thus, $M_{k} \precneq \set{m}$, and since $\set{m} \subseteq
  Q$, we obtain
  \begin{equation*}
    M_{k} \precneq \set{m} \preceq Q,
  \end{equation*}
  contradicting the fact that $Q \preceq P_{k} \preceq M_{k}$.  Therefore, $Q \subseteq
  M_{k}$.

  Let us first consider the case $Q \neq Q_{\con K_{n}}''$, and assume by contradiction
  that $Q$ is $(\mathcal{S}_{n} \cup \mathcal{K}_{n})$-closed.  Then by construction
  \begin{equation*}
    Q \preceq P_{k}^{2} \preceq P_{k},
  \end{equation*}
  and thus $Q = P_{k}$.  Then $Q \neq Q_{\con K_{n}}''$ means $P_{k} \neq (P_{k})_{\con
    K_{n}}''$, thus $P_{k} \neq P_{k}^{\con K_{n}, c}$, and therefore
  \begin{equation*}
    (P_{k} \to P_{k}^{\con K_{n}, c}) \in \mathcal{K}_{n}.
  \end{equation*}
  But this means that $Q$ is not $(\mathcal{S}_{n} \cup \mathcal{K}_{n})$-closed, a
  contradiction.  Therefore, $Q$ is not $(\mathcal{S}_{n} \cup \mathcal{K}_{n})$-closed,
  as it was claimed.

  Let us now consider the case that $Q = Q_{\con K_{n}}''$, and we have to show that $Q =
  P_{\ell}$ for $\ell \in \set{ 0, \dots, n }$.  Since $Q = Q_{\con K_{n}}''$, it is also
  true that $Q = Q_{\con K_{k}}''$, since
  \begin{equation*}
    I_{k} = (G \times M_{k}) \cap I_{n}.
  \end{equation*}
  But then $Q \preceq P_{k}^{1} \preceq P_{k}$, and thus $Q = P_{k}$, as required.
\end{Proof}

\begin{Theorem}
  \label{thm:confident-bases-with-growing-sets-of-attributes}
  Let $\con K$ be a finite formal context, $c \in [0,1]$, and suppose that
  \Cref{alg:confident-base/growing-attributes} applied to $\con K$ and $c$ terminates
  after $n$ iterations.  Then $\mathcal{K}_{n} \subseteq \Th_{c}(\con K_{n})$ and
  $\mathcal{K}_{n}$ is a base for $\Th_{c}(\con K_{n})$ with background knowledge
  $\mathcal{S}_{n}$.
\end{Theorem}
\begin{Proof}
  The fact that $\mathcal{K}_{n} \cup \mathcal{S}_{n} \subseteq \Cn(\Th_{c}(\con K_{n}))$
  is clear from the definition of $\mathcal{K}_{n}$ and $\mathcal{S}_{n}$, and we thus
  only need to show that $\mathcal{S}_{n} \cup \mathcal{K}_{n}$ is complete for
  $\Th_{c}(\con K_{n})$.  For this we shall use
  \Cref{lem:characterization-of-completeness} and show that every set $Q \subseteq M_{n}$
  which is $(\mathcal{S}_{n} \cup \mathcal{K}_{n})$-closed is also $\Th_{c}(\con
  K_{n})$-closed.

  To this end, let us assume by contradiction that $Q$ is $(\mathcal{S}_{n} \cup
  \mathcal{K}_{n})$-closed, but not $\Th_{c}(\con K_{n})$-closed.  Then there exists an
  implication $(P \to \set{m}) \in \Th_{c}(\con K_{n})$ such that $P \subseteq Q$ and $m
  \notin Q$.  Furthermore, since $Q$ is $(\mathcal{S}_{n} \cup \mathcal{K}_{n})$-closed,
  it follows from \Cref{prop:property-of-confident-bases-with-growing-sets-of-attributes}
  that $Q = Q_{\con K_{n}}''$.  Then since
  \begin{equation*}
    \conf_{\con K_{n}}(P \to \set{m}) = \conf_{\con K_{n}}(P_{\con K_{n}}'' \to \set{m}),
  \end{equation*}
  we can assume that $P = P_{\con K_{n}}''$.  But then, using
  \Cref{prop:property-of-confident-bases-with-growing-sets-of-attributes} again, we obtain
  that $P = P_{k}$ for some $k \in \set{ 0, \dots, n }$, and thus
  \begin{equation*}
    (P \to P^{\con K_{n}, c}) = (P_{k} \to P_{k}^{\con K_{n}, c}) \in \mathcal{K}_{n},
  \end{equation*}
  because $P_{k} \neq P_{k}^{\con K_{n}, c}$, since $m \notin P_{k} \subseteq Q$, but $m
  \in P_{k}^{\con K_{n}, c}$.  Now since $Q$ is $\mathcal{K}_{n}$-closed, $P_{k} \subseteq
  Q$ implies $P_{k}^{\con K_{n}, c} \subseteq Q$, and since $m \in P_{k}^{\con K_{n}, c}$,
  we obtain $m \in Q$, a contradiction.

  Therefore, every set $Q$ that is $(\mathcal{S}_{n} \cup \mathcal{K}_{n})$-closed is also
  $\Th_{c}(\con K_{n})$-closed, and thus $\mathcal{S}_{n} \cup \mathcal{K}_{n}$ is
  complete for $\Th_{c}(\con K_{n})$, as it was claimed.
\end{Proof}

In this theorem, the set $\mathcal{K}_{n}$ does not necessarily contain implications whose
confidence is at least $c$, \ie $\mathcal{K}_{n} \subseteq \Th_{c}(\con K)$ is not
necessarily true.  However, a simple modification of
\Cref{alg:confident-base/growing-attributes} achieves that the computed base is indeed a
confident base of $\Th_{c}(\con K)$.  For this we define the return value of
\Cref{alg:confident-base/growing-attributes} as
\begin{equation*}
  \hat{\mathcal{K}} := \set{ P \to \set{m} \mid (P \to P^{\con K, c}) \in \mathcal{K}_{n},
    m \in P^{\con K, c} }.
\end{equation*}
Then, by definition of $P^{\con K, c}$, it is true that $\hat{\mathcal{K}} \subseteq
\Th_{c}(\con K)$.  Of course, instead of choosing $m$ in $P^{\con K, c}$, it also suffices
to consider only $m \in P^{\con K, c} \setminus \mathcal{S}_{n}(P)$.

With \Cref{alg:confident-base/growing-attributes} we have now obtained an algorithm that
allows us to compute bases of $\Th_{c}(\con K)$, where the attribute set can be added
incrementally during the computation.  Based on this algorithm, we now want to turn back
to our initial problem of finding bases of $\Th_{c}(\con K_{1}) \cap \Th(\con K_{2})$,
where both formal contexts $\con K_{1}$ and $\con K_{2}$ have the same attribute set $M$.

The main idea to adapt \Cref{alg:confident-base/growing-attributes} to this setting is to
``divide'' the working context into two formal contexts $\con K_{i+1}$ and $\con L_{i+1}$,
such that in $\con K_{i+1}$ (the ``untrusted'' part) we apply the usual confidence
heuristics, and in $\con L_{i+1}$ (the ``trusted'' part) we consider only valid
implications.  Then instead of computing the sets $P^{\con K_{i+1}, c}$, we consider
\begin{equation*}
  P^{\con K_{i+1}, c} \cap P_{\con L_{i+1}}'' = \set{ m \in P_{\con L_{i+1}}'' \mid
    \conf_{\con K_{i+1}}(P \to \set{m}) \ge c }.
\end{equation*}
The division of the working context into $\con K_{i+1}$ and $\con L_{i+1}$ can be
represented by using the subposition $\con K_{i+1} \div \con L_{i+1}$ as the working
context.  Then clearly
\begin{equation}
  \label{eq:54}
  P_{\con K_{i+1} \div \con L_{i+1}}'' = P_{\con K_{i+1}}'' \cap P_{\con L_{i+1}}''
  \subseteq P^{\con K_{i+1}, c} \cap P_{\con L_{i+}}''.
\end{equation}
The resulting algorithm is shown in
\Cref{alg:confident-base/growing-attributes-trusted-objects}.

\addfunctionname{confident-base/trusted-objects}

\begin{figure}[tp]
  \begin{Algorithm} Axiomatize Confident Implications with Trusted Objects
    \hspace*{0cm}
    \label{alg:confident-base/growing-attributes-trusted-objects}
    \begin{lstlisting}
define confident-base/trusted-objects($\con K = (G_{1}, M, I)$, $\con L = (G_{2}, M, J)$, $c \in [0,1]$)
  $i$ := $0$
  $M_i$ := $M$
  $I_i$ := $I$
  $J_{i}$ := $J$
  $\mathcal{S}_i$ := $P_i$ := $\mathcal{K}_i$ := $\emptyset$
  choose ${\leq_{M_{i}}} \text{ as a linear order on } M_{i}$
  
  forever do
    read $M_{i+1} \text{ such that } M_{i} \subseteq M_{i+1}$
    read $I_{i+1} \text{ such that } I_{i} = (G_{1} \times M_{i}) \cap I_{i+1}$
    read $J_{i+1} \text{ such that } J_{i} = (G_{2} \times M_{i}) \cap J_{i+1}$
    $\con K_{i+1}$ := $(G_{1}, M_{i+1}, I_{i+1})$
    $\con L_{i+1}$ := $(G_{2}, M_{i+1}, J_{i+1})$
    read $\mathcal{S}_{i+1} \text{ such that } \mathcal{S}_i \subseteq \mathcal{S}_{i+1}
\subseteq \Th_{c}(\con K_{i+1}) \cap \Th(\con L_{i+1})$
    choose ${\leq_{M_{i+1}}} \text{ such that (\ref{eq:48}) and (\ref{eq:49}) hold}$

    $\mathcal{K}_{i+1}$ := $\set{ P_k \to P_k^{\con K_{i+1}, c} \cap P_{\con L_{i+1}}'' \mid k \in \set{0, \ldots, i}, P_k \neq P_k^{\con K_{i+1}, c} \cap P_{\con L_{i+1}}''}$

    $P_{i+1}^1$ := next-closure($M_{i+1}$, $\leq_{M_{i+1}}$, $P_i$, $\con K_{i+1} \div \con L_{i+1}$)
    $P_{i+1}^2$ := next-closure($M_{i+1}$, $\leq_{M_{i+1}}$, $P_i$, $\mathcal{S}_{i+1} \cup \mathcal{K}_{i+1}$)

    $P_{i+1}$ := $\min\nolimits_{\preceq}(P_{i+1}^1, P_{i+1}^2)$.
    if $P_{i+1} =$ nil exit

    $i$ := $i + 1$
  end

  return $\mathcal{K}_{i+1}$  
end
    \end{lstlisting}
  \end{Algorithm}
\end{figure}

Because of \Cref{eq:54} the proofs of
\Cref{prop:property-of-confident-bases-with-growing-sets-of-attributes} and
\Cref{thm:confident-bases-with-growing-sets-of-attributes} can be carried over to
\Cref{alg:confident-base/growing-attributes-trusted-objects} almost literally, essentially
by replacing every occurrence of $(\cdot)_{\con K_{i}}''$ by $(\cdot)_{\con K_{i} \div
  \con L_{i}}''$, and by replacing every expression of the form $P^{\con K_{i}, c}$ by
$P^{\con K_{i}, c} \cap P_{\con L_{i}}''$.  From this we obtain the validity of the
following results.

\begin{Proposition}
  Consider a terminating run of
  \Cref{alg:confident-base/growing-attributes-trusted-objects}, and let $n$ be the number
  of iterations of this run.  Let $Q \subseteq M_{n}$.  Then the following statements
  hold.
  \begin{enumerate}[i. ]
  \item If $Q \neq Q_{\con K_{n} \div \con L_{n}}''$, then $Q$ is not $(\mathcal{S}_{n} \cup
    \mathcal{K}_{n})$-closed.
  \item If $Q = Q_{\con K_{n} \div \con L_{n}}''$, then $Q = P_{k}$ for some $k \in \set{
      0, \dots, n }$.
  \end{enumerate}
\end{Proposition}
\begin{Theorem}
  \label{thm:confident-bases-with-growing-sets-of-attributes-and-trusted-objects}
  Let $\con K, \con L$ be two finite formal contexts with attribute set $M$ and disjoint
  sets of objects, $c \in [0,1]$, and suppose that
  \Cref{alg:confident-base/growing-attributes-trusted-objects} applied to $\con K$, $\con
  L$ and $c$ terminates after $n$ iterations.  Then $\mathcal{K}_{n} \subseteq
  \Th_{c}(\con K_{n}) \cap \Th(\con L_{n})$ and $\mathcal{K}_{n}$ is a base for
  $\Th_{c}(\con K_{n}) \cap \Th(\con L_{n})$ with background knowledge $\mathcal{S}_{n}$.
\end{Theorem}

\subsection{Computing Bases of Finite Interpretations with Untrusted Elements}
\label{sec:expl-conf-gcis}

We shall now use the results of the previous section to devise an algorithm that allows us
to compute bases of interpretations with untrusted elements.  More precisely, let
$\mathcal{J}$ be a finite interpretation over $N_{C}$ and $N_{R}$, and let $\mathcal{I}$
be a finite, connected subinterpretation of $\mathcal{J}$, such that $\mathcal{J}
\setminus \mathcal{I}$ is also a connected subinterpretation of $\mathcal{J}$.  We then
want to adapt \Cref{alg:confident-base/growing-attributes-trusted-objects} to compute a
finite base of $\Th_{c}(\mathcal{J}, \mathcal{I})$.

To this end, we consider as input for
\Cref{alg:confident-base/growing-attributes-trusted-objects} the induced formal context
$\con K_{\mathcal{J}}$, represented as the subposition of $\con K_{\mathcal{J}}
\restricted_{\Delta^{\mathcal{I}}}$ and $\con K_{\mathcal{J}}
\restricted_{\Delta^{\mathcal{J}} \setminus \Delta^{\mathcal{I}}}$, where the common
attribute set $M_{\mathcal{I}}$ is computed incrementally as in
\Cref{alg:base-of-interpretation}.  The result is shown in
\Cref{alg:confident-base-gcis/trusted-objects}.  Note that in this algorithm we again
utilize the idea of using
\begin{equation*}
  \mathcal{S}_{i+1} = \set{ \set{ C } \to \set{ D } \mid C, D \in M_{i+1}, C \sqsubseteq D }
\end{equation*}
as background knowledge, because the resulting set $\bigsqcap \mathcal{S}_{i+1}$ of GCIs
is trivial, but the set $\mathcal{S}_{i+1}$ of implications is not.

\addfunctionname{confident-base-gcis/trusted-elements}

\begin{figure}[tp]
  \begin{Algorithm} Axiomatize Confident GCIs in the Presence of Trusted Elements
    \hspace*{0cm}
    \label{alg:confident-base-gcis/trusted-objects}
    \begin{lstlisting}
define confident-base-gcis/trusted-elements($\mathcal{J}$, $\mathcal{I}$, $c \in [0,1]$)
  $i$ := $0$
  $M_i$ := $N_{C} \cup \set{ \bot }$
  $\mathcal{S}_i$ := $\set{ \set{ \bot } \to \set{ A } \mid A \in N_{C}}$
  $P_i$ := $\mathcal{K}_i$ := $\emptyset$
  choose ${\leq_{M_{i}}} \text{ as a linear order on } M_{i}$
  
  forever do
    $M_{i+1}$ := $M_{i} \cup \set{ \exists r. (\bigsqcap P_{i})^{\mathcal{J}\mathcal{J}} \mid r \in N_{R} }$ ;; union up to equivalence $\label{lst:confident-base-gcis-line-1}$
    $\con K_{i+1}$ := induced-context($\mathcal{I}$, $M_{i+1}$)
    $\con L_{i+1}$ := induced-context($\mathcal{J} \setminus \mathcal{I}$, $M_{i+1}$) $\label{lst:confident-base-gcis-line-2}$
    $\mathcal{S}_{i+1}$ := $\set{ \set{ C } \to \set{ D } \mid C, D \in M_{i+1}, C
\sqsubseteq D }$.
    choose ${\leq_{M_{i+1}}} \text{ such that (\ref{eq:48}) and (\ref{eq:49}) hold}$

    $\mathcal{K}_{i+1}$ := $\set{ P_k \to P_k^{\con K_{i+1}, c} \cap P_{\con L_{i+1}}'' \mid k \in \set{0, \ldots, i}, P_k \neq P_k^{\con K_{i+1}, c} \cap P_{\con L_{i+1}}''}$

    $P_{i+1}^1$ := next-closure($M_{i+1}$, $\leq_{M_{i+1}}$, $P_i$, $\con K_{i+1} \div \con L_{i+1}$)
    $P_{i+1}^2$ := next-closure($M_{i+1}$, $\leq_{M_{i+1}}$, $P_i$, $\mathcal{S}_{i+1} \cup \mathcal{K}_{i+1}$)

    $P_{i+1}$ := $\min\nolimits_{\preceq}(P_{i+1}^1, P_{i+1}^2)$.
    if $P_{i+1} =$ nil exit

    $i$ := $i + 1$
  end

  return $\mathcal{K}_{i+1}$  
end
    \end{lstlisting}
  \end{Algorithm}
\end{figure}

To see that \Cref{alg:confident-base-gcis/trusted-objects} indeed computes a base of
$\Th_{c}(\mathcal{J}, \mathcal{I})$, we shall first argue that it is of the form of
\Cref{alg:confident-base/growing-attributes-trusted-objects}.  Thereafter, we shall show
that when \Cref{alg:confident-base-gcis/trusted-objects} is applied to $\mathcal{J}$,
$\mathcal{I}$, and $c \in [0,1]$, and terminates after $n$ iterations, that then $M_{n} =
M_{\mathcal{J}}$ up to equivalence.  Thus, by
\Cref{thm:confident-bases-with-growing-sets-of-attributes-and-trusted-objects}, the
algorithm computes a base $\mathcal{K}_{n}$ of $\Th_{c}(\con K_{\mathcal{J}}
\restricted_{\Delta^{\mathcal{I}}}) \cap \Th(\con K_{\mathcal{J}}
\restricted_{\Delta^{\mathcal{J}} \setminus \Delta^{\mathcal{I}}})$, and then
\Cref{thm:bases-of-untrusted-interpretations-from-bases-of-contexts} yields that
$\bigsqcap \mathcal{K}_{n}$ is a base of $\Th_{c}(\mathcal{J}, \mathcal{I})$.

To argue that \Cref{alg:confident-base-gcis/trusted-objects} is of the form of
\Cref{alg:confident-base/growing-attributes-trusted-objects} we need to show that the
variables $M_{i+1}, \con K_{i+1}, \con L_{i+1}, \mathcal{S}_{i+1}$ computed in
\Cref{alg:confident-base-gcis/trusted-objects} satisfy the constraints given in
\Cref{alg:confident-base/growing-attributes-trusted-objects}.  However, this it is quite
clear from the definition of these variables: it is clear that $M_{i} \subseteq M_{i+1}$,
and that the incidence relations of $\con K_{i+1}$ and $\con L_{i+1}$ restricted to
$M_{i}$ are the incidence relations of $\con K_{i}$ and $\con L_{i}$, respectively.
Furthermore, $\mathcal{S}_{i+1} \subseteq \Th_{c}(\con K_{i+1}) \cap \Th(\con L_{i+1})$,
because $\mathcal{S}_{i+1}$ is even valid in $\con K_{i+1} \div \con L_{i+1}$.

Note that since $\mathcal{J}$ is a finite interpretation, the set $M_{\mathcal{J}}$ is
finite.  Since $M_{i} \subseteq M_{\mathcal{J}}$ holds for all iterations $i$, up to
equivalence, it is true that from a certain iteration $\ell$ on, $M_{\ell} = M_{k}$ is
true for all $k \geq \ell$.  Thus, \Cref{alg:confident-base-gcis/trusted-objects}
terminates on input $\mathcal{J}$, $\mathcal{I}$, and $c$.

To show that upon termination of \Cref{alg:confident-base-gcis/trusted-objects} that
$M_{n} = M_{\mathcal{J}}$ is true up to equivalence, we start with the following result,
which is an adaption of \cite[Lemma~6.7]{Diss-Felix}.

\begin{Proposition}
  \label{prop:Felix-6.7-adapted}
  Consider a terminating run of \Cref{alg:confident-base-gcis/trusted-objects}, and let
  $n$ be the number of iterations.  Then for every $U \subseteq M_{n}$ and $r \in N_{R}$
  it is true that
  \begin{equation*}
    \exists r.(\bigsqcap U)^{\mathcal{J}\mathcal{J}} \in M_{n}
  \end{equation*}
  up to equivalence.
\end{Proposition}
\begin{Proof}
  Note that because both $\mathcal{I}$ and $\mathcal{J} \setminus \mathcal{I}$ are
  connected subinterpretations of $\mathcal{J}$, \Cref{lem:Felix-6.12} shows that $\con
  K_{n} \div \con L_{n}$ is indeed the induced context of $\mathcal{J}$ and $M_{n}$.
  Thus, we obtain from \Cref{prop:connection-I-prime-2} that
  \begin{equation*}
    (\bigsqcap U_{\con K_{n} \div \con L_{n}}'')^{\mathcal{J}} = U_{\con K_{n} \div \con
      L_{n}}''' = U_{\con K_{n} \div \con L_{n}}' = (\bigsqcap U)^{\mathcal{J}}.
  \end{equation*}
  Therefore, it is true that
  \begin{equation*}
    \exists r. (\bigsqcap U_{\con K_{n} \div \con L_{n}}'')^{\mathcal{J}\mathcal{J}}
    \equiv \exists r. (\bigsqcap U)^{\mathcal{J}\mathcal{J}}.
  \end{equation*}
  Since \Cref{alg:confident-base-gcis/trusted-objects} is a special case of
  \Cref{alg:confident-base/growing-attributes-trusted-objects},
  \Cref{prop:property-of-confident-bases-with-growing-sets-of-attributes} is also
  applicable to \Cref{alg:confident-base-gcis/trusted-objects}, and we thus obtain a $k
  \in \set{ 0, \dots, n }$ such that $U_{\con K_{n} \div \con L_{n}}'' = P_{k}$.  Since
  $\exists r. (\bigsqcap P_{k})^{\mathcal{J}\mathcal{J}} \in M_{k+1} \subseteq M_{n}$, we obtain
  \begin{equation*}
    \exists r. (\bigsqcap U)^{\mathcal{J}\mathcal{J}} \equiv \exists r. (\bigsqcap U_{\con
      K_{n} \div \con L_{n}}'')^{\mathcal{J}\mathcal{J}} \equiv \exists r. (\bigsqcap
    P_{k})^{\mathcal{J}\mathcal{J}} \in M_{n},
  \end{equation*}
  as desired.
\end{Proof}

Using this proposition, we shall now show the correctness of
\Cref{alg:confident-base-gcis/trusted-objects}.

\begin{Theorem}
  \label{thm:confident-base-gcis-trusted-objects-is-correct}
  Let $\mathcal{J}$ be a finite interpretation over $N_{C}$ and $N_{R}$, and let
  $\mathcal{I}$ be a connected subinterpretation of $\mathcal{J}$ such that $\mathcal{J}
  \setminus \mathcal{I}$ is also a connected subinterpretation of $\mathcal{J}$.  Let $c
  \in [0,1]$, and let $n$ be the number of iterations of
  \Cref{alg:confident-base-gcis/trusted-objects} when applied to $\mathcal{J}$,
  $\mathcal{I}$, and $c \in [0,1]$.  Then $\bigsqcap \mathcal{K}_{n}$ is a base of
  $\Th_{c}(\mathcal{J}, \mathcal{I})$.
\end{Theorem}

The proof of this theorem, which is an adaption of the proofs of~\cite[Lemma~6.8,
Theorem~6.9]{Diss-Felix}, uses induction over the role depth of concept descriptions.
Since $\bigsqcap \mathcal{K}_{n}$ may contain proper $\ELgfpbot$ concept descriptions, it
may not be immediately obvious how this can be done, and we need an extra result that
allows us to use this argumentation.

\begin{Lemma}[Lemma~5.6 from~\cite{Diss-Felix}]
  \label{lem:Felix-5.6}
  Let $\mathcal{I}$ be a finite interpretation over $N_{C}$ and $N_{R}$, and let $C$ be an
  $\ELgfpbot$ concept description over $N_{C}$ and $N_{R}$.  Then there exists an $\ELbot$
  concept description over $N_{C}$ and $N_{R}$ such that
  \begin{equation*}
    C^{\mathcal{I}} = D^{\mathcal{I}} \quad\text{and}\quad C \sqsubseteq D.
  \end{equation*}
\end{Lemma}

We now prove \Cref{thm:confident-base-gcis-trusted-objects-is-correct}.

\begin{Proof}[\Cref{thm:confident-base-gcis-trusted-objects-is-correct}]
  We first show that $M_{n} = M_{\mathcal{J}}$ is true up to equivalence, \ie every
  element of $M_{n}$ is equivalent to some element in $M_{\mathcal{J}}$ and vice versa.
  Since $M_{n} \subseteq M_{\mathcal{J}}$ up to equivalence by definition of $M_{n}$, it
  suffices to show that every element of $M_{\mathcal{J}}$ is equivalent to some element
  in $M_{n}$.

  To show that $M_{\mathcal{J}} \subseteq M_{n}$ holds up to equivalence, we shall show
  that for each $r \in N_{R}$ and $X \subseteq \Delta^{\mathcal{J}}$ there exists $C \in
  M_{n}$ such that $C \equiv \exists r. X^{\mathcal{J}}$.  To this end, we observe that by
  \Cref{lem:Felix-5.6} there exists an $\ELbot$ concept description $D$ over $N_{C}$ and
  $N_{R}$ such that
  \begin{equation*}
    D^{\mathcal{J}} = X^{\mathcal{J}\mathcal{J}}.
  \end{equation*}
  Since $D^{\mathcal{J}\mathcal{J}} = X^{\mathcal{J}\mathcal{J}\mathcal{J}} =
  X^{\mathcal{J}}$, it is sufficient for showing $M_{\mathcal{J}} \subseteq M_{n}$ to show
  that for each $\ELbot$ concept description $D$ and each $r \in N_{R}$ it is true that
  \begin{equation*}
    \exists r. D^{\mathcal{J}\mathcal{J}} \in M_{n}
  \end{equation*}
  up to equivalence.  We shall show this claim now by induction over the role-depth of
  $D$.

  The \textit{base case} is $D = \bot$, or $D$ being a conjunction of concept names from
  $N_{C}$.  The case $D = \bot$ is trivial, as $\exists r. \bot^{\mathcal{J}\mathcal{J}}
  \equiv \bot \in M_{n}$ for all $r \in N_{R}$.  If $D = \bigsqcap S$ for some $S
  \subseteq N_{C}$, then since $S \subseteq M_{n}$, \Cref{prop:Felix-6.7-adapted} implies
  for all $r \in N_{R}$ that
  \begin{equation*}
    \exists r. D^{\mathcal{J}\mathcal{J}} = \exists r. (\bigsqcap
    S)^{\mathcal{J}\mathcal{J}} \in M_{n}
  \end{equation*}
  up to equivalence.

  For the \textit{step case} let $D$ be an $\ELbot$ concept description with role-depth $d
  > 0$, and let $r \in N_{R}$.  Assume by induction hypothesis that for all $\ELbot$
  concept descriptions $E$ over $N_{C}$ and $N_{R}$ with role depth smaller than $d$ that
  \begin{equation}
    \label{eq:55}
    \exists s. E^{\mathcal{J}\mathcal{J}} \in M_{n}
  \end{equation}
  is true up to equivalence for all $s \in N_{R}$.

  Since $D$ is an $\ELbot$ concept description, there exist $U \subseteq N_{C}$, $r_{1},
  \dots, r_{k} \in N_{R}$ and $E_{1}, \dots, E_{k} \in \ELbot(N_{C}, N_{R})$ such that
  \begin{equation*}
    D \equiv \bigsqcap U \sqcap \bigsqcap_{i=1}^{k} \exists r_{i}. E_{i}.
  \end{equation*}
  Then by \Cref{prop:double-II-under-I}
  \begin{align*}
    D^{\mathcal{J}\mathcal{J}}
    &\equiv \bigl( \bigsqcap U \sqcap \bigsqcap_{i=1}^{k} \exists r_{i}. E_{i}
    \bigr)^{\mathcal{J}\mathcal{J}} \\
    &\equiv \bigl( \bigsqcap U \sqcap \bigsqcap_{i=1}^{k} \exists
    r_{i}. E_{i}^{\mathcal{J}\mathcal{J}} \bigr)^{\mathcal{J}\mathcal{J}}.
  \end{align*}
  By induction hypothesis \Cref{eq:55}, $\exists r_{i}. E_{i}^{\mathcal{J}\mathcal{J}} \in
  M_{n}$ up to equivalence, for all $i = 1, \dots, k$.  But then
  \begin{equation*}
    V := U \cup \set{ \exists r_{i}. E_{i}^{\mathcal{J}\mathcal{J}} \mid i = 1, \dots, k }
    \subseteq M_{n},
  \end{equation*}
  and \Cref{prop:Felix-6.7-adapted} implies that
  \begin{equation*}
    \exists r. D^{\mathcal{J}\mathcal{J}} \equiv \exists r. (\bigsqcap
    V)^{\mathcal{J}\mathcal{J}} \in M_{n}
  \end{equation*}
  up to equivalence.  This completes the induction step and shows that $M_{\mathcal{J}} =
  M_{n}$ holds up to equivalence.

  By \Cref{thm:confident-bases-with-growing-sets-of-attributes-and-trusted-objects} we
  know that $\mathcal{K}_{n}$ is a base of
  \begin{equation*}
    \Th_{c}(\con K_{n}) \cap \Th(\con L_{n})
  \end{equation*}
  with background knowledge $\mathcal{S}_{n} = \set{ \set{ C } \to \set{ D } \mid C, D \in
    M_{n}, C \sqsubseteq D }$.  Since $M_{n} = M_{\mathcal{J}}$ up to equivalence,
  $\mathcal{K}_{n}$ is, up to equivalence, also a base of
  \begin{equation*}
    \Th_{c}(\con K_{\mathcal{J}} \restricted_{\Delta^{\mathcal{I}}}) \cap \Th(\con
    K_{\mathcal{J}} \restricted_{\Delta^{\mathcal{J}} \setminus \Delta^{\mathcal{I}}})
  \end{equation*}
  with background knowledge $\mathcal{S}_{n}$.  By
  \Cref{thm:bases-of-untrusted-interpretations-from-bases-of-contexts}, $\bigsqcap
  (\mathcal{K}_{n} \cup \mathcal{S}_{n})$ is a base of $\Th_{c}(\mathcal{J},
  \mathcal{I})$, and since $\bigsqcap (\mathcal{K}_{n} \cup \mathcal{S}_{n})$ is
  element-wise equivalent to $\bigsqcap \mathcal{K}_{n}$, we obtain that $\bigsqcap
  \mathcal{K}_{n}$ is a base of $\Th_{c}(\mathcal{J}, \mathcal{I})$, as required.
\end{Proof}

\subsection{Exploring Confident GCIs with Expert Interaction}
\label{sec:expl-conf-gcis-1}

We are now prepared to devise an algorithm for model exploration by confidence, by
replacing in \Cref{alg:confident-base-gcis/trusted-objects} every explicit access to the
interpretation $\mathcal{J}$ by suitable expert interaction.  Recall that this we want to
do because we now consider $\mathcal{J}$ as the background interpretation of the
exploration process, which we cannot access directly.

To conduct this adaption of \Cref{alg:confident-base-gcis/trusted-objects}, we observe
that there are to lines in this algorithm where $\mathcal{J}$ is accessed directly, namely
\begin{enumerate}[i. ]
\item\label{item:21} in the computation of $M_{i+1}$
  (line~\ref{lst:confident-base-gcis-line-1}), more precisely in the computation of the
  concept description $\exists r.(\bigsqcap P_{i})^{\mathcal{J}\mathcal{J}}$, and
\item\label{item:22} in the computation of the formal context $\con L_{i+1}$
  (line~\ref{lst:confident-base-gcis-line-2}) as the induced context of $\mathcal{J}
  \setminus \mathcal{I}$ and $M_{i+1}$.
\end{enumerate}

For \cref{item:21} we can argue as in \Cref{sec:an-algor-expl}, using
\Cref{lem:Felix-6.14}: if $\mathcal{I}_{\ell}$ denotes the current working interpretation,
which is a connected subinterpretation of the background interpretation $\mathcal{J}$,
then if the expert confirms the GCI
\begin{equation*}
  \bigsqcap P_{i} \sqsubseteq (\bigsqcap P_{i})^{\mathcal{I}_{\ell}\mathcal{I}_{\ell}},
\end{equation*}
then $(\bigsqcap P_{i})^{\mathcal{I}_{\ell}\mathcal{I}_{\ell}} \equiv (\bigsqcap
P_{i})^{\mathcal{J}\mathcal{J}}$ is true.

Handling \cref{item:22} is a bit more problematic, though: the formal context $\con
L_{i+1}$ is used for the computation of the next candidate set $P_{i+1}^{1}$, and for this
we need to ensure that $\con L_{i+1}$ contains all ``relevant'' objects are present in
$\con L_{i+1}$, and this may include objects which are not counterexamples to GCIs
proposed to the expert.

Let us consider this in more detail: we observe that the computation of $P_{i+1}^{1}$ in
our hypothetical adaption of \Cref{alg:confident-base-gcis/trusted-objects} would
\emph{not} be correct if the lectically next intent after $P_{i}$ of the current working
context $\con K_{i+1} \div \con L_{i+1}$ in \Cref{alg:confident-base-gcis/trusted-objects}
is not an intent of the working context of our hypothetical adaption.  In other words, if
we denote the working context of our adapted algorithm by $\con K_{i+1} \div \bar{\con
  L}_{i+1}$, then this means that
\begin{align*}
  P_{i+1}^{1} &= (P_{i+1}^{1})_{\con K_{i+1} \div \con L_{i+1}}''\\
  P_{i+1}^{1} &\neq (P_{i+1}^{1})_{\con K_{i+1} \div \bar{\con L}_{i+1}}''.
\end{align*}
If $\bar{\mathcal{S}}_{i+1}$ denotes the currently known implications and
$\bar{\mathcal{K}}_{i+1}$ the currently confirmed implications in our hypothetical
adaption, then we know that $\bar{\mathcal{S}}_{i+1} \cup \bar{\mathcal{K}}_{i+1}$ is
valid in $\bar{\con L}_{i+1}$, and from this we obtain
\begin{align*}
  (\mathcal{S}_{i+1} \cup \mathcal{K}_{i+1})(P_{i+1}^{1})
  &= (\mathcal{S}_{i+1} \cup \mathcal{K}_{i+1})(P_{i+1}^{1})_{\con L_{i+1}}''\\
  (\mathcal{S}_{i+1} \cup \mathcal{K}_{i+1})(P_{i+1}^{1})
  &\neq (\mathcal{S}_{i+1} \cup \mathcal{K}_{i+1})(P_{i+1}^{1})_{\bar{\con L}_{i+1}}''
\end{align*}
Therefore, the implication
\begin{equation*}
  (\mathcal{S}_{i+1} \cup \mathcal{K}_{i+1})(P_{i+1}^{1}) \to
  (\mathcal{S}_{i+1} \cup \mathcal{K}_{i+1})(P_{i+1}^{1})_{\bar{\con L}_{i+1}}''
\end{equation*}
must be rejected by the expert, because it does not hold in $\con L_{i+1}$, and upon
rejection all necessary counterexamples are added to $\bar{\con L}_{i+1}$.  Thus, if we
ask all implications, or their corresponding GCIs, we can ensure that the computation of
the set $P_{i+1}^{1}$ is done as required in
\Cref{alg:confident-base-gcis/trusted-objects}.

\addfunctionname{model-exploration-by-confidence}

\begin{figure}[tp]
  \begin{Algorithm}
    \label{alg:model-exploration-by-confidence} Model Exploration by Confidence
    \begin{lstlisting}
define model-exploration-by-confidence($\mathcal{I}$, $c$)
  $i$ := $\ell$ := $0$
  $\mathcal{I}_\ell$ := $\mathcal{I}$
  $\bar P_i$ := $\bar{\mathcal{K}}_i$ := $\emptyset$
  $\bar M_i$ := $N_{C} \cup \set{ \bot }$
  $\bar{\mathcal{S}}_i$ := $\set{ \set{ \bot } \to \set{ A } \mid A \in N_{C} }$
  choose ${\leq_{M_{i}}} \text{ as a linear order on } M_{i}$

  forever do
    ;; present new GCI to the expert
    while $\text{expert rejects } \bigsqcap \bar P_i \sqsubseteq (\bigsqcap \bar P_i) ^{\mathcal{I}_\ell\mathcal{I}_\ell}$ do
      $\mathcal{I}_{\ell + 1}$ := $\text{expert-defined extension of } \mathcal{I}_\ell$
      $\ell$ := $\ell$ + 1
    end

    $\bar M_{i+1}$ := $\bar M_i \cup \set{ \exists r. (\bigsqcap P_i)^{\mathcal{I}_\ell\mathcal{I}_\ell} \mid r \in N_R}$ ;; union up to equivalence $\label{item:23}$
    choose ${\leq_{M_{i+1}}} \text{ such that (\ref{eq:48}) and (\ref{eq:49}) hold}$

    ;; ensure relevant counterexamples for already known GCIs
    while $\text{expert rejects } \bigsqcap \bar P_k \sqsubseteq \bigsqcap ( \bar P_k^{\con K_{\mathcal{I}, \bar M_{i+1}}, c} \cap (\bar P_k)^{\prime\prime_{\con K_{\mathcal{I}_\ell \setminus \mathcal{I}, \bar M_{i+1}}}}) \text{ for some } k \in \set{ 0, \dots, n }\label{item:26}$ do
      $\mathcal{I}_{\ell+1}$ := $\text{expert-defined extension of } \mathcal{I}_\ell$
      $\ell$ := $\ell$ + 1
    end

    $\bar{\con K}_{i+1}$ := induced-context($\mathcal{I}$, $\bar M_{i+1}$)$\label{item:24}$
    $\bar{\con L}_{i+1}$ := induced-context($\mathcal{I}_{\ell} \setminus \mathcal{I}$, $\bar M_{i+1}$)  
    $\bar{\mathcal{S}}_{i+1}$ := $\set{ \set{C} \to \set{D} \mid C, D \in \bar M_{i+1}, C \sqsubseteq D }$

    $\bar{\mathcal{K}}_{i+1}$ := $\set{ \bar P_k \to \bar P_k^{\bar{\con K}_{i+1}, c} \cap
      (\bar P_k)^{\prime\prime_{\bar{\con L}_{i+1}}} \mid k \in \set{0, \ldots, i}, \bar P_k \neq
      \bar P_k^{\bar{\con K}_{i+1}, c} \cap (\bar P_k)^{\prime\prime_{\bar{\con L}_{i+1}}} }$

    ;; additional expert interaction $\label{item:25}$
    forall $Q \succeq P$ being $(\bar{\mathcal{K}}_{i+1} \cup \bar{\mathcal{S}}_{i+1})\text{-closed}$ do $\label{ask-many-1}$
      while $\text{expert rejects } \bigsqcap Q \sqsubseteq \bigsqcap Q^{\prime\prime_{\bar{\con L}_{i+1}}}$ do
        $\mathcal{I}_{\ell+1}$ := $\text{expert-defined extension of } \mathcal{I}_\ell$
        $\ell$ := $\ell$ + 1
        $\bar{\con L}_{i+1}$ := induced-context($\mathcal{I}_\ell \setminus \mathcal{I}$, $\bar M_{i+1}$)
      end
    end $\label{ask-many-2}$

    $\bar P_{i+1}^1$ := next-closure($\bar M_{i+1}$, $\leq_{M_{i+1}}$, $\bar P_i$, $\frac{\bar{\con K}_{i+1}}{\bar{\con L}_{i+1}}$)
    $\bar P_{i+1}^2$ := next-closure($\bar M_{i+1}$, $\leq_{M_{i+1}}$, $\bar P_i$, $\bar{\mathcal{S}}_{i+1} \cup \bar{\mathcal{K}}_{i+1}$)

    $P_{i+1}$ := $\min\nolimits_{\preceq}(P_{i+1}^1, P_{i+1}^2)$.
    if $P_{i+1} =$ nil exit

    $i$ := $i + 1$
  end

  return $\bigsqcap \bar{\mathcal{K}}_{i+1}$
end
    \end{lstlisting}
  \end{Algorithm}
\end{figure}

Let us make this argumentation more concrete, and consider
\Cref{alg:model-exploration-by-confidence} as an adaption of
\Cref{alg:confident-base-gcis/trusted-objects} to provide an algorithm for model
exploration by confidence.  Observe that in \Cref{alg:model-exploration-by-confidence}, as
in \Cref{alg:model-exploration}, counterexamples collected into the current working
interpretation $\mathcal{I}_{\ell}$ are supposed to be connected subinterpretations of the
background interpretation.  Furthermore, since required by
\Cref{thm:confident-base-gcis-trusted-objects-is-correct}, we also need to ensure that
$\mathcal{I}_{\ell} \setminus \mathcal{I}$ is a connected subinterpretation of
$\mathcal{I}_{\ell}$, \ie the expert is not allowed to add role-successors from elements
of $\mathcal{I}_{\ell} \setminus \mathcal{I}$ to elements of $\mathcal{I}$ when adding
counterexamples.  These constraints are suppose to be satisfied in every line of the form
\begin{equation*}
  \mathcal{I}_{\ell+1} := \text{expert-defined extension of } \mathcal{I}_\ell
\end{equation*}

To prove that \Cref{alg:model-exploration-by-confidence} indeed provides an algorithm for
model exploration we shall show that this algorithm computes, when applied to
$\mathcal{I}$ and using an expert that represents the background interpretation
$\mathcal{I}_{\mathsf{back}}$, up to equivalence the same intermediate values as
\Cref{alg:confident-base-gcis/trusted-objects} when directly applied to $\mathcal{I}$ and
$\mathcal{I}_{\mathsf{back}}$.  Then, since \Cref{alg:confident-base-gcis/trusted-objects}
terminates, \Cref{alg:model-exploration-by-confidence} will also terminate and returns a
base of $\Th_{c}(\mathcal{I}_{\mathsf{back}}, \mathcal{I})$.

\begin{Theorem}
  \label{thm:model-exploration-by-confidence}
  Let $\mathcal{I}_{\mathsf{back}}$ be a finite interpretation over $N_{C}$ and $N_{R}$,
  and let $\mathcal{I}$ be a connected subinterpretation of $\mathcal{I}_{\mathsf{back}}$
  such that $\mathcal{I}_{\mathsf{back}} \setminus \mathcal{I}$ is also a connected
  subinterpretation of $\mathcal{I}_{\mathsf{back}}$.  Let $c \in [0,1]$.

  Then \Cref{alg:model-exploration-by-confidence} applied to $\mathcal{I}$, $c$ and using
  an expert that represents $\mathcal{I}_{\mathsf{back}} \setminus \mathcal{I}$
  terminates.  If $n$ is the number of iterations of this run, then $\bigsqcap
  \bar{\mathcal{K}}_{n}$ is a base of $\Th_{c}(\mathcal{I}_{\mathsf{back}}, \mathcal{I}) =
  \Th_{c}(\mathcal{I}) \cap \Th(\mathcal{I}_{\mathsf{back}} \setminus \mathcal{I})$.
\end{Theorem}
\begin{Proof}
  We show that \Cref{alg:model-exploration-by-confidence} with the input
  $\mathcal{I}_{\mathsf{back}}$, $\mathcal{I}$ and $c$ has the same output as
  \Cref{alg:confident-base-gcis/trusted-objects}.  Then the claim follows from
  \Cref{thm:confident-base-gcis-trusted-objects-is-correct}.

  To this end, we shall show that for all $i \in \set{ 0, \dots, n }$ it is true that
  $\bar P_{i} = P_{i}, \bar M_{i} = M_{i}, \bar{\mathcal{K}}_{i} = \mathcal{K}$ and
  $\bar{\mathcal{S}}_{i} = \mathcal{S}_{i}$ is true up to equivalence.  In other words, we
  shall show that every element of $\bar P_{i}$ is equivalent to some element in $P_{i}$,
  and vice versa; likewise for $\bar M_{i}$ and $M_{i}$.  Furthermore, we shall show that
  for each implication $(\bar A \to \bar B) \in \bar{\mathcal{K}}_{i}$ there exists an
  implication $(A \to B) \in \mathcal{K}_{i}$ such that $\bar A = A$ and $\bar B = B$ is
  true up to equivalence, and vice versa; likewise for $\bar{\mathcal{S}}_{i}$ and
  $\mathcal{S}_{i}$.

  We shall show these claims by induction over $i$.

  In the following argumentation, we shall not mention the linear orders we choose on the
  sets $\bar M_{i+1}$ and $M_{i+1}$ explicitly.  However, we shall
  choose the linear orders $\leq_{\bar M_{i+1}}$ and $\leq_{M_{i+1}}$ on $\bar M_{i+1}$
  and $M_{i+1}$ such that
  \begin{equation*}
    \bar C \leq_{\bar M_{i+1}} \bar D \iff C \leq_{M_{i+1}} D
  \end{equation*}
  for $\bar C, \bar D \in \bar M_{i+1}, C, D \in M_{i+1}$ and $\bar C \equiv C, \bar D
  \equiv D$.

  \textit{Base Case: }  For $i = 0$, it is true that $\bar P_{i} = \emptyset = P_{i}$,
  $\bar M_{i} = N_{C} \cup \set{ \bot } = M_{i}$, $\bar{\mathcal{K}}_{i} = \emptyset =
  \mathcal{K}_{i}$ and $\bar{\mathcal{S}}_{i} = \set{ \set{ \bot } \to \set{ A } \mid A
    \in N_{C} } = \mathcal{S}_{i}$.  Thus, the claim holds for $i = 0$.

  \textit{Step Case: } Let us assume that $0 < i < n$ and that the claim holds for all $m
  \leq i$.

  Denote with $\mathcal{I}_{l}$ the current working interpretation when the algorithm has
  reached line~\ref{item:23}.  The algorithm can only reach this line if the expert has
  confirmed $\bigsqcap \bar P_{i} \sqsubseteq (\bigsqcap \bar
  P_{i})^{\mathcal{I}_{l}\mathcal{I}_{l}}$.  Since $\mathcal{I}_{l}$ is a connected
  subinterpretation of $\mathcal{I}_{\mathsf{back}}$, \Cref{lem:Felix-6.14} implies that
  that $(\bigsqcap \bar P_{i})^{\mathcal{I}_{l}\mathcal{I}_{l}} = (\bigsqcap \bar
  P_{i})^{\mathcal{I}_{\mathsf{back}}\mathcal{I}_{\mathsf{back}}}$.  Since $\bar M_{i} =
  M_{i}$ up to equivalence holds by induction hypothesis, it is therefore true that $\bar
  M_{i+1} = M_{i+1}$ up to equivalence.  This also implies $\bar{\mathcal{S}}_{i} =
  \mathcal{S}_{i}$ up to equivalence, since the definition of these sets only depends on
  $\bar M_{i+1}$ and $M_{i}$, respectively.

  To show $\bar{\mathcal{K}}_{i} = \mathcal{K}$, it is sufficient to verify that
  \begin{equation}
    \label{eq:56}
    \bar P_{k}^{\bar{\con K}_{i+1}, c} \cap (\bar P_{k})_{\bar{\con L}_{i+1}}'' =
    P_{k}^{\con K_{i+1}, c} \cap (P_{k})_{\con L_{i+1}}''
  \end{equation}
  is true up to equivalence for all $k \in \set{ 0, \dots, i }$.  To this end we first
  observe that $\bar{\con K}_{i+1}$ is the induced context of $\mathcal{I}$ and $\bar
  M_{i+1}$, and $\con K_{i+1}$ is the induced context of $\mathcal{I}$ and $M_{i+1}$.  In
  particular, since $\bar M_{i+1} = M_{i+1}$ and $\bar P_{k} = P_{k}$ holds up to
  equivalence for all $0 \leq k \leq i$, we obtain
  \begin{equation*}
    \bar P_{k}^{\bar{\con K}_{i+1}, c} = P_{k}^{\con K_{i+1}, c}
  \end{equation*}
  up to equivalence for all $k \in \set{ 0, \dots, i }$.

  Let $\mathcal{I}_{m}$ be the current working interpretation in iteration $i$ when
  line~\ref{item:24} is reached.  Recall that $\bar{\con L}_{i+1}$ is the induced context
  of $\mathcal{I}_{m} \setminus \mathcal{I}$ and $\bar M_{i+1}$, and that $\con L_{i+1}$
  is the induced context of $\mathcal{I}_{\mathsf{back}} \setminus \mathcal{I}$ and
  $M_{i+1}$.  Since $\mathcal{I}_{m} \setminus \mathcal{I}$ is a connected
  subinterpretation of $\mathcal{I}_{\mathsf{back}} \setminus \mathcal{I}$, we can
  consider $\bar{\con L}_{i+1}$ as a subcontext of $\con L_{i+1}$, and thus obtain
  \begin{equation*}
    (\bar P_{k})_{\bar{\con L}_{i+1}}'' \supseteq (P_{k})_{\con L_{i+1}}''
  \end{equation*}
  up to equivalence for all $k \in \set{ 0, \dots, i }$.

  Assume now by contradiction that
  \begin{equation*}
    \bar P_{k}^{\bar{\con K}_{i+1}, c} \cap (\bar P_{k})_{\con L_{i+1}}'' \neq
    P_{k}^{\bar{\con K}, c} \cap (P_{k})_{\con L_{i+1}}''
  \end{equation*}
  is true for some $k \in \set{ 0, \dots, i }$.  By the above considerations, this means
  that there exists a concept description $\bar C \in \bar P_{k}^{\bar{\con K}_{i+1}, c}
  \cap (\bar P_{k})_{\bar{\con L}_{i+1}}''$ that is not equivalent to any element in
  $P_{k}^{\con K_{i+1}, c} \cap (P_{k})_{\con L_{i+1}}''$.  Then
  \begin{enumerate}[i. ]
  \item the expert has confirmed the implication
    \begin{equation*}
      \bigsqcap \bar P_{k} \sqsubseteq \bigsqcap \bar P_{k}^{\bar{\con K}_{i+1}, c} \cap
      (\bar P_{k})_{\bar{\con L}_{i+1}}''
    \end{equation*}
    since it was proposed to her in line~\ref{item:26}.  In particular, the GCI $\bigsqcap
    \bar P_{k} \sqsubseteq \bar C$ is valid in $\mathcal{I}_{\mathsf{back}} \setminus
    \mathcal{I}$.
  \item The GCI $\bigsqcap P_{k} \sqsubseteq \bar C$ is not confirmed by the expert.  To
    see this, observe that since $\bar C \in M_{i+1}$, there exists $C \in M_{i+1}$ such
    that $\bar C \equiv C$.  Then if $\bigsqcap P_{k} \sqsubseteq \bar C$ were confirmed
    by the expert, it would be true that $C \in (P_{k})_{\con L_{i+1}}''$.  Furthermore,
    it is true that $\conf_{\bar{\con K}_{i+1}}(\bar P_{k} \to \set{ \bar C }) =
    \conf_{\con K_{i+1}}(P_{k} \to \set{ C })$ because of $\bar P_{k} = P_{k}, \bar
    M_{i+1} = M_{i+1}$ up to equivalence, and $\bar C \equiv C$.  Since $\bar C \in \bar
    P_{k}^{\bar{\con K}_{i+1}, c}$, it is true that $\conf_{\bar{\con K}_{i+1}}(\bar P_{k}
    \to \set{ \bar C }) \ge c$, and thus $\conf_{\con K_{i+1}}(P_{k} \to \set{ C }) \ge
    c$, and hence $C \in P_{k}^{\con K_{i+1}, c}$.  Thus, we obtain
    \begin{equation*}
      \bar C \equiv C \quad\text{and}\quad C \in P_{k}^{\con K_{i+1}, c} \cap
      (P_{k})_{\con L_{i+1}}'',
    \end{equation*}
    contradicting our choice of $\bar C$.  Therefore, $\bigsqcap P_{k} \sqsubseteq \bar C$
    is not confirmed by the expert, and in particular is not valid in
    $\mathcal{I}_{\mathsf{back}} \setminus \mathcal{I}$.
  \end{enumerate}
  However, since $\bar P_{k} = P_{k}$ up to equivalence, the fact that $\bigsqcap \bar
  P_{k} \sqsubseteq \bar C$ is valid in $\mathcal{I}_{\mathsf{back}} \setminus
  \mathcal{I}$ but $\bigsqcap P_{k} \sqsubseteq \bar C$ is not yields the desired
  contradiction.  Therefore, we have established the validity of \Cref{eq:56}, and thus
  can infer that $\bar{\mathcal{K}}_{k} = \mathcal{K}_{k}$ is true up to equivalence.

  It remains to be shown that
  \begin{equation}
    \label{eq:57}
    \bar P_{i+1} = P_{i+1}
  \end{equation}
  is true up to equivalence.  To this end, we observe that $\bar P_{i+1}^{2} =
  P_{i+1}^{2}$ is true up to equivalence, since $\bar P_{i} = P_{i}$ and
  $\bar{\mathcal{K}}_{i+1} \cup \bar{\mathcal{S}}_{i+1} = \mathcal{K}_{i+1} \cup
  \mathcal{S}_{i+1}$ up to equivalence, and the linear orders on $\bar M_{i+1}$ and
  $M_{i}$ are chosen suitably.  Thus, to show \Cref{eq:57} it suffices to verify that
  \begin{equation}
    \label{eq:58}
    \bar P_{i+1}^{1} = P_{i+1}^{1}.
  \end{equation}
  For this recall that the formal contexts $\bar{\con K}_{i+1}$ and $\con K_{i+1}$ can be
  considered the same, because $\bar M_{i+1} = M_{i+1}$ up to equivalence.  Also recall
  that we can consider $\bar{\con L}_{i+1}$ as a subcontext of $\con L_{i+1}$, again up to
  equivalence.  Therefore, we obtain $\bar P_{i+1}^{1} \succeq P_{i+1}^{1}$ up to
  equivalence.

  Suppose by contradiction that $\bar P_{i+1}^{1} \succneq P_{i+1}^{1}$.  Then
  $P_{i+1}^{1}$, viewed as a subset of $\bar M_{i+1}$, is not an intent of $\bar{\con
    K}_{i+1} \div \bar{\con L}_{i+1}$, as otherwise $\bar P_{i+1}^{1} \preceq
  P_{i+1}^{1}$.  Since $P_{i+1}^{1}$ is an intent of $\con K_{i+1} \div \con L_{i+1}$, we
  can thus infer that
  \begin{equation*}
    (P_{i+1}^{1})_{\bar{\con L}_{i+1}}'' \setminus (P_{i+1}^{1})_{\con L_{i+1}}'' \neq \emptyset.
  \end{equation*}
  Let $D \in (P_{i+1}^{1})_{\bar{\con L}_{i+1}}'' \setminus (P_{i+1}^{1})_{\con
    L_{i+1}}''$.  Since $\bar{\mathcal{K}}_{i+1} \cup \bar{\mathcal{S}}_{i+1}$ is sound
  for $\con L_{i+1}$ up to equivalence, we obtain that
  \begin{equation*}
    ((\bar{\mathcal{K}}_{i+1} \cup \bar{\mathcal{S}}_{i+1})(P_{i+1}^{1}))_{\con L_{i+1}}''
    = (P_{i+1}^{1})_{\con L_{i+1}}''
  \end{equation*}
  and thus
  \begin{equation*}
    D \not\in ((\bar{\mathcal{K}}_{i+1} \cup \bar{\mathcal{S}}_{i+1})(P_{i+1}^{1}))_{\con L_{i+1}}''.
  \end{equation*}
  Therefore, the corresponding implication
  \begin{equation*}
    (\bar{\mathcal{K}}_{i+1} \cup \bar{\mathcal{S}}_{i+1})(P_{i+1}^{1})
    \to ((\bar{\mathcal{K}}_{i+1} \cup \bar{\mathcal{S}}_{i+1})(P_{i+1}^{1}))_{\bar{\con L}_{i+1}}''
  \end{equation*}
  does not hold in $\con L_{i+1}$, because
  \begin{equation*}
    ((\bar{\mathcal{K}}_{i+1} \cup \bar{\mathcal{S}}_{i+1})(P_{i+1}^{1}))_{\bar{\con L}_{i+1}}''
    \not\subseteq ((\bar{\mathcal{K}}_{i+1} \cup
    \bar{\mathcal{S}}_{i+1})(P_{i+1}^{1}))_{\con L_{i+1}}''.
  \end{equation*}
  Therefore, the corresponding GCI
  \begin{equation}
    \label{eq:59}
    \bigsqcap(\bar{\mathcal{K}}_{i+1} \cup \bar{\mathcal{S}}_{i+1})(P_{i+1}^{1})
    \sqsubseteq \bigsqcap ((\bar{\mathcal{K}}_{i+1} \cup
    \bar{\mathcal{S}}_{i+1})(P_{i+1}^{1}))_{\bar{\con L}_{i+1}}''
  \end{equation}
  will be rejected by the expert, since $\con L_{i+1}$ is the induced context of
  $\mathcal{I}_{\mathsf{back}} \setminus \mathcal{I}$.

  However, when computing $P_{i+1}^{1}$, we have passed the lines~\ref{ask-many-1} up
  to~\ref{ask-many-2}, and the expert has confirmed the GCI given in
  \Cref{eq:59}. Therefore, our initial assumption that $\bar P_{i+1}^{1} \succneq
  P_{i+1}^{1}$ is not true, and thus we obtain $\bar P_{i+1}^{1} = P_{i+1}^{1}$ up to
  equivalence.  This finishes the proof.
\end{Proof}

% Note that in the proof we have
% \begin{equation*}
%   (\bar{\mathcal{K}}_{i+1} \cup \bar{\mathcal{S}}_{i+1})(P_{i+1}^{1}) = P_{i+1}^{1},
% \end{equation*}
% since $P_{i+1}^{1}$ is an intent of $\con K_{i+1} \div \con L_{i+1}$ and
% $\bar{\mathcal{K}}_{i+1} \cup \bar{\mathcal{S}}_{i+1}$ is sound for $\con K_{i+1} \cup
% \con L_{i+1}$ up to equivalence.  Therefore, it suffices in line~\ref{ask-many-1} in
% \Cref{alg:model-exploration-by-confidence} to ask only those GCIs
% \begin{equation*}
%   \bigsqcap Q \sqsubseteq \bigsqcap Q_{\bar{\con L}_{i+1}}''
% \end{equation*}
% where $Q$ is $\bar{\mathcal{K}}_{i+1} \cup \bar{\mathcal{S}}_{i+1}$-closed and $Q$ is
% lectically smaller or equal to the smallest intent of $\bar{\con K}_{i+1} \div \bar{\con
%   L}_{i+1}$, which is greater or equal to $\bar P_{i}$.

%%% Local Variables: 
%%% mode: latex
%%% TeX-master: "../main"
%%% End: 

%  LocalWords:  gcis
