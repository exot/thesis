\chapter[Model Exploration by Confidence]{Model Exploration by Confidence with Completely Specified Counterexamples}
\label{cha:model-expl-conf}

We have seen how we can extend our understanding of attribute exploration to include the
possibility to explore implications which enjoy a high confidence in some given formal
context.  In this chapter we want to extend this generalization of attribute exploration
even further to be able to explore GCIs with high confidence in finite interpretations.
The basis for this extension will be Distel's \emph{model exploration}
algorithm~\cite{Diss-Felix}, an extension of attribute exploration to explore valid GCIs
of finite interpretations.

Model exploration, similarly to attribute exploration, assumes that a certain domain of
interest is representable by a finite interpretation $\mathcal{I}_{\mathsf{back}}$, which
we shall call the \emph{background interpretation} of the exploration process.  However,
we assume that this interpretation is not directly accessible, but instead we have given
an expert that allows us to decide whether certain GCIs are valid in
$\mathcal{I}_{\mathsf{back}}$ or not.  In addition, if a given GCI $C \sqsubseteq D$ is
not valid in $\mathcal{I}_{\mathsf{back}}$, then the expert provides a counterexample for
$C \sqsubseteq D$ in a suitable way.

The principal way how model exploration now works is again very akin to attribute
exploration.  Given a finite \emph{connected subinterpretation} $\mathcal{I}$ of
$\mathcal{I}_{\mathsf{back}}$ and a set $\mathcal{B}$ of valid GCIs of
$\mathcal{I}_{\mathsf{back}}$, the algorithm successively generates valid GCIs $C
\sqsubseteq D$ of $\mathcal{I}$ which do not follow from $\mathcal{B}$ and presents them
to some expert.  If the expert confirms $C \sqsubseteq D$, then it is added to
$\mathcal{B}$.  If the expert rejects $C \sqsubseteq D$, then she provides a
counterexample to it in the form of a connected subinterpretation of
$\mathcal{I}_{\mathsf{back}}$, which is added to $\mathcal{I}$.  Since $\mathcal{I}$ and
$\mathcal{B}$ thus play the same role as the working context and the set of known
implications during attribute exploration, we shall refer to them as the \emph{working
  interpretation} and the \emph{set of known GCIs}, respectively.  If no more GCIs can be
generated to be asked to the expert the algorithm stops.  It can be shown that at this
point, the set $\mathcal{B}$ is a finite base of $\mathcal{I}_{\mathsf{back}}$.

The foregone description already suggests that there are some difficulties in transferring
attribute exploration to the setting of GCIs and finite interpretations.  The most
apparent is that the set of GCIs we potentially have to cover is infinite, as the set of
valid GCIs of $\mathcal{I}_{\mathsf{back}}$ is infinite.  The other problem is that
validity of GCIs in interpretations deploy a \emph{closed-world} semantics.

\section{Model Exploration with Valid GCIs}
\label{sec:model-expl-with}

\todo[inline]{Write: 13-11/4}

\subsection{Growing Sets of Attributes}
\label{sec:grow-sets-attr}

\subsection{Exploring Valid GCIs with Known Background Interpretation}
\label{sec:comp-bases-given}

\subsection{An Algorithm for Exploring Interpretations}
\label{sec:an-algor-expl}

\section{Model Exploration with Confident GCIs}
\label{sec:model-expl-with-1}

\todo[inline]{Write: 13-11/5}%

\subsection{Trusted and Untrusted Individuals}
\label{sec:trust-untr-indiv}

\subsection{Growing Sets of Attributes}
\label{sec:grow-sets-attr-1}

\subsection{Exploring Confident GCIs with Known Background Interpretation}
\label{sec:expl-conf-gcis}

\subsection{Exploring Confident GCIs with Expert Interaction}
\label{sec:expl-conf-gcis-1}

%%% Local Variables: 
%%% mode: latex
%%% TeX-master: "../main"
%%% End: 
