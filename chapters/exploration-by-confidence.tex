\chapter{Exploration by Confidence}
\label{cha:expl-conf}

Recall that we have introduced GCIs with high confidence as an approach to extract
terminological knowledge from erroneous data.  For this approach we have seen in
\Cref{sec:comp-conf-bases} that the thus obtained GCIs require an additional validation
step, \ie an expert has to verify whether the extracted GCIs, whose confidence in the data
$\mathcal{I}$ is above a pre-chosen threshold $c \in [0,1]$, are indeed valid in the
domain of interest, \ie whether the counterexamples present in the data are valid or not.

This additional validation step can be very costly, and thus it should be avoided as much
as possible.  On the other hand, we have also seen in \Cref{sec:comp-conf-bases} that as
soon as some GCI has been confirmed, others may be entailed by it, and a manual validation
is no longer necessary.  This observation indeed may save a lot of work.

Indeed, utilizing this observation may also save computation time as well: if during the
computation of bases of $\Th_c(\mathcal{I})$ an expert already rejects a GCI and confirms
that some counterexamples to it which are present in $\mathcal{I}$ are indeed correct,
then all GCIs which are falsified by this counterexamples can also be rejected,
irrespective of whether their confidence in $\mathcal{I}$ is above $c$.

Having these thoughts in mind it may seem desirable to allow expert interaction already
during the time of the computation of bases of $\Th_c(\mathcal{I})$, and not only at the
end of the computation, as a separate validation step.

There is also another observation which requires expert interaction, and which has already
been addressed by Distel~\cite{Diss-Felix}: the data $\mathcal{I}$ may not only contain
errors, but it may also be \emph{incomplete} in the sense that it lacks certain
counterexamples.  This would mean that some GCIs are valid in $\mathcal{I}$, even so they
are not valid in the domain of interest.  If $\mathcal{I}$ may contain errors, it may
happen that some invalid GCIs $C \sqsubseteq D$ are only falsified in $\mathcal{I}$ by
erroneous counterexamples, and that correct counterexamples in $\mathcal{I}$ are indeed
not present.  In such a case, the GCI $C \sqsubseteq D$ would be accepted (since all
counterexamples are erroneous) although it is not valid.  Here an expert could, as soon as
$C \sqsubseteq D$ would be computed as an element of a base, provide correct
counterexamples, inhibiting this GCI from being included in a base.  Such provided
counterexamples would also affect the following computation, as all GCIs invalided by it
would also be rejected.

The approach followed by Distel to solve this problem is to adapt attribute exploration
from its original setting of implications and formal contexts to the setting of GCIs and
finite interpretations.  Recall that the attribute exploration algorithm, which we had
already discussed previously in \Cref{sec:attr-expl}, saves a problem which is very
similar to the one sketched above: given a finite formal context $\con K$, we want to
compute a base of $\con K$.  However, it may be the case that $\con K$ does not completely
represent the domain we are interested in, \ie some implications which are not valid in
our domain actually hold in $\con K$.  Then attribute exploration provides an interactive
base computation procedure which, as soon as an implication is computed, presents it to
the expert and ask for validation.  If the expert confirms this implication, then the
implication is added to the base.  If the expert does not confirm this implication, she
has to provide a counterexample, which is included in $\con K$.  At the end, the attribute
exploration algorithm yields a base of the domain we are interested in, and not only of
the formal context $\con K$.

Distel's generalizations of attribute exploration to the setting of GCIs and finite
interpretation are called \emph{model exploration} and \emph{ABox exploration}.  Their
abstract behavior is very similar to the one of attribute exploration: during the
computation of a base of $\mathcal{I}$, GCIs $C \sqsubseteq D$ are asked to the expert.
If confirmed, they are added to the base.  If rejected, the expert has to provide a
counterexample which is being added to $\mathcal{I}$.  Upon termination, the algorithm
yields a base of the domain which is represented by the expert.

Model exploration and ABox exploration differ in the way counterexamples are provided, and
we shall discuss the difficulties in some more details in the following chapter.  In model
exploration, counterexamples are directly added to the interpretation $\mathcal{I}$,
whereas in ABox exploration the counterexamples are being added to a separate ABox.  While
the latter is much more user-friendly from the point of view how much needs to be
specified from a counterexample, it is also much more technical and complicated.
Therefore, in this work we shall concentrate on model exploration only.

In the following two chapters we want to generalize model exploration to our setting of
extracting GCIs with high confidence from finite interpretations.  More precisely, what we
want to obtain is an algorithm that computes bases of $\Th_c(\mathcal{I})$, and, as soon
as a GCI for the base is computed, asks the expert for validation.  As above, the expert
may confirm or reject this GCI, where in the latter case she has to provide a
counterexample.  Upon termination, the algorithm should yield a base of the domain which
is represented by the expert, to the extent in which this domain is contained in
$\Th_c(\mathcal{I})$.  We shall make this more precise in \Cref{sec:expl-conf}.

However, before we can discuss model exploration in our setting of GCIs with high
confidence, we first have to adapt attribute exploration to the setting of implications
with high confidence.  This \emph{exploration by confidence} should work the same way as
attribute exploration works, with the notable difference that not only valid implications
are asked, but also those which have a high confidence in the formal context.  Within this
generalization, we have to deal with the problem that counterexamples provided by the
expert are not subject to the confidence measure, as we assume that the expert only gives
valid counterexamples: a single counterexample provided by the expert suffices to reject a
GCIs, even if it otherwise has enough confidence in the data.  We shall discuss this
problem along with its solution in detail in \Cref{sec:expl-conf}.

Our discussion of exploration by confidence will be based on a more general discussion of
attribute exploration, which we shall conduct in \Cref{sec:an-abstract-view}.  Therein, we
shall view attribute exploration from a more general and more formal perspective.  This
generalization will allow us to not only explore formal contexts, but also arbitrary sets
of implications, and in our discussion of exploration by confidence these sets will be of
the form $\Th_c(\con K)$.  For this, we shall introduce two variants of a
\emph{generalized attribute exploration}, one which in its technical details is very
closely related to attribute exploration, and one which is more liberal in the way
implications are computed.  Finally, we shall also give a formal specification an
\emph{expert} has to satisfy for our exploration algorithms (including the classical
attribute exploration) to work.

\section{An Abstract View on Attribute Exploration}
\label{sec:an-abstract-view}

\todo[inline]{Write: give abstract view on exploration}%
\todo[inline]{Write: formalize notion of expert}%

\subsection{A Generalized Form of Attribute Exploration}
\label{sec:gener-form-attr}

\todo[inline]{Write: 13-04/3.2}

\subsection{A Weaker Generalization of Attribute Exploration}
\label{sec:weak-gener-attr}

\todo[inline]{Write: 13-04/3.3}

\section{Exploration by Confidence}
\label{sec:expl-conf}

\todo[inline]{Write: 13-04/4}

\subsection{Exploration by Confidence with Optimality}
\label{sec:expl-conf-1}

\todo[inline]{Write: 13-04/4.1}

\subsection{An Exploration by Confidence with Faster Generation of Questions}
\label{sec:poss-fast-expl}

\todo[inline]{Write: 13-04/4.2}

%%% Local Variables: 
%%% mode: latex
%%% TeX-master: "../main"
%%% End: 
