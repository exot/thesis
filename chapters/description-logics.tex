\chapter{Description Logics}
\label{cha:description-logics}

Description logics~\cite{DLhandbook} are a family of logic-based formalisms for knowledge
representation, which originated in the 1980s as an attempt to provide knowledge
representation formalisms with well-defined semantics.  This was mostly inspired by other
formalisms which lacked such well-defined semantics, such as \emph{semantic
  networks}~\cite{SemanticNetworks} or \emph{frames}~\cite{Minsky-Frames}.  In those
formalisms, the way the representation of the knowledge was interpreted either depended on
the intuition of the particular observer or details of a corresponding implementation.
These properties are undesired if one wants to bring precise knowledge to a machine.

In this section, we shall introduce some basic notions of description logics as they are
needed for the purpose of this work.  This shall include a discussion of some of the basic
\emph{syntax} and \emph{semantics} of description logics
(Section~\ref{sec:basic-noti-descr}), as well as a discussion of \emph{knowledge bases}
and \emph{standard reasoning tasks} (Section~\ref{sec:knowledge-bases}).

Our focus in this work will be upon the light-weight description logic \ELbot, which shall
be the target logic for the knowledge we want to extract from data.  However, at a later
point in this work we shall see that the expressivity of \ELbot is not sufficient for our
needs, and that we need to consider an extension of \ELbot instead.  This extension shall
be the logic \ELgfpbot, an extension of \ELbot by \emph{greatest fixpoint semantics}.  We
shall introduce the syntax and semantics of \ELgfpbot in
Section~\ref{sec:descr-logics-elbot}.

\section{Syntax and Semantics of Basic Description Logics}
\label{sec:basic-noti-descr}

\todo[inline]{Write: some syntax of DLs}%
\todo[inline]{Write: introduce \ELbot}%
\todo[inline]{Write: interpretations and semantics of DLs}%

\section{Knowledge Bases}
\label{sec:knowledge-bases}

\todo[inline]{Write: cyclic and general TBoxes}%
\todo[inline]{Write: ABoxes, knowledge bases}%
\todo[inline]{Write: standard reasoning tasks}%

\section{The Description Logic \ELgfpbot}
\label{sec:descr-logics-elbot}

\todo[inline]{Write: talk about model-base most-specific concept descriptions}%
\todo[inline]{Write: introduce \ELgfpbot}%
\todo[inline]{Write: introduce unravelling of \ELgfpbot concept descriptions}

%%% Local Variables: 
%%% mode: latex
%%% TeX-master: "../main"
%%% End: 
