\chapter{Description Logics}
\label{cha:description-logics}

Description logics~\cite{DLhandbook} are a family of logic-based formalisms for knowledge
representation, which originated in the 1980s as an attempt to provide knowledge
representation formalisms with well-defined semantics.  This was mostly inspired by other
formalisms which lacked such well-defined semantics, such as \emph{semantic
  networks}~\cite{SemanticNetworks} and \emph{frames}~\cite{Minsky-Frames}.  In those
formalisms, the way the representation of the knowledge was interpreted either depended on
the intuition of the particular observer or details of a corresponding implementation.
These properties are undesired if one wants to bring precise knowledge to a machine.

In this section, we shall introduce some basic notions of description logics as they are
needed for the purpose of this work.  This shall include a discussion of some of the basic
\emph{syntax} and \emph{semantics} of description logics
(Section~\ref{sec:basic-noti-descr}), as well as a discussion of \emph{knowledge bases}
and \emph{standard reasoning tasks} (Section~\ref{sec:knowledge-bases}).

Our focus in this work will be upon the light-weight description logic \ELbot, which shall
be the target logic for the knowledge we want to extract from data.  However, at a later
point in this work we shall see that the expressivity of \ELbot is not sufficient for our
needs, and that we need to consider an extension of \ELbot instead.  This extension shall
be the logic \ELgfpbot, an extension of \ELbot by \emph{greatest fixpoint semantics}.  We
shall therefore also include an introduction to the syntax and semantics of \ELgfpbot in
this chapter.  This will be done in Section~\ref{sec:descr-logics-elbot}.

\section{Syntax and Semantics of the Description Logic \ELbot}
\label{sec:basic-noti-descr}

We start our introduction to description logics by defining the syntax of the description
logic \ELbot, one of the simplest and least-expressive description logics under
consideration.  To this end, we fix two disjoint sets $N_C$ and $N_R$, called the sets of
\emph{concept names} and \emph{role names}, respectively.  Based upon this choice we
introduce \emph{\ELbot concept descriptions} over $N_C$ and $N_R$ as follows.
\begin{Definition}
  \label{def:ELbot-concept-descriptions}
  Let $N_C$ and $N_R$ be two disjoint sets.  An \emph{\EL concept description} $C$ (over
  $N_C$ and $N_R$) is of the form
  \begin{itemize}
  \item $C = A$ for some $A \in N_C$, or
  \item $C = C_1 \sqcap C_2$ for $C_1, C_2$ two \EL concept descriptions, or
  \item $C = \exists r. D$ for a \EL concept description $D$ and $r \in N_R$, or
  \item $C = \top$.
  \end{itemize}
  The constructors are called \emph{conjunction} ($\sqcap$), \emph{existential
    restriction} ($\exists$) and the \emph{top concept} ($\top$), respectively.

  An \ELbot concept description $C$ (over $N_C$ and $N_R$) is then either an \EL concept
  description or $C$ is of the form $C = \bot$.
\end{Definition}

We shall occasionally denote the set of all \EL concept descriptions over $N_C$ and $N_R$
by $\EL(N_C, N_R)$, and likewise the set of all \ELbot concept descriptions over $N_C$ and
$N_R$ by $\ELbot(N_C, N_R)$.  The same is true for other descriptions logics which we
shall introduce at a later point.

The sets $N_C$ and $N_R$ of concept and role names are also called the \emph{signature} of
the logic.  In the following we shall always assume that those sets are finite.  Sometimes
this signature is extended by another set $N_I$ called the sets of \emph{individual
  names}.  Such individual names allow a logic to directly refer to individual elements of
a domain.  However, this expressiveness is not available in the description logics \EL and
\ELbot, but we shall nevertheless include them in our further discussion as they are
needed to define the semantics of assertional axioms.  See also
Section~\ref{sec:knowledge-bases}.

\begin{Example}
  \label{expl:tom-and-jerry-1}
  Let $N_C := \set{ \mathsf{Cat}, \mathsf{Mouse} }$ and let $N_R := \set{ \mathsf{hunts}
  }$.  Then examples of \ELbot concept descriptions would be
  \begin{equation*}
    \mathsf{Cat}, \, \mathsf{\exists hunts. Mouse}, \, \mathsf{Cat \sqcap Mouse}, \, \bot.
  \end{equation*}
  One can think of these concept descriptions as ``describing'' things in a certain domain
  (we shall make this much more precise shortly).  For example, the concept description
  $\mathsf{Cat}$ describes everything which is a cat.  The concept description
  $\mathsf{\exists hunts. Mouse}$ describes everything which hunts a mouse (alternatively:
  a things that hunts something which is a mouse).  The concept description $\mathsf{Cat
    \sqcap Mouse}$ describes things which are both a cat and a mouse.  Finally, the
  concept description just describes nothing.
\end{Example}

The intuition of a concept description to ``describe things'' conveyed in the previous
example is very vague, and we shall make it more precise by introducing the notion of an
\emph{interpretation}.

\begin{Definition}[Interpretation]
  \label{def:interpretation}
  Let $N_C$, $N_R$ and $N_I$ be three pairwise disjoint sets.  An \emph{interpretation}
  $\mathcal{I} = (\Delta^{\mathcal{I}}, \cdot^{\mathcal{I}})$ (over $N_C$, $N_R$ and
  $N_I$) consists of a set $\Delta^{\mathcal{I}}$ of \emph{elements}, and an
  \emph{interpretation function} $\cdot^{\mathcal{I}}$ such that
  \begin{enumerate}[i. ]
  \item $A^{\mathcal{I}} \subseteq \Delta^{\mathcal{I}}$ for all $A \in N_C$,
  \item $r^{\mathcal{I}} \subseteq \Delta^{\mathcal{I}} \times \Delta^{\mathcal{I}}$ for
    all $r \in N_R$, and
  \item $a^{\mathcal{I}} \in \Delta^{\mathcal{I}}$ for all $a \in N_I$.
  \end{enumerate}
  We furthermore make the \emph{unique name assumption}: if $a, b \in N_I$ and $a \neq b$,
  then $a^{\mathcal{I}} \neq b^{\mathcal{I}}$.

  We say that $\mathcal{I}$ is finite if $\Delta^{\mathcal{I}}$ is finite.
\end{Definition}

If the set $N_I$ in the definition of an interpretation is not important for the current
consideration, then we shall set $N_I = \emptyset$ and just speak of an interpretation
over $N_C$ and $N_R$.

\begin{Example}
  \label{expl:tom-hunts-jerry-hunts-tom-interpretation}
  Let us consider an example interpretation $\mathcal{I}_{\mathsf{MGM}}$ for our signature
  $N_C = \set{ \mathsf{Cat}, \mathsf{Mouse} }$ and $N_R = \set{ \mathsf{hunts} }$.  For
  this we set
  \begin{equation*}
    \mathcal{I}_{\mathsf{MGM}} := (\set{ \mathsf{tom}, \mathsf{jerry} }, \cdot^{\mathcal{I}_{\mathsf{MGM}}})
  \end{equation*}
  where
  \begin{align*}
    \mathsf{Cat}^{\mathcal{I}_{\mathsf{MGM}}} &= \set{ \mathsf{tom} }, \\
    \mathsf{Mouse}^{\mathcal{I}_{\mathsf{MGM}}} &= \set{ \mathsf{jerry} }, \\
    \mathsf{hunts}^{\mathcal{I}_{\mathsf{MGM}}} &= \set{ (\mathsf{tom}, \mathsf{jerry}),
      (\mathsf{jerry}, \mathsf{tom}) }.
  \end{align*}
  The interpretation $\mathcal{I}_{\mathsf{MGM}}$ can naturally be represented as a graph,
  where we consider the elements of $\Delta^{\mathcal{I}_{\mathsf{MGM}}}$ as vertices and
  the pairs in $\mathsf{hunts}^{\mathcal{I}_{\mathsf{MGM}}}$ as directed edges:
  \begin{center}
    \begin{tikzpicture}[>=stealth']
      \begin{scope}[every node/.style = { draw, circle }]
        \node (Tom) {\textsf{tom}};
        \node[right=2cm of Tom, inner sep = .1cm] (Jerry) {\textsf{jerry}}; % fix
      \end{scope}
      \path (Tom) edge[->, bend left=30] node[midway, above] {\textsf{hunts}} (Jerry);
      \path (Jerry) edge[->, bend left=30] node[midway, below] {\textsf{hunts}} (Tom);
    \end{tikzpicture}
  \end{center}
\end{Example}

The interpretation function of a given interpretation can now be naturally extended to the
set of \EL and \ELbot concept descriptions.

\begin{Definition}
  \label{def:extending-the-interpretation-function}
  Let $\mathcal{I} = (\Delta^{\mathcal{I}}, \cdot^{\mathcal{I}})$ be an interpretation.
  Then the \emph{extension} of $\cdot^{\mathcal{I}}$ to all \ELbot concept descriptions is
  inductively defined as follows: let $C_1, C_2, C \in \ELbot(N_C, N_R)$ and $r \in N_R$.
  Then
  \begin{enumerate}[i. ]
  \item $\top^{\mathcal{I}} = \Delta^{\mathcal{I}}$,
  \item $\bot^{\mathcal{I}} = \emptyset$,
  \item $(C_1 \sqcap C_2)^{\mathcal{I}} := C_1^{\mathcal{I}} \cap C_2^{\mathcal{I}}$, and
  \item $(\exists r. C)^{\mathcal{I}} := \set{ x \in \Delta^{\mathcal{I}} \mid \exists y
      \in \Delta^{\mathcal{I}} \st (x, y) \in r^{\mathcal{I}} \land y \in C^{\mathcal{I}} }$.
  \end{enumerate}
  We shall call the set $C^{\mathcal{I}}$ the \emph{extension} of $C$ in $\mathcal{I}$.
\end{Definition}

The notion of an extension of an \ELbot concept description in an interpretation now
formalizes our previous intuitive understanding that concept descriptions ``describe
things:'' an element $x \in \Delta^{\mathcal{I}}$ is described by a \ELbot concept
description $C$ in $\mathcal{I}$ if and only if $x \in C^{\mathcal{I}}$.

\begin{Example}
  \label{expl:tom-and-jerry-2}
  Let us consider the example interpretation $\mathcal{I}_{\mathsf{MGM}}$ from
  Example~\ref{expl:tom-hunts-jerry-hunts-tom-interpretation} again, and let us compute
  for the example \ELbot concept descriptions from Example~\ref{expl:tom-and-jerry-1}
  their extensions.  We obtain
  \begin{align*}
    (\mathsf{\exists hunts. Mouse})^{\mathcal{I}_{\mathsf{MGM}}} &= \set{ x \in
      \Delta^{\mathcal{I}_{\mathsf{MGM}}} \mid \exists y \in
      \Delta^{\mathcal{I}_{\mathsf{MGM}}} \st (x, y) \in
      \mathsf{hunts}^{\mathcal{I}_{\mathsf{MGM}}} \land y \in
      \mathsf{Mouse}^{\mathcal{I}_{\mathsf{MGM}}} }\\
    &= \set{ \mathsf{tom} } \\
    &= \mathsf{Cat}^{\mathcal{I}_{\mathsf{MGM}}},\\
    \mathsf{Cat \sqcap Mouse}^{\mathcal{I}_{\mathsf{MGM}}} &= \emptyset \\
    &= \bot^{\mathcal{I}_{\mathsf{MGM}}}.
  \end{align*}
\end{Example}

The definitions we have given so far only apply to the description logics \EL and \ELbot,
which are of course not the only ones.  Another notable description logic is $\ALC$, a
description logic that provides for conjunction, disjunction, negation, $\top$, $\bot$ as
well as existential and value restrictions.  The definition of the syntax of \ALC is
analogous to the one for \EL.  The semantics of \ALC is again based on interpretations,
and the corresponding extension of the interpretation function to all \ALC concept
description is given in Table~\ref{tab:ALC-syntax}.

\begin{table}[tp]
  \centering
  \renewcommand{\arraystretch}{1.2}
  \begin{tabular}[c]{l l l}
    \toprule
    Constructor Name         & Syntax           & Semantics \\
    \midrule
    top concept              & $\top$           & $\Delta^{\mathcal{I}}$ \\
    bottom concept           & $\bot$           & $\emptyset$ \\
    conjunction              & $C_1 \sqcap C_2$ & $C_1^{\mathcal{I}} \cap C_2^{\mathcal{I}}$ \\
    disjunction              & $C_1 \sqcup C_2$ & $C_1^{\mathcal{I}} \cup C_2^{\mathcal{I}}$ \\
    existential restriction  & $\exists r. C$   & $\set{ x \in \Delta^{\mathcal{I}} \mid
      \exists y \in \Delta^{\mathcal{I}} \st (x, y) \in r^{\mathcal{I}} \land y \in
      C^{\mathcal{I}} }$ \\
    value restriction        & $\forall r. C$   & $\set{ x \in \Delta^{\mathcal{I}} \mid
      \forall y \in \Delta^{\mathcal{I}} \holds (x, y) \in r^{\mathcal{I}} \implies y \in
      C^{\mathcal{I}} }$ \\
    negation                 & $\neg C$         & $\Delta^{\mathcal{I}} \setminus
    C^{\mathcal{I}}$ \\
    \bottomrule
  \end{tabular}
  \caption{Syntax Constructors of \ALC}
  \label{tab:ALC-syntax}
\end{table}

\ALC usually serves as a touchstone for the expressivity of a description logic: a
description logic is usually called \emph{inexpressive} if it does not provide (directly
or indirectly) all constructors of \ALC.  A description logic which provides all
constructors of \ALC is usually called \emph{expressive}.

\section{Knowledge Bases and Reasoning}
\label{sec:knowledge-bases}

Having defined a description logic one can use it to state \emph{axioms}.  These axioms
can be of different nature: they can either state facts about individuals, or they can
state facts about concept descriptions in general.  In the former case, axioms are called
\emph{assertional axioms}, in the latter case they are called \emph{terminological
  axioms}.  A \emph{knowledge base} (or \emph{ontology}) $\mathcal{K} = (\mathcal{A},
\mathcal{T})$ then consists of those axioms, where the assertional axioms are collected
into an \emph{ABox} $\mathcal{A}$, and where the terminological axioms are collected into
a \emph{TBox} $\mathcal{T}$.

If then one has given such a knowledge base $\mathcal{K}$ one can conduct \emph{reasoning}
with it.  Essentially, reasoning is the extraction of knowledge which is \emph{entailed}
by the knowledge base $\mathcal{K}$.  Of course, for this to make sense we need to define
a semantics for the knowledge base $\mathcal{K}$.  Standard reasoning tasks are then
n\emph{subsumption}, \emph{instance checking}, \emph{consistency checking} and
\emph{satisfiability}.

Let us start by formally introducing the notions of \emph{assertional axioms} and
\emph{ABoxes}.

\begin{Definition}[Assertional Axiom, ABox]
  \label{def:assertional-axiom-and-ABox}
  Let $N_C$, $N_R$ and $N_I$ be three pairwise disjoint sets.  An \emph{assertional axiom}
  (over $N_C$, $N_R$ and $N_I$) is an expression of the form
  \begin{equation*}
    C(a) \quad\text{or}\quad r(a, b)
  \end{equation*}
  where $a, b \in N_I, r \in N_R$ and $C \in \ELbot(N_C, N_R)$ is an \ELbot concept
  description.  An assertional axiom $C(a)$ and $r(a,b)$ \emph{holds} in an interpretation
  $\mathcal{I}$, written as $\mathcal{I} \models C(a)$ and $\mathcal{I} \models r(a, b)$,
  if and only if
  \begin{align*}
    a^{\mathcal{I}} \in C^{\mathcal{I}} \quad\text{and}\quad (a^{\mathcal{I}},
    b^{\mathcal{I}}) \in r^{\mathcal{I}},
  \end{align*}
  respectively.  An \emph{ABox} $\mathcal{A}$ is then just a set of assertional axioms
  (over $N_C$, $N_R$ and $N_I$).  An interpretation $\mathcal{I}$ is a \emph{model} of an
  ABox $\mathcal{A}$, written $\mathcal{I} \models \mathcal{A}$ if and only if every
  assertional axiom in $\mathcal{A}$ holds in $\mathcal{I}$.
\end{Definition}

\begin{Example}
  \label{expl:tom-and-jerry-ABox}
  Let us consider Example~\ref{expl:tom-and-jerry-1} again and let us consider some
  assertional axioms over $N_C$, $N_R$ and $N_I$.  For example we can state that
  \textsf{tom} is a \textsf{Cat}, \textsf{jerry} is a \textsf{Mouse}, that \textsf{tom}
  \textsf{hunts} \textsf{jerry} and that \textsf{jerry} \textsf{hunts} \textsf{tom}.  This
  can be achieved (in that order) with the following ABox:
  \begin{equation*}
    \mathcal{A}_{\mathsf{MGM}} = \set{ \mathsf{Cat}(\mathsf{tom}), \mathsf{Mouse}(\mathsf{jerry}),
      \mathsf{hunts}(\mathsf{tom}, \mathsf{jerry}),
      \mathsf{hunts}(\mathsf{jerry}, \mathsf{tom}) }.
  \end{equation*}
  The interpretation $\mathcal{I}_{\mathsf{MGM}}$ from
  Example~\ref{expl:tom-hunts-jerry-hunts-tom-interpretation} is a model of
  $\mathcal{A}_{\mathsf{MGM}}$, \ie $\mathcal{I}_{\mathsf{MGM}} \models
  \mathcal{A}_{\mathsf{MGM}}$.
\end{Example}

Complementary to assertional axioms are \emph{terminological axioms}, which state
connections between concept descriptions.  As already stated, they are collected into
\emph{TBoxes}.  However, in contrast to ABoxes, they are a lot of different flavors of
TBoxes.  We shall restrict our attention to two of these, namely \emph{cyclic} TBoxes and
\emph{general} TBoxes.  For the latter, we shall also introduce the notion of
\emph{general concept inclusions}, which is crucial for our further considerations.

\begin{Definition}[Primitive Concept Definitions, Cyclic TBoxes]
  \label{def:primitive-concept-definitions-cyclic-TBoxes}
  Let $N_C$ and $N_R$ be two disjoint sets, and let $N_D$ be another set disjoint to both
  $N_C$ and $N_R$.  A \emph{primitive concept definition} (over $N_C, N_R$ and $N_D$) is
  an expression of the form
  \begin{equation*}
    B \equiv D
  \end{equation*}
  where $B \in N_D$ and $D \in \ELbot(N_C \cup N_D, N_R)$.  The element $B$ is called the
  \emph{left-hand side} and the concept description $D$ is called the \emph{right-hand
    side} of the concept definition, respectively.

  A \emph{cyclic TBox} $\mathcal{T}$ (over $N_C, N_R$ and $N_D$) is a set consisting of
  primitive concept definitions over $N_C$, $N_R$ and $N_D$ which additionally satisfies
  the condition that for every $B \in N_D$ there exists exactly one primitive concept
  definition in $\mathcal{T}$ with $B$ on the left-hand side.  The set $N_D$ is called the
  set of \emph{defined concept names} of $\mathcal{T}$, and is also denoted by
  $N_D(\mathcal{T})$.

  A primitive concept definition $B \equiv D$ is said to \emph{hold} in an interpretation
  $\mathcal{I}$ over $N_C \cup N_D$ and $N_R$, written $\mathcal{I} \models (B \equiv D)$,
  if and only if
  \begin{equation*}
    B^{\mathcal{I}} = D^{\mathcal{I}}
  \end{equation*}
  is true.  The interpretation $\mathcal{I}$ is a \emph{model} for a cyclic TBox
  $\mathcal{T}$, written $\mathcal{I} \models \mathcal{T}$, if and only if every primitive
  concept definition in $\mathcal{T}$ holds in $\mathcal{I}$.
\end{Definition}

\begin{Example}
  \label{expl:tom-and-jerry-cyclic-TBox}
  Let us consider again $N_C = \set{ \mathsf{Cat}, \mathsf{Mouse} }, N_R = \set{
    \mathsf{hunts} }$ and in addition let $N_D = \set{ \mathsf{MouseHuntingCat} }$.  Then
  a primitive concept definition over $N_C, N_R$ and $N_D$ would be
  \begin{equation*}
    \mathsf{MouseHuntingCat} \equiv \mathsf{Cat} \sqcap \exists \mathsf{hunts}.\mathsf{Mouse}
  \end{equation*}
  and the singleton set containing this primitive concept definition would be a cyclic
  TBox over $N_C, N_R$ and $N_D$.
\end{Example}

\todo[inline]{Write: general TBoxes}%
\todo[inline]{Write: knowledge bases}%
\todo[inline]{Write: standard reasoning tasks}%

\section{The Description Logic \ELgfpbot}
\label{sec:descr-logics-elbot}

\todo[inline]{Write: talk about model-base most-specific concept descriptions}%
\todo[inline]{Write: introduce \ELgfpbot}%
\todo[inline]{Write: talk about descriptive and greatest-fixpoint semantics}%
\todo[inline]{Write: introduce unravelling of \ELgfpbot concept descriptions (not here,
  but later)}

%%% Local Variables: 
%%% mode: latex
%%% TeX-master: "../main"
%%% End: 
