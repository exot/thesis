\chapter{Conclusions and Outlook}
\label{cha:conclusions}

This work was concerned with axiomatizing all GCIs expressible in the description logic
$\ELbot$ that enjoy a certain minimal confidence in a given finite interpretation.  The
motivation for this research stemmed from the observation that the results from
Distel~\cite{Diss-Felix} on finding finite bases of finite interpretations are very
sensitive towards sporadic errors in the given interpretation, and thus may not perform
well on real-world data, which can be assumed to always contain errors.

The way errors affect the results obtained by Distel's results has been assessed in detail
in \Cref{sec:computing-bases-from}, where we have argued how \emph{linked data} can be
seen as a variant of finite interpretations, and where we have computed finite bases of a
finite interpretation $\Idbpedia$ that has been obtained from the DBpedia data set by
restriction to the \textsf{child} role.  Computing finite bases of the interpretation
$\Idbpedia$ turned out to be not very insightful, mostly because the concept names
contained in $\Idbpedia$ consisted mostly of job descriptions, and it cannot be expected
that the job of a parent affects the jobs of their children, or vice versa, in a way
expressible as a GCI.  Furthermore, the interpretation $\Idbpedia$ turned out to be
erroneous to the extent that it contained things that are usually not associated with the
\textsf{child} role, like places, books or bands.

Despite the deficits of $\Idbpedia$ we have argued that one still can expect certain GCIs
to be valid in $\Idbpedia$, most notably
\begin{equation*}
  \exists \mathsf{child}. \top \sqsubseteq \mathsf{Person}.
\end{equation*}
However, this GCI turned out to be not valid in $\Idbpedia$.  On the other hand, we have
seen that there were only 4 (erroneous) counterexamples in $\Idbpedia$, among 2551
elements in $\Idbpedia$ to which it is applicable.  This observation motivated us to
extend the results on axiomatizing valid GCIs into the direction of axiomatizing GCIs with
\emph{high confidence} in the given data.

This extension has been discussed in \Cref{sec:confident-gcis}, where we had adapted
results obtain by Luxenburger~\cite{diss:Luxenburger} on \emph{partial implications} in
finite contexts to the setting of GCIs in finite interpretations $\mathcal{I}$.  From this
adaption we have obtained finite confident bases of finite interpretations
(\Cref{thm:conf-base}, \Cref{thm:luxenbuger-base-for-gcis}), and a way to compute finite
confident bases of GCIs from finite confident bases of implications
(~\Cref{thm:confident-bases-of-GCIs-from-confident-bases-of-implications}).  We have also
seen how we can make use of the canonical base of the induced context of $\mathcal{I}$ to
complete bases of $\Th_{c}(\mathcal{I}) \setminus \Th(\mathcal{I})$ to confident bases of
$\Th_{c}(\mathcal{I})$.  Finally, we have generalized in \Cref{sec:unrav-elgfpb-bases}
results of Distel to obtain finite confident $\ELbot$ bases from finite confident
$\ELgfpbot$ bases, using the technique of \emph{unravelling} $\ELgfpbot$ concept
descriptions.

The results we had obtained about computing finite confident bases of $\mathcal{I}$ were
then in turn applied to our example interpretation $\Idbpedia$, to examine how the
approach of computing finite confident bases performs on this data set.  As expected, we
could recover the GCIs $\exists \mathsf{child}. \top \sqsubseteq \mathsf{Person}$, as well
as another GCI.  In particular, we have observed that for relatively large values of the
confidence threshold $c$, the number of additional GCIs to be examined by the expert is
relatively small.  Thus, one could argue that the approach of considering GCIs with high
confidence does not add too much overhead for practical applications, but promises to
deliver more robust results.

However, the approach of considering GCIs with high confidence is still a heuristic
approach, and the GCIs thus obtained need to be validated by an external source of
information, like a human expert.  Indeed, the same is true for valid GCIs of the input
interpretation $\mathcal{I}$, as it may be that this input data misses some crucial
counterexamples for the domain of interest.  For this purpose, Distel developed
\emph{model exploration} to interactively consult an expert during the computation of a
finite base of $\mathcal{I}$ to provide missing counterexamples if needed.  We have argued
that providing such an exploration algorithm for the computation of finite confident bases
may also be helpful for practical applications.

To this end, we have first discussed in \Cref{cha:expl-conf} an approach to obtain an
exploration algorithm for \emph{implications with high confidence} in some given formal
context $\con K$, called \emph{exploration by confidence}.  For this, we have introduced
in \Cref{sec:an-abstract-view} an abstraction of the classical attribute exploration
algorithm as an algorithm to explore the set $\Th(\con K)$ of implications.  Then in
\Cref{sec:expl-conf-1}, we have argued how this abstraction can be transferred to the set
$\Th_{c}(\con K)$, automatically yielding an algorithm that explores the set $\Th_{c}(\con
K)$.  This algorithm, which indeed computes the canonical base of $\Th_{c}(\con K) \cap
\Th(p)$ with respect to some given expert $p$, has the practical disadvantage that it
requires the computation of closures under $\Th_{c}(\con K)$, which may not be feasible in
certain applications.  To remedy this, we have introduced in \Cref{sec:poss-fast-expl} an
exploration algorithm that avoids this computation, but does not necessarily compute the
canonical base of $\Th_{c}(\con K) \cap \Th(p)$ anymore.

The purpose of the then following \Cref{cha:model-expl-conf} was to extend the algorithm
for exploration by confidence to provide an algorithm for \emph{model exploration by
  confidence}.  Such an algorithm was obtained in \Cref{sec:expl-conf-gcis-1}, and as a
byproduct of the argumentation used, we also obtained more practical algorithms for
computing confident bases of formal contexts (~\Cref{sec:grow-sets-attr-1}) and of finite
interpretations (~\Cref{sec:expl-conf-gcis}).

\todo[inline]{Write: support for GCIs can be unintuitive (WDPM paper)}

\todo[inline]{Write: exploiting connection between FCA and data-mining to obtain fast
  algorithms to compute confident bases of finite interpretations}

\todo[inline]{Write: cardinality constraints on the premises of the implications/GCIs}

\todo[inline]{Write: make model exploration by confidence more practical}

%%% Local Variables: 
%%% mode: latex
%%% TeX-master: "../main"
%%% End: 

%  LocalWords:  gcis thm luxenbuger conf expl attr
