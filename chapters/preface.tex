\vspace*{23ex}

\todo[inline]{Find suitable dedication}
\centerline{\textit{dedication goes here}}

\chapter*{Preface}
\label{cha:preface}

\thispagestyle{empty}

Knowledge is required to solve problems and to act intelligently in almost all situations
-- this platitude is not only true for humans, but also for computers.  Thus, to enable
humans and computers to act intelligently, one has to first solve the problem of
\emph{acquiring knowledge}, also referred to as \emph{learning}.  In the case of
computers, another problem has to be tackled, namely the problem of \emph{representing
  knowledge} in a way that suitable for a machine to work with.

An approach to represent knowledge is to use different flavors of logics, and among these
flavors \emph{description logics} are a very popular choice.  Description logics are a
family of different logic-based formalisms with varying reasoning complexity and
expressiveness, specifically tailored towards efficient decision procedures.  Moreover,
\emph{description logic knowledge bases} allow for a very powerful mechanism to represent
knowledge, which is used in practical applications like various bio-medical ontologies, or
in the semantic web.

The knowledge representable by description logic knowledge bases can be \emph{assertional
  knowledge} or \emph{terminological knowledge}.  Obtaining assertional knowledge, like
the fact that Abraham Lincoln was a president of the United States, or that John McCarthy
was a professor at Stanford University, is comparably easy, and can to a certain extent
even be done automatically.  This is mostly due to the rather local nature of assertional
knowledge, which just concerns one or two individuals.  On the other hand, obtaining
terminological knowledge is much more involved, because this knowledge concepts to each
other: the fact that every human is mortal, or that every human has a head are examples
for terminological knowledge.  Obtaining such knowledge, which may concern a wide and
mostly indefinite range of individuals or things, is much harder, and hoping to obtain it
automatically seems gullible.

Nevertheless, there have been various approaches to obtain terminological knowledge, at
least in a preliminary version, from various sources of data, like natural language texts,
databases or linked data.  One of these approaches has been developed by Franz Baader and
Felix Distel, and allows the extraction of general terminological knowledge in the form of
\emph{general concept inclusions} which are valid in a given interpretation.
Interpretations are structures which are used in description logics to defined the
semantics of different logics, and can be thought of as directed edge- and vertex-labeled
graphs.  Therefore, interpretations can be thought of as a variant of \emph{linked data},
and then the approach by Baader and Distel allows in principle the extraction of
terminological knowledge from this form of data, which is ubiquitous in the realm of the
semantic web.  Thus, one could turn the vast amount of linked data into description logic
knowledge bases and use them to make computers act more intelligently.



\todo[inline]{Write: preface}

%%% Local Variables: 
%%% mode: latex
%%% TeX-master: "../main"
%%% End: 
