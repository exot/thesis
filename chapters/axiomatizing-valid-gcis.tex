
\chapter{Axiomatizing Valid \EL-GCIs of Finite Interpretations}
\label{cha:axiom-valid-el}

\todo[inline]{Plan part on axiomatizing valid \EL-GCIs of Finite Interpretations}

\section{Model-Based Most-Specific Concept Descriptions}
\label{sec:motivation}

\subsection{Motivation and Definition}
\label{sec:motiv-defin}

\todo[inline]{Write: list analogies of FCA and DL, and say that there is something
  missing}%
\todo[inline]{Write: definition of model-based most-specific concept descriptions}%

\subsection{Existence}
\label{sec:existence}

\todo[inline]{Write: non-existence in \ELbot and existence in \ELgfpbot}%
\todo[inline]{Write: mention the Galois-connection here and write down some results}%

\subsection{A Contextual Representation}
\label{sec:cont-repr-all}

\todo[inline]{Write: introduce $M_{\mathcal{I}}$ and induced contexts (can we do that with
  a set $M$ such that every mmsc is expressible in terms of $M$?)}%
\todo[inline]{Write: repeat all kinds of results from MATH-AL-08-2012/4.{1,2} and
  12-06/4.2}%

\section{Bases of Valid GCIs of a Finite Interpretation}
\label{sec:base-all-valid}

\subsection{A General Construction}
\label{sec:first-base-1}

\todo[inline]{Write: show that every base of the induced formal contexts gives rise to a
  base of the underlying interpretation}%

\subsection{A Minimal Base}
\label{sec:minimal-base}

\todo[inline]{Write: formulate minimality result}%

%%% Local Variables: 
%%% mode: latex
%%% TeX-master: "../main"
%%% End: 
