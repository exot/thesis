\chapter{Axiomatizing Valid \EL-GCIs of Finite Interpretations}
\label{cha:axiom-valid-el}

Our considerations about extracting general concept inclusions from erroneous data will be
based on previous results by \textcite{Diss-Felix} on extracting all \emph{valid} general
concept from finite interpretations.  We shall therefore review in this section all
necessary notions from this work that are needed for our work.

The problem of extracting all valid general concept inclusions from a finite
interpretation can be made more precise as follows.  Let $\mathcal{I} =
(\Delta^{\mathcal{I}}, \cdot^{\mathcal{I}})$ be a finite interpretation over $N_C$ and
$N_R$, \ie $\Delta^{\mathcal{I}}$ is a finite set.  The task is then to find the set of
all general concept inclusions $C \subseteq D$ with $C, D \in \ELbot(N_C, N_R)$ which are
valid in $\mathcal{I}$.

Of course, this set is infinite in general.  This is because if $C \sqsubseteq D$ holds in
$\mathcal{I}$, and $r \in N_R$, then $\exists r. C \sqsubseteq \exists r. D$ holds in
$\mathcal{I}$ as well.  Such an infinite set is hardly usable to represent knowledge
suitable for machine consumption.  Therefore, the considerations in~\cite{Diss-Felix}
concentrate on finding \emph{finite bases} of $\mathcal{I}$, \ie set of valid general
concept inclusions of $\mathcal{I}$ which are also \emph{complete}.  One of the main
results of~\cite{Diss-Felix} is that finite bases for finite interpretations $\mathcal{I}$
always exist.

A notion crucial for this result is the one of \emph{model-based most-specific concept
  descriptions}, which we have already seen briefly in \Cref{sec:descr-logics-elbot}.
This notion is motivated by an attempt to mimic the derivation operators within the
description logic \ELbot, to allow methods from formal concept analysis to be carried over
to this logic.  We shall discuss model-based most-specific concept descriptions in detail
in \Cref{sec:motivation}.  Thereafter, we shall talk about \emph{induced contexts}, which
will allow us to obtain various connections between derivation operators on the one hand,
and concept extensions as well as model-based most-specific concept description on the
other.  These connections are most often of technical nature, but are crucial for our
further considerations.




\section{Model-Based Most-Specific Concept Descriptions}
\label{sec:motivation}

\subsection{Motivation and Definition}
\label{sec:motiv-defin}

\todo[inline]{Write: list analogies of FCA and DL, and say that there is something
  missing}%
\todo[inline]{Write: definition of model-based most-specific concept descriptions}%
\todo[inline]{Write: non-existence in \ELbot and existence in \ELgfpbot}%
\todo[inline]{Write: mention the Galois-connection here and write down some results}%

\subsection{Induced Contexts}
\label{sec:induced-contexts}

\todo[inline]{Write: induced contexts, relationships between derivation operators and
  model-based most-specific concept descriptions}%

\subsection{A Contextual Representation}
\label{sec:cont-repr-all}

\todo[inline]{Write: MATH-AL-08-2012/4.2 and 12-06/4.2}%

\section{Bases of Valid GCIs of a Finite Interpretation}
\label{sec:base-all-valid}

\subsection{Bases of General Concept Inclusions}
\label{sec:bases-gener-conc}

General concept inclusions are very similar to implications, and we shall see in
\Cref{cha:axiom-valid-el} that a precise connection between them can be established.  In
particular, we can talk about \emph{entailment} between general concept inclusions and
\emph{completeness} of sets of general concept inclusions.  Notice the similarity between
the following definition and \Cref{def:sound-complete-base}.

\begin{Definition}
  \label{def:entailment-of-gcis}
  Let $N_C$ and $N_R$ be two disjoint sets, and let $\mathcal{L} \cup \set{ C \sqsubseteq
    D }$ be a set of general concept inclusions over $N_C$ and $N_R$.  We shall say that
  $\mathcal{L}$ \emph{entails} $C \sqsubseteq D$, written $\mathcal{L} \models (C
  \sqsubseteq D)$ if and only if for all interpretations over $N_C$ and $N_R$ it is true
  that if $\mathcal{I} \models \mathcal{L}$, then $\mathcal{I} \models \set{ C \sqsubseteq
    D }$ as well.

  Let $\mathcal{K}$ be another set of general concept inclusions over $N_C$ and $N_R$.
  Then $\mathcal{K}$ is said to be \emph{sound} for $\mathcal{L}$ if and only if all
  general concept inclusions in $\mathcal{K}$ are entailed by $\mathcal{L}$.
  $\mathcal{K}$ is said to be \emph{complete} for $\mathcal{L}$ if and only if all general
  concept inclusions in $\mathcal{L}$ are \emph{entailed} by $\mathcal{K}$.  $\mathcal{K}$
  is said to be a base for $\mathcal{L}$ if and only if $\mathcal{K}$ is sound and
  complete for $\mathcal{L}$.
\end{Definition}

Let $\mathcal{I}$ be a finite interpretation over $N_C$ and $N_R$, and let us denote with
$\Th(\mathcal{I})$ the set of all general concept inclusions which are valid in
$\mathcal{I}$.  Note that this definition requires that the logic which is within the
general concept inclusions is known from the context.  If this is not the case, we shall
add a subscript to clarify this.  For example, the set of all valid general concept
inclusions with concept descriptions in $\ELbot(N_C, N_R)$ shall be denoted by
$\Th_{\ELbot(N_C, N_R)}(\mathcal{I})$.

Let $\mathcal{K}$ be a set of general concept inclusions over $N_C$ and $N_R$.  If
$\mathcal{K}$ is a base for $\Th(\mathcal{I})$ we shall simply say that $\mathcal{K}$ is
base of $\mathcal{I}$.  Notice that in this case, all general concept inclusions have to
hold in $\mathcal{I}$: the set $\Th(\mathcal{I})$ is \emph{closed under entailment} in the
sense that every general concept inclusion over $N_C$ and $N_R$ which is entailed by
$\Th(\mathcal{I})$ is already an element of it.  Therefore, if $\mathcal{K}$ is sound for
$\Th(\mathcal{I})$, it must be contained in this set and thus $\mathcal{K}$ is a set of
general concept inclusions which are valid in $\mathcal{I}$.  Moreover, since
$\mathcal{K}$ is complete for $\Th(\mathcal{I})$, every general concept inclusion over
$N_C$ and $N_R$ that holds in $\mathcal{I}$ is entailed by $\mathcal{K}$.

\subsection{A General Construction}
\label{sec:first-base-1}

\todo[inline]{Write: show that every base of the induced formal contexts gives rise to a
  base of the underlying interpretation}%

\subsection{A Minimal Base}
\label{sec:minimal-base}

\todo[inline]{Write: formulate minimality result}%

%%% Local Variables: 
%%% mode: latex
%%% TeX-master: "../main"
%%% End: 
