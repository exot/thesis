\chapter{Axiomatizing Valid General Concept Inclusions of Finite
  Interpretations}
\label{cha:axiom-valid-el}

Our considerations about extracting general concept inclusions from erroneous data will be
based on previous results obtained by \textcite{Diss-Felix} on extracting all \emph{valid}
general concept from finite interpretations.  We shall therefore review in this section
all necessary notions from this work that are needed for our own.

The problem of extracting all valid general concept inclusions from a finite
interpretation can be made more precise as follows.  Let $\mathcal{I} =
(\Delta^{\mathcal{I}}, \cdot^{\mathcal{I}})$ be a finite interpretation over $N_C$ and
$N_R$, \ie $\Delta^{\mathcal{I}}$ is a finite set.  The task is then to find the set of
all general concept inclusions $C \sqsubseteq D$ with $C, D \in \ELbot(N_C, N_R)$ which
are valid in $\mathcal{I}$.

Of course, this set is infinite of valid general concept inclusions in general.  This is
because if $C \sqsubseteq D$ holds in $\mathcal{I}$, and $r \in N_R$, then $\exists r. C
\sqsubseteq \exists r. D$ holds in $\mathcal{I}$ as well.  Such an infinite set is hardly
usable to represent knowledge suitable for machine consumption.  Therefore, the
considerations in~\cite{Diss-Felix} concentrate on finding \emph{finite bases} of
$\mathcal{I}$, \ie set of valid general concept inclusions of $\mathcal{I}$ which are also
\emph{complete}.  We shall introduce these notions briefly in \Cref{sec:bases-gener-conc}.

One of the main results of~\cite{Diss-Felix} is then that finite bases for finite
interpretations $\mathcal{I}$ always exist, and we shall discuss them in
\Cref{sec:base-all-valid}.  These results have been obtained by exploiting a close
connection between description logics and formal concept analysis.  It is therefore
crucial that we introduce this connection first, and we shall do so
in~\Cref{sec:motivation}.  In particular, we shall talk about \emph{induced contexts} and
\emph{model-based most-specific concept descriptions}.

In what follows, we shall fix the two disjoint sets $N_C$ and $N_R$ and shall not mention
them explicitly anymore.  This means, for example, that whenever we are talking about
interpretations or concept descriptions, we are actually talking about interpretations
over $N_C$ and $N_R$, and about concept descriptions over $N_C$ and $N_R$.

\section{Bases of General Concept Inclusions}
\label{sec:bases-gener-conc}

General concept inclusions have a \emph{model-based semantics}, their semantics is defined
in terms of being valid in some interpretation.  We can therefore introduce a notion of
\emph{entailment} and \emph{completeness}, much like as we did for implications.

\begin{Definition}
  \label{def:entailment-of-gcis}
  Let $\mathcal{L} \cup \set{ C \sqsubseteq D }$ be a set of general concept inclusions
  over $N_C$ and $N_R$.  We shall say that $\mathcal{L}$ \emph{entails} $C \sqsubseteq D$,
  written $\mathcal{L} \models (C \sqsubseteq D)$ if and only if for all interpretations
  over $N_C$ and $N_R$ it is true that if $\mathcal{I} \models \mathcal{L}$, then
  $\mathcal{I} \models \set{ C \sqsubseteq D }$ as well.

  Let $\mathcal{K}$ be another set of general concept inclusions over $N_C$ and $N_R$.
  Then $\mathcal{K}$ is said to be \emph{sound} for $\mathcal{L}$ if and only if all
  general concept inclusions in $\mathcal{K}$ are entailed by $\mathcal{L}$.
  $\mathcal{K}$ is said to be \emph{complete} for $\mathcal{L}$ if and only if all general
  concept inclusions in $\mathcal{L}$ are \emph{entailed} by $\mathcal{K}$.  $\mathcal{K}$
  is said to be a base for $\mathcal{L}$ if and only if $\mathcal{K}$ is sound and
  complete for $\mathcal{L}$.
\end{Definition}

Notice the similarity of this definition to \Cref{def:sound-complete-base}.

Let $\mathcal{I}$ be a finite interpretation over $N_C$ and $N_R$, and let us denote with
$\Th_{\ELbot(N_C, N_R)}(\mathcal{I})$ the set of all \ELbot general concept inclusions
over $N_C$ and $N_R$ which are valid in $\mathcal{I}$, and likewise denote with
$\Th_{\ELgfpbot(N_C, N_R)}(\mathcal{I})$ the set of all \ELgfpbot general concept
inclusions over $N_C$ and $N_R$ which are valid in $\mathcal{I}$.  If the logic used is
clear from the context, then we shall drop the subscript.

Let $\mathcal{K}$ be a set of general concept inclusions over $N_C$ and $N_R$.  If
$\mathcal{K}$ is a set of \ELbot general concept inclusions and is a base for
$\Th_{\ELbot(N_C, N_R)}(\mathcal{I})$ we shall simply say that $\mathcal{K}$ is an
\emph{\ELbot base} of $\mathcal{I}$.  If we replace \ELbot by \ELgfpbot, then we call
$\mathcal{K}$ an \emph{\ELgfpbot base} instead.  If the logic is clear from the context or
not relevant for the current consideration, we shall leave it out.

Notice that in the case that $\mathcal{K}$ is an \ELbot base of $\mathcal{I}$, all general
concept inclusions in $\mathcal{K}$ have to hold in $\mathcal{I}$: the set
$\Th(\mathcal{I})$ is \emph{closed under entailment} in the sense that every \ELbot
general concept inclusion over $N_C$ and $N_R$ which is entailed by $\Th(\mathcal{I})$ is
already contained in this set.  Therefore, if $\mathcal{K}$ is sound for
$\Th(\mathcal{I})$, it must be contained in this set and thus $\mathcal{K}$ is a set of
general concept inclusions which are valid in $\mathcal{I}$.  Moreover, since
$\mathcal{K}$ is complete for $\Th(\mathcal{I})$, every general concept inclusion over
$N_C$ and $N_R$ that holds in $\mathcal{I}$ is entailed by $\mathcal{K}$.

\section{Linking Formal Concept Analysis and Description Logics}
\label{sec:motivation}

Description logics and formal concept analysis are connected by a number of similar
notions.  As an example, let us consider a formal context $\con K = (G, M, I)$ and a set
$A \subseteq M$.  Then the set $A'$ is the set of all objects of $\con K$ which have all
the attributes in $A$.  We can view this fact from another perspective: if $A = \set{ m_1,
  \dots, m_n }$, then we can think of the attributes $m_1, \dots, m_n$ as
\emph{propositions}, and the fact that $(g, m) \in I$ as saying that $g$ \emph{satisfies}
the proposition $m$.  Then $g \in A'$ means that $g$ \emph{satisfies} the conjunction of
all propositions in $A$.

Let us reformulate this using description logics, and let us define $N_C := M$ and $N_R =
\emptyset$.  Then we can think of $\con K$ as an interpretation $\mathcal{I}_{\con K} =
(G, \cdot^{\mathcal{I}_{\con K}})$ where
\begin{equation}
  \label{eq:17}
  m^{\mathcal{I}_{\con K}} := \set{ g \in G \mid (g, m) \in I } = \set{ m }'.
\end{equation}
Then indeed we have that $A' = (m_1 \sqcap \dots \sqcap m_n)^{\mathcal{I}_{\con K}}$ for
all finite $A = \set{ m_1, \dots, m_n } \subseteq M$.  Indeed, if we would consider a
description logic that only allows for conjunction $\sqcap$, then we can view finite
formal contexts, derivation of sets of attributes and even implications as special cases
of finite interpretations, extensions of concept descriptions and general concept
inclusions.  Thus the derivation operator $(\cdot)' \colon \subsets{M} \to \subsets{G}$
corresponds naturally to computing the extension of concept descriptions in
interpretations.  However, the other derivation operator $(\cdot)' \colon \subsets{G} \to
\subsets{M}$ does not have such a correspondence in description logics.  This gap shall be
filled by considering \emph{model-based most-specific concept descriptions}, which we
introduce in \Cref{sec:defin-and-basic}.

The connection between description logics and formal concept analysis expressed in
\eqref{eq:17} only works in one direction: it allows to represent basic notions of formal
concept analysis in terms of description logics, but not vice versa.  Even if we restrict
our attention to the rather light-weight description logic \ELbot, it is not clear how to
represent an interpretation by means of notions from formal concept analysis.

To approach this issue, we shall introduce \emph{induced contexts} in
\Cref{sec:induced-contexts}.  Such contexts allow to express tight connections between the
notions of formal concept analysis and description logics, and, since induced contexts are
just formal contexts, still allows the application of standard methods from formal concept
analysis, such as the extraction of bases.  This fact will be exploited when we discuss
the computation of finite bases of in \Cref{sec:base-all-valid}.

\subsection{Model-Based Most-Specific Concept Descriptions}
\label{sec:defin-and-basic}

Let $\con K = (G, M, I)$ and Let us try to motivate how to find a natural correspondence
of the derivation operator $(\cdot)' \colon \subsets{G} \to \subsets{M}$ within
description logics.  Let $B \subseteq G$ be a set of objects of $\con K$.  Then the set $A
:= B'$ can be thought of as the \emph{most-specific} set of attributes that
\emph{describe} $B$, \ie
\begin{enumerate}[i. ]
\item $B \subseteq A'$, \ie $A$ \emph{describes} $B$, and
\item for all sets $C \subseteq M$ that satisfy $B \subseteq C'$ (that describe $B$) it is
  true that $C \subseteq A$, ($A$ contains \emph{more} attributes than $C$, \ie is
  \emph{more specific}).
\end{enumerate}
The last point is true because if $B \subseteq C'$, then by
\Cref{lem:derivation-is-galois-connection} it is true that $C \subseteq B' = A$.  Notice
that the description of $A$ as a most-specific description of $B$ is also a
characterization, \ie if $A$ is the most-specific description of $B$ in the above sense,
then $A = B'$.

To mimic this \emph{most-specific description} in description logics, Distel introduces
the notion of \emph{most-specific concept descriptions}.

\begin{Definition}[Most-Specific Concept Description]
  \label{def:most-specific-concept-description}
  Let $\mathcal{I} = (\Delta^{\mathcal{I}}, \cdot^{\mathcal{I}})$ be an interpretation, and
  let $X \subseteq \Delta^{\mathcal{I}}$.  A \emph{most-specific concept description} of
  $X$ in $\mathcal{I}$ is a concept description $C$ such that
  \begin{enumerate}[i. ]
  \item $X \subseteq C^{\mathcal{I}}$, and
  \item for each concept description $D$ satisfying $X \subseteq D^{\mathcal{I}}$ it is
    true that $C \sqsubseteq D$, \ie $C$ is subsumed by $D$.
  \end{enumerate}
\end{Definition}

If a model-based most-specific concept description $C$ for $X$ in $\mathcal{I}$ exists, it
is unique up to equivalence: if $D$ is another such model-based most-specific concept
description, than $C \sqsubseteq D$ and $D \sqsubseteq C$, by the last condition of the
definition.  Therefore, $C \equiv D$.  Because of this, we can talk about \emph{the}
model-based most-specific of $X$ in $\mathcal{I}$, and shall denoted it with
$X^{\mathcal{I}}$, to stress the similarity to the derivation operator from formal concept
analysis.  We shall also write $X^{\mathcal{I}\mathcal{I}}$ instead of
$(X^{\mathcal{I}})^{\mathcal{I}}$ and $C^{\mathcal{I}\mathcal{I}}$ instead of
$(C^{\mathcal{I}})^{\mathcal{I}}$ for syntactic convenience.

The existence of model-based most-specific concept descriptions, however, is not clear per
se, and the choice of the description logic in which we seek for model-based most-specific
concept descriptions is crucial here: if we only consider \ELbot concept descriptions,
then model-based most-specific concept descriptions do not necessarily exist, as is shown
in \Cref{expl:mmscs-may-not-exist-in-ELbot}.  However, if we allow all concept
descriptions in \Cref{def:most-specific-concept-description} to be \ELgfp or \ELgfpbot
concept descriptions, then the existence of model-based most-specific concept descriptions
can be guaranteed.

\begin{Theorem}[Theorem 4.7 of~\cite{Diss-Felix}]
  \label{thm:existence-of-mmscs-in-ELgfpbot}
  Model-based most-specific concept descriptions exist in \ELgfp and \ELgfpbot for all
  interpretations $\mathcal{I} = (\Delta^{\mathcal{I}}, \cdot^{\mathcal{I}})$ and sets $X
  \subseteq \Delta^{\mathcal{I}}$, and they can be computed effectively.
\end{Theorem}

The computation of model-based most-specific concept descriptions can be achieved using
\emph{\EL description graphs}, least common subsumers and
\emph{simulations}~\cite{DBLP:conf/ijcai/Baader03a,Diss-Felix}.  See~\cite[Section
4.1.2]{Diss-Felix} for details on this.

We have motivated model-based most-specific concept descriptions by most-specific
descriptions in formal contexts, and for this we have made use of the fact the derivation
operators form a Galois connection.  It is therefore only natural to expect that
model-based most-specific concept descriptions are also part of a Galois connection.
However, we have to notice that we cannot expect to obtain a Galois connection in the
sense of \Cref{sec:galois-connections}, simply because the relation $\sqsubseteq$ is not
antisymmetric, and thus not an order relation: it may be the case that $C \sqsubseteq D$
and $C \sqsubseteq D$, but $C \neq D$.  We can remedy this fact by considering concept
descriptions only \emph{up to equivalence}: instead of considering only a single concept
description $C$, we always consider the set $[C]$ of all concept descriptions which are
equivalent to $C$.  Then $[C] \sqsubseteq [D]$ is well-defined for all concept
descriptions $C$ and $D$, and $\sqsubseteq$ indeed yields an order relation this way.
This is only a technical detail, however, and we shall not make it explicit in our
following considerations.

\begin{Lemma}[Lemma 4.1 of~\cite{Diss-Felix}]
  \label{lem:mmsc-and-extension-are-galois-connection}
  Let $\mathcal{I} = (\Delta^{\mathcal{I}}, \cdot^{\mathcal{I}})$ be an interpretation
  over $N_C$ and $N_R$, $X \subseteq \Delta^{\mathcal{I}}$ and $C$ an \ELgfpbot concept
  description over $N_C$ and $N_R$.  Then
  \begin{equation}
    \label{eq:19}
    X \subseteq C^{\mathcal{I}} \iff X^{\mathcal{I}} \sqsubseteq C.
  \end{equation}
  In particular, for $X, Y \subseteq \Delta^{\mathcal{I}}$ and for \ELgfpbot concept
  descriptions $C, D$ over $N_C$ and $N_R$, it is true that
  \begin{enumerate}[i. ]
  \item $X \subseteq Y \implies X^{\mathcal{I}} \sqsubseteq Y^{\mathcal{I}}$,
  \item $C \sqsubseteq D \implies C^{\mathcal{I}} \subseteq D^{\mathcal{I}}$,
  \item $X \subseteq X^{\mathcal{I}\mathcal{I}}$,
  \item $C^{\mathcal{I}\mathcal{I}} \sqsubseteq C$,
  \item $X^{\mathcal{I}} \equiv X^{\mathcal{I}\mathcal{I}\mathcal{I}}$,
  \item $C^{\mathcal{I}} = C^{\mathcal{I}\mathcal{I}\mathcal{I}}$.
  \end{enumerate}
\end{Lemma}
\begin{Proof}
  We only show \eqref{eq:19}, the other claims follow from
  \Cref{lem:properties-of-galois-connections} and the above-made considerations.  If $X
  \subseteq C^{\mathcal{I}}$, then $X^{\mathcal{I}} \subseteq C$ because $X^{\mathcal{I}}$
  is by definition the most-specific concept description that contains $X$ in its
  extension.  Conversely, if $X^{\mathcal{I}} \sqsubseteq C$, then by definition
  $X^{\mathcal{I}\mathcal{I}} \subseteq C^{\mathcal{I}}$.  But since $X^{\mathcal{I}}$ is
  the model-based most-specific concept description of $X$ in $\mathcal{I}$, it contains
  $X$ in its extension, \ie $X \subseteq X^{\mathcal{I}\mathcal{I}}$.  Therefore, $X
  \subseteq C^{\mathcal{I}}$.
\end{Proof}

Another useful property is the following, rather technical proposition.

\begin{Proposition}[Lemma 4.2 of~\cite{Diss-Felix}]
  \label{prop:double-II-under-I}
  Let $\mathcal{I}$ be an interpretation over $N_C$ and $N_R$, and let $C, D$ be \ELgfpbot
  concept descriptions over $N_C$ and $N_R$ and let $r \in N_R$.  Then
  \begin{enumerate}[i. ]
  \item $(C \sqcap D)^{\mathcal{I}} = (C^{\mathcal{I}\mathcal{I}} \sqcap
    D)^{\mathcal{I}}$, and
  \item $(\exists r. C)^{\mathcal{I}} = (\exists r. C^{\mathcal{I}\mathcal{I}})^{\mathcal{I}}$.
  \end{enumerate}
\end{Proposition}
\begin{Proof}
  For the first claim we use \Cref{lem:mmsc-and-extension-are-galois-connection} and obtain
  \begin{equation*}
    (C \sqcap D)^{\mathcal{I}} = C^{\mathcal{I}} \cap D^{\mathcal{I}} =
    C^{\mathcal{I}\mathcal{I}\mathcal{I}} \cap D^{\mathcal{I}} =
    (C^{\mathcal{I}\mathcal{I}} \sqcap D)^{\mathcal{I}}.
  \end{equation*}
  For the second one we observe that
  \begin{align*}
    (\exists r. C^{\mathcal{I}\mathcal{I}})^{\mathcal{I}}
    &= \set{ x \in \Delta^{\mathcal{I}} \mid \exists y \in \Delta^{\mathcal{I}} \st (x, y)
      \in r^{\mathcal{I}} \wedge y \in C^{\mathcal{I}\mathcal{I}\mathcal{I}} }\\
    &= \set{ x \in \Delta^{\mathcal{I}} \mid \exists y \in \Delta^{\mathcal{I}} \st (x, y)
      \in r^{\mathcal{I}} \wedge y \in C^{\mathcal{I}} } \\
    &= (\exists r. C)^{\mathcal{I}},
  \end{align*}
  again because of $C^{\mathcal{I}} = C^{\mathcal{I}\mathcal{I}\mathcal{I}}$ from
  \Cref{lem:mmsc-and-extension-are-galois-connection}.
\end{Proof}


\subsection{Induced Contexts}
\label{sec:induced-contexts}

We have already seen how formal contexts can be represented as interpretations.  In this
section we shall introduce the approach of \textcite{Diss-Felix} of \emph{induced
  contexts}, which somehow provides the inverse direction, \ie to represent
interpretations as formal contexts.  The notion of induced contexts was also implicitly
used by works of \textcite{books/math/Prediger00} in her study on \emph{terminological
  attribute logic}.

\begin{Definition}[Induced Context]
  \label{def:induced-context}
  Let $\mathcal{I} = (\Delta^{\mathcal{I}}, \cdot^{\mathcal{I}})$ be a finite
  interpretation over $N_C$ and $N_R$, and let $M$ be a set of concept descriptions over
  $N_C$ and $N_R$.  The \emph{induced context} of $\mathcal{I}$ and $M$ is the formal
  context $\con K_{\mathcal{I}, M} = (\Delta^{\mathcal{I}}, M, \nabla)$, where for $x \in
  \Delta^{\mathcal{I}}$ and $C \in M$
  \begin{equation*}
    (x, C) \in \nabla \diff x \in C^{\mathcal{I}}.
  \end{equation*}  
\end{Definition}

Induced formal contexts do not necessarily represent all of the interpretation
$\mathcal{I}$; indeed, what is represented of $\mathcal{I}$ heavily depends on the choice
of the set $M$ of concept descriptions.  We shall see later that we can choose this $M$ to
obtain a close connection between bases of $\con K_{\mathcal{I}, M}$ and bases of
$\mathcal{I}$.

We start our considerations about induced contexts by introducing some auxiliary notions
first.  For a finite set $U \subseteq M$ we define the set
\begin{equation*}
  \bigsqcap U :=
  \begin{cases}
    \top & \text{ if } U = \emptyset, \\
    \bigsqcap_{V \in U} V & otherwise.
  \end{cases}
\end{equation*}
We call $\bigsqcap U$ the \emph{concept description defined by} $U$.  Furthermore, for a
concept description $C$ we define the \emph{projection} of $C$ onto $M$ as
\begin{equation*}
  \pr_M(C) := \set{ D \in M \mid C \sqsubseteq D }.
\end{equation*}

Concept descriptions defined by subsets of $M$ together with projections capture some kind
of notion of \emph{upper approximation} in terms of $M$: if $C$ is a concept description,
then the most-specific concept description $D$ satisfying $C \sqsubseteq D$ that can be
defined by a subset of $M$ is given by
\begin{equation*}
  D = \bigsqcap \pr_M(C).
\end{equation*}
This looks familiar to our introductory motivation for model-based most-specific concept
descriptions, and indeed there are similarities.  One of them is that $U \mapsto \bigsqcap
U$ and $C \mapsto \pr_M(C)$ satisfy the main condition of an antitone Galois connection.

\begin{Lemma}
  \label{lem:pr-bigsqcap-forms-Galois-connection}
  Let $M$ be a finite set of concept descriptions over $N_C$ and $N_R$.  Then for each $U
  \subseteq M$ and each concept description $C$ over $N_C$ and $N_R$ it is true that
  \begin{equation*}
    C \sqsubseteq \bigsqcap U \iff U \subseteq \pr_M(C).
  \end{equation*}
  In particular, the following statements holds for all $U, V \subseteq M$ and all concept
  descriptions $C, D$ over $N_C$ and $N_R$.
  \begin{enumerate}[i. ]
  \item $C \sqsubseteq D \implies \pr_M(D) \subseteq \pr_M(C)$,
  \item $U \subseteq V \implies \bigsqcap V \sqsubseteq \bigsqcap U$,
  \item $C \sqsubseteq \bigsqcap \pr_M(C)$,
  \item $U \subseteq \pr_M(\bigsqcap U)$.
  \end{enumerate}
\end{Lemma}
\begin{Proof}
  Assume $C \sqsubseteq \bigsqcap U$.  Then $\pr_M(\bigsqcap U) \subseteq \pr_M(C)$, since
  every concept description $D \in M$ satisfying $\bigsqcap U \sqsubseteq D$ also
  satisfies $C \sqsubseteq D$.  Furthermore, $U \subseteq \pr_M(\bigsqcap U)$, since for
  each $F \in U$ it is true that $\bigsqcap U \sqsubseteq F$.  Thus
  \begin{equation*}
    U \subseteq \pr_M(\bigsqcap U) \subseteq \pr_M(C).
  \end{equation*}
  For the converse direction assume that $U \subseteq \pr_M(C)$.  Then $\bigsqcap \pr_M(C)
  \sqsubseteq \bigsqcap U$.  Since for each $D \in \pr_M(C)$ it is true that $C
  \sqsubseteq D$, we also have $C \sqsubseteq \bigsqcap \pr_M(C)$.  In sum, we obtain
  \begin{equation*}
    C \sqsubseteq \bigsqcap \pr_M(C) \sqsubseteq \bigsqcap U.
  \end{equation*}
\end{Proof}

For certain concept descriptions $C$, the upper approximation provided by $\bigsqcap
\pr_M(C)$ coincides with $C$.  Those concept descriptions are exactly those which are
\emph{expressible in terms of} $M$, \ie there exists a subset $N \subseteq M$ such that $C
\equiv \bigsqcap N$, as the following small result shows.

\begin{Lemma}[\cite{Diss-Felix}]
  \label{lem:characterizing-expressible-in-terms-of}
  Let $M \cup \set{C}$ be a set of concept descriptions over $N_C$ and $N_R$.  Then $C$ is
  expressible in terms of $M$ if and only if
  \begin{equation*}
    C \equiv \bigsqcap \pr_M(C).
  \end{equation*}
\end{Lemma}
\begin{Proof}
  Clearly, if $C \equiv \bigsqcap \pr_M(C)$, then $C$ is expressible in term of $M$.
  Conversely, if $C$ is expressible in terms of $M$, then $C \equiv \bigsqcap N$ for some
  $N \subseteq M$.  Then $C \sqsubseteq D$ for all $D \in N$, and therefore $N \subseteq
  \pr_M(C)$.  By \Cref{lem:pr-bigsqcap-forms-Galois-connection}, it is thus true that
  \begin{equation*}
    C \sqsubseteq \bigsqcap \pr_M(C) \sqsubseteq \bigsqcap N \equiv C
  \end{equation*}
  and therefore $C \equiv \bigsqcap \pr_M(C)$.
\end{Proof}

We can now state some connections between the derivation operators of an induced context
on one side, and computing the extension of a concept description as well as model-based
most-specific concept descriptions on the other.  These results are rather technical but
necessary for our further considerations.  We include the proofs of these statements here,
as they are rather simple and may help to better understand to corresponding claims.

\begin{Proposition}[Lemma~4.11 and~4.12 of~\cite{Diss-Felix}]
  \label{prop:connection-I-prime-1}
  Let $\mathcal{I}$ be a finite interpretation and $M$ be a finite set of concept
  descriptions.  Then for every concept description expressible in terms of $M$ it is true
  that
  \begin{equation*}
    C^{\mathcal{I}} = (\pr_M(C))',
  \end{equation*}
  and for $O \subseteq \Delta^{\mathcal{I}}$ it is true that
  \begin{equation*}
    O' = \pr_M(O^{\mathcal{I}}),
  \end{equation*}
  where the derivation is conducted in $\con K_{\mathcal{I}, M}$.
\end{Proposition}
\begin{Proof}
  Since $C$ is expressible in terms of $M$,
  \Cref{lem:characterizing-expressible-in-terms-of} yields $C \equiv \bigsqcap \pr_M(C)$.
  Thus
  \begin{align*}
    x \in C^{\mathcal{I}}
    & \iff x \in (\bigsqcap \pr_M(C))^{\mathcal{I}} \\
    & \iff \forall D \in \pr_M(C) \holds x \in D^{\mathcal{I}} \\
    & \iff x \in (\pr_M(C))',
  \end{align*}
  since $(\pr_M(C))' = \set{ x \in \Delta^{\mathcal{I}} \mid \forall D \in \pr_M(C) \holds
    x \in D^{\mathcal{I}} }$.

  If $O \subseteq \Delta^{\mathcal{I}}$, then
  \begin{align*}
    D \in O'
    & \iff \forall g \in O \holds g \in D^{\mathcal{I}} \\
    & \iff O \subseteq D^{\mathcal{I}} \\
    & \iff O^{\mathcal{I}} \sqsubseteq D \\
    & \iff D \in \pr_M(O^{\mathcal{I}}),
  \end{align*}
  where $O \subseteq D^{\mathcal{I}} \iff O^{\mathcal{I}} \sqsubseteq D$ holds due to
  \Cref{lem:mmsc-and-extension-are-galois-connection}.
\end{Proof}

\begin{Proposition}[Lemma~4.10 and~4.11 of~\cite{Diss-Felix}]
  \label{prop:connection-I-prime-2}
  Let $\mathcal{I}$ be a finite interpretation and let $M$ be a finite set of concept
  descriptions.  Then each $B \subseteq M$ satisfies
  \begin{equation*}
    B' = (\bigsqcap B)^{\mathcal{I}},
  \end{equation*}
  and if $A \subseteq \Delta^{\mathcal{I}}$ is such that $A^{\mathcal{I}}$ is expressible
  in terms of $M$, then
  \begin{equation*}
    \bigsqcap A' \equiv A^{\mathcal{I}},
  \end{equation*}
  where all derivations are conducted in $\con K_{\mathcal{I}, M} = (\Delta^{\mathcal{I}},
  M, \nabla)$.
\end{Proposition}
\begin{Proof}
  Observe that $x \in B'$ if and only if $x \in C^{\mathcal{I}}$ for all $C \in B$.
  Therefore
  \begin{equation*}
    x \in B' \iff \forall C \in B \holds x \in C^{\mathcal{I}} \iff x \in \bigcap_{C \in
      B} C^{\mathcal{I}} = (\bigsqcap B)^{\mathcal{I}},
  \end{equation*}
  and therefore $B' = (\bigsqcap B)^{\mathcal{I}}$.

  If $A \subseteq \Delta^{\mathcal{I}}$ is such that $A^{\mathcal{I}}$ is expressible in
  terms of $M$, then by \Cref{lem:characterizing-expressible-in-terms-of} it is true that
  \begin{equation*}
    A^{\mathcal{I}} \equiv \bigsqcap \pr_M(A^{\mathcal{I}}).
  \end{equation*}
  By \Cref{prop:connection-I-prime-1}, $\pr_M(A^{\mathcal{I}}) = A'$, and thus
  $A^{\mathcal{I}} \equiv \bigsqcap A'$ as required.
\end{Proof}

\begin{Proposition}
  \label{prop:connection-I-prime-3}
  Let $\mathcal{I} = (\Delta^{\mathcal{I}}, \cdot^{\mathcal{I}})$ be a finite
  interpretation and let $M$ be a set of concept descriptions.  Let $A \subseteq
  \Delta^{\mathcal{I}}$ such that $A^{\mathcal{I}}$ is expressible in terms of $M$.  Then
  $A^{\mathcal{I}\mathcal{I}} = X''$, where the derivations are conducted in $\con
  K_{\mathcal{I}, M}$.
\end{Proposition}
\begin{Proof}
  Again, by \Cref{lem:characterizing-expressible-in-terms-of} we have $A^{\mathcal{I}}
  \equiv \bigsqcap \pr_M(X^{\mathcal{I}})$ and thus
  \begin{align*}
    X^{\mathcal{I}\mathcal{I}}
    &= \bigl(\bigsqcap \pr_M(X^{\mathcal{I}})\bigr)^{\mathcal{I}} \\
    &= \pr_M(X^{\mathcal{I}})' \\
    &= X''
  \end{align*}
  by \Cref{prop:connection-I-prime-1} and \Cref{prop:connection-I-prime-2}.
\end{Proof}

We can rephrase some of the above results as follows.  Let $\mathcal{I}$ be a finite
interpretation and let us call a concept description $C$ a \emph{model-based most-specific
  concept description} of $\mathcal{I}$ if it is the model-based most-specific concept
description of some set $X \subseteq \Delta^{\mathcal{I}}$.  Note that $C$ is a
model-based most-specific concept description of $\mathcal{I}$ if and only if $C \equiv
C^{\mathcal{I}\mathcal{I}}$.

Let $M$ be a set of concept descriptions such that all model-based most-specific concept
descriptions are expressible in terms of $M$.  If we then identify equivalent model-based
most-specific concept descriptions and order them by $\sqsubseteq$, then the resulting
ordered set is dually isomorphic to the lattice of intents of $\con K_{\mathcal{I}, M}$.
Recall that with $\Int(\con K_{\mathcal{I}, M})$ we denote the set of intents of $\con
K_{\mathcal{I}, M}$.

\begin{Corollary}
  \label{cor:mmsc-lattice}
  Let $\mathcal{I}$ be a finite interpretation and let $M$ be a set of concept
  descriptions such that model-based most-specific concept descriptions of $\mathcal{I}$
  are expressible in terms of $M$.  Denote with $\mathcal{M}$ the set of all model-based
  most-specific concept descriptions considered up to equivalence.  Then the mapping
  \begin{equation*}
    \begin{array}{cccc}
      \phi \colon & \Int(\con K_{\mathcal{I}, M}) & \to     & \mathcal{M} \\
      ~           & U                         & \mapsto & \bigsqcap U
    \end{array}
  \end{equation*}
  is an order-isomorphism between $(\Int(\con K_{\mathcal{I}, M}), \subseteq)$ and
  $(\mathcal{M}, \sqsupseteq)$, where
  \begin{equation*}
    \phi^{-1}(C) = \pr_M(C) \quad (C \in \mathcal{M}).
  \end{equation*}
  In particular this means
  \begin{enumerate}[i. ]
  \item\label{item:10} $\bigsqcap U \in \mathcal{M}$ for all $U \in \Int(\con K_{\mathcal{I}, M})$,
  \item\label{item:11} $\pr_M(C) \in \Int(\con K_{\mathcal{I}, M})$ for all $C \in \mathcal{M}$,
  \item\label{item:12} $U \subseteq V$ implies $\bigsqcap U \sqsupseteq \bigsqcap V$ for
    all $U, V \subseteq M$,
  \item\label{item:13} $C \sqsubseteq D$ implies $\pr_M(C) \supseteq \pr_M(D)$ for all $C,
    D \in \mathcal{M}$,
  \item\label{item:14} $\pr_M(\bigsqcap U) = U$ for all $U \in \Int(\con K_{\mathcal{I}, M})$,
  \item\label{item:15} $\bigsqcap \pr_M(C) \equiv C$ for each $C \in \mathcal{M}$.
  \end{enumerate}
  Additionally,
  \begin{equation}
    \label{eq:18}
    \begin{aligned}
      U'' &= \pr_M((\bigsqcap U)^{\mathcal{I}\mathcal{I}}), \\
      C^{\mathcal{I}\mathcal{I}} &= \bigsqcap (\pr_M(C))''
    \end{aligned}
  \end{equation}
  is true for all $U \subseteq M$ and all concept descriptions $C$ expressible in terms of
  $M$, and where the derivations are done in $\con K_{\mathcal{I}, M}$.
\end{Corollary}
\begin{Proof}
  Claims~\ref{item:12} and~\ref{item:13} are already contained in
  \Cref{lem:pr-bigsqcap-forms-Galois-connection}, and \cref{item:15} is just
  \Cref{lem:characterizing-expressible-in-terms-of} again.  We show the other claims step
  by step.

  For~\cref{item:10} let $U \in \Int(\con K_{\mathcal{I}, M})$.  Then $U = U''$, and thus
  \begin{equation*}
    \bigsqcap U = \bigsqcap U'' \equiv (U')^{\mathcal{I}} = (\bigsqcap U)^{\mathcal{I}\mathcal{I}}
  \end{equation*}
  by \Cref{prop:connection-I-prime-2}.  Thus $U \in \mathcal{M}$ up to equivalence.

  For~\cref{item:11} let $C \in \mathcal{M}$.  Then $C \equiv C^{\mathcal{I}\mathcal{I}}$
  and $C$ is expressible in terms of $M$.  From \Cref{prop:connection-I-prime-1} it
  follows
  \begin{align*}
    \pr_M(C)
    &= \pr_M(C^{\mathcal{I}\mathcal{I}}) \\
    &= (C^{\mathcal{I}})' \\
    &= \pr_M(C)''
  \end{align*}
  and thus $\pr_M(C) \in \Int(\con K_{\mathcal{I}, M})$.

  For~\cref{item:14} let again $U \in \Int(\con K_{\mathcal{I}, M})$.  We first observe
  that because of \Cref{lem:pr-bigsqcap-forms-Galois-connection} it is true that $U
  \subseteq \pr_M(\bigsqcap U)$.  Furthermore, for each concept description $D$ it is true
  that
  \begin{align*}
    D \in \pr_M(\bigsqcap U)
    &\iff \bigsqcap U \sqsubseteq D \\
    &\:\implies (\bigsqcap U)^{\mathcal{I}} \subseteq D^{\mathcal{I}} \\
    &\iff U' \subseteq \set{ D }'\\
    &\iff U'' \supseteq \set{ D }'' \ni D \\
    &\iff D \in U'' = U,
  \end{align*}
  using \Cref{prop:connection-I-prime-2} for $(\bigsqcap U)^{\mathcal{I}} = U'$, and the
  definition of $\con K_{\mathcal{I}, M}$ to obtain $D^{\mathcal{I}} = \set{ D }'$.  Thus,
  $\pr_M(\bigsqcap U) \subseteq U$ and equality follows.

  For the equations given in~(\ref{eq:18}) we observe
  \begin{align*}
    \pr_M((\bigsqcap U)^{\mathcal{I}\mathcal{I}}
    &= \pr_M((U')^{\mathcal{I}}) \\
    &= U''
  \intertext{by \Cref{prop:connection-I-prime-2} and \Cref{prop:connection-I-prime-1}, and}
    \bigsqcap (\pr_M(C))''
    &\equiv (\pr_M(C)')^{\mathcal{I}} \\
    &= C^{\mathcal{I}\mathcal{I}},
  \end{align*}
  again because of \Cref{prop:connection-I-prime-2} and \Cref{prop:connection-I-prime-1},
  for every $U \subseteq M$ and every concept description $C$ expressible in terms of $M$.
\end{Proof}

The equivalence $\bigsqcap (\pr_M(C))'' \equiv C^{\mathcal{I}\mathcal{I}}$ does not hold
in general for concept descriptions $C$, as the following trivial example shows.

\begin{Example}
  \label{expl:counterexample}
  Let $N_C = \emptyset$ and $N_R = \set{ \mathsf{r} }$, and let $\mathcal{I} =
  (\Delta^{\mathcal{I}}, \cdot^{\mathcal{I}})$ be an interpretation over $N_C$ and $N_R$
  with $\Delta^{\mathcal{I}} = \set{ x }$ and $r^{\mathcal{I}} = \emptyset$.  Then the
  model-based most-specific concept descriptions of $\mathcal{I}$ are, up to equivalence,
  just $\top$ and $\bot$.  Let $M = \set{ \bot }$.  Then clearly all model-based
  most-specific concept descriptions of $\mathcal{I}$ are expressible in terms of $M$.
  Then
  \begin{equation*}
    \con K_{\mathcal{I}, M} =
    \begin{array}{c|c}
      ~ & \bot \\\midrule
      x & . \\
    \end{array}
  \end{equation*}
  Now consider $C = \exists r. \top$.  Then on the one hand,
  \begin{equation*}
    C^{\mathcal{I}\mathcal{I}} = \emptyset^{\mathcal{I}} = \bot,
  \end{equation*}
  but on the other hand
  \begin{equation*}
    \bigsqcap \pr_M(C)'' = \bigsqcap \emptyset'' = \bigsqcap \emptyset = \top,
  \end{equation*}
  so $C^{\mathcal{I}\mathcal{I}} \neq \bigsqcap \pr_M(C)''$.
\end{Example}

\section{Computing Bases of Valid GCIs of a Finite Interpretation}
\label{sec:base-all-valid}


\todo[inline]{Write: main idea why model-based most-specific concept description yield
  bases}%
\todo[inline]{Write: acyclic case?}%
\todo[inline]{Write: finite case, introduce and motivate $M_{\mathcal{I}}$}%
\todo[inline]{Write: formulate minimality result}%
\todo[inline]{Write: show that every base of the induced formal contexts gives rise to a
  base of the underlying interpretation}%

%%% Local Variables: 
%%% mode: latex
%%% TeX-master: "../main"
%%% End: 

%  LocalWords:  Prediger
