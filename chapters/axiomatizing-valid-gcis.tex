\chapter{Axiomatizing Valid General Concept Inclusions of Finite
  Interpretations}
\label{cha:axiom-valid-el}

Our considerations about extracting general concept inclusions from erroneous data will be
based on previous results obtained by Baader and Distel~\cite{Diss-Felix} on extracting
all \emph{valid} general concept inclusions from a given finite interpretation.  In this
section, we shall therefore review the notions and results from this work that are
necessary for our own.

The problem of extracting all valid general concept inclusions from a finite
interpretation can be made more precise as follows.  Let $\mathcal{I} =
(\Delta^{\mathcal{I}}, \cdot^{\mathcal{I}})$ be a finite interpretation over $N_C$ and
$N_R$, \ie $\Delta^{\mathcal{I}}$ is a finite set.  The task is then to find the set of
all general concept inclusions $C \sqsubseteq D$ with $C, D \in \ELbot(N_C, N_R)$ which
are valid in $\mathcal{I}$.

Of course, the set of all valid general concept inclusions is infinite in general.  This
is because if $C \sqsubseteq D$ holds in $\mathcal{I}$, and $r \in N_R$, then $\exists
r. C \sqsubseteq \exists r. D$ holds in $\mathcal{I}$ as well.  Such an infinite set is
hardly usable to represent knowledge suitable for machine consumption.  Therefore, the
considerations in~\cite{Diss-Felix} concentrate on finding \emph{finite bases} of
$\mathcal{I}$, \ie sets of valid general concept inclusions of $\mathcal{I}$ which are
also \emph{complete}.  We shall introduce these notions briefly in
\Cref{sec:bases-gener-conc}.

One of the main results of~\cite{Diss-Felix} then is that finite bases for finite
interpretations $\mathcal{I}$ always exist, and we shall discuss them in
\Cref{sec:base-all-valid}.  These results have been obtained by exploiting a close
connection between description logics and formal concept analysis.  It is therefore
crucial that we introduce this connection first, and we shall do so
in~\Cref{sec:motivation}.  In particular, we shall talk about \emph{induced contexts} and
\emph{model-based most-specific concept descriptions}.

\section{Bases of General Concept Inclusions}
\label{sec:bases-gener-conc}

General concept inclusions have a \emph{model-based semantics}, \ie their semantics is
defined in terms of being valid in some interpretation.  We can therefore introduce the
notions of \emph{entailment} and \emph{completeness} as follows.  Also notice the
similarity of this definition to \Cref{def:sound-complete-base}.

\begin{Definition}
  \label{def:entailment-of-gcis}
  Let $\mathcal{L} \cup \set{ C \sqsubseteq D }$ be a set of general concept inclusions
  over $N_C$ and $N_R$.  We shall say that $\mathcal{L}$ \emph{entails} $C \sqsubseteq D$,
  written $\mathcal{L} \models (C \sqsubseteq D)$ if and only if for all interpretations
  over $N_C$ and $N_R$ it is true that if $\mathcal{I} \models \mathcal{L}$, then
  $\mathcal{I} \models \set{ C \sqsubseteq D }$ as well.

  Let $\mathcal{K}$ be another set of general concept inclusions over $N_C$ and $N_R$.
  Then $\mathcal{K}$ is said to be \emph{sound} for $\mathcal{L}$ if and only if all
  general concept inclusions in $\mathcal{K}$ are entailed by $\mathcal{L}$.
  $\mathcal{K}$ is said to be \emph{complete} for $\mathcal{L}$ if and only if all general
  concept inclusions in $\mathcal{L}$ are \emph{entailed} by $\mathcal{K}$.  $\mathcal{K}$
  is said to be a base for $\mathcal{L}$ if and only if $\mathcal{K}$ is sound and
  complete for $\mathcal{L}$.
\end{Definition}

Let $\mathcal{I}$ be a finite interpretation over $N_C$ and $N_R$, and let us denote with
$\Th(\mathcal{I})$ the set of all \ELgfpbot general concept inclusions over $N_C$ and
$N_R$ which are valid in $\mathcal{I}$, \ie
\begin{equation*}
  \Th(\mathcal{I}) := \set{ C \sqsubseteq D \mid C, D \in \ELgfpbot(N_{C}, N_{R}),
    C^{\mathcal{I}} \subseteq D^{\mathcal{I}} }.
\end{equation*}
Let $\mathcal{K}$ be a set of general concept inclusions over $N_C$ and $N_R$.  If
$\mathcal{K}$ is a base of $\Th(\mathcal{I})$, we shall simply say that $\mathcal{K}$ is a
\emph{base} of $\mathcal{I}$.  If $\mathcal{K}$ consists of $\ELbot$ general concept
inclusions only, we shall say that $\mathcal{K}$ is an \emph{\ELbot base} of
$\mathcal{I}$.  Otherwise, we shall occasionally say that $\mathcal{K}$ is an
\emph{\ELgfpbot base} of $\mathcal{I}$.

Notice that in the case that $\mathcal{K}$ is a base of $\mathcal{I}$, all general concept
inclusions in $\mathcal{K}$ have to hold in $\mathcal{I}$: the set $\Th(\mathcal{I})$ is
\emph{closed under entailment} in the sense that every \ELgfpbot general concept inclusion
over $N_C$ and $N_R$ which is entailed by $\Th(\mathcal{I})$ is already contained in this
set.  Therefore, if $\mathcal{K}$ is sound for $\Th(\mathcal{I})$, it must be contained in
this set and thus $\mathcal{K}$ is a set of general concept inclusions which are valid in
$\mathcal{I}$.  Moreover, since $\mathcal{K}$ is complete for $\Th(\mathcal{I})$, every
\ELgfpbot general concept inclusion over $N_C$ and $N_R$ that holds in $\mathcal{I}$ is
entailed by $\mathcal{K}$.

\section{Linking Formal Concept Analysis and Description Logics}
\label{sec:motivation}

Description logics and formal concept analysis are connected by a number of similar
notions.  As an example, let us consider a formal context $\con K = (G, M, I)$ and a set
$A \subseteq M$.  Then the set $A'$ is the set of all objects of $\con K$ which have all
the attributes in $A$.  We can view this fact from another perspective: if $A = \set{ m_1,
  \dots, m_n }$, then we can think of the attributes $m_1, \dots, m_n$ as
\emph{propositions}, and the fact that $(g, m) \in I$ as saying that $g$ \emph{satisfies}
the proposition $m$.  Then $g \in A'$ means that $g$ \emph{satisfies} the conjunction of
all propositions in $A$.

Let us reformulate this using description logics.  To this end, let us define $N_C := M$
and $N_R = \emptyset$.  Then we can think of $\con K$ as an interpretation
$\mathcal{I}_{\con K} = (G, \cdot^{\mathcal{I}_{\con K}})$ where
\begin{equation}
  \label{eq:17}
  m^{\mathcal{I}_{\con K}} := \set{ g \in G \mid (g, m) \in I } = \set{ m }'.
\end{equation}
Then we have $A' = (m_1 \sqcap \dots \sqcap m_n)^{\mathcal{I}_{\con K}}$ for all finite $A
= \set{ m_1, \dots, m_n } \subseteq M$.  Indeed, if we would consider a description logic
that only allows for conjunction $\sqcap$, then we can view finite formal contexts,
derivation of sets of attributes and even implications as special cases of finite
interpretations, extensions of concept descriptions and general concept inclusions.  Thus
the derivation operator $(\cdot)' \colon \subsets{M} \to \subsets{G}$ naturally
corresponds to computing the extension of concept descriptions in interpretations.
However, the other derivation operator $(\cdot)' \colon \subsets{G} \to \subsets{M}$ does
not have such a correspondence in description logics.  This gap shall be filled by
considering \emph{model-based most-specific concept descriptions}, which we introduce in
\Cref{sec:defin-and-basic}.

The connection between description logics and formal concept analysis expressed in
\eqref{eq:17} only works in one direction: it allows to represent basic notions of formal
concept analysis in terms of description logics, but not vice versa.  Even if we restrict
our attention to the rather light-weight description logic \ELbot, it is not clear how to
represent an interpretation by means of notions from formal concept analysis.

To approach this issue, we shall introduce \emph{induced contexts} in
\Cref{sec:induced-contexts}.  Such contexts allow to express tight connections between the
notions of formal concept analysis and description logics, and, since induced contexts are
just formal contexts, still allow the application of standard methods from formal concept
analysis, such as the extraction of bases.  This fact will be exploited when we discuss
the computation of finite bases in \Cref{sec:base-all-valid}.

\subsection{Model-Based Most-Specific Concept Descriptions}
\label{sec:defin-and-basic}

Let $\con K = (G, M, I)$, and let us try to motivate how to find a natural correspondence
of the derivation operator $(\cdot)' \colon \subsets{G} \to \subsets{M}$ within
description logics.  Let $B \subseteq G$ be a set of objects of $\con K$.  Then the set $A
:= B'$ can be thought of as the \emph{most-specific} set of attributes that
\emph{describe} $B$, \ie
\begin{enumerate}[i. ]
\item $B \subseteq A'$, \ie $A$ \emph{describes} $B$, and
\item for all sets $C \subseteq M$ that satisfy $B \subseteq C'$ (that describe $B$) it is
  true that $C \subseteq A$, ($A$ contains \emph{more} attributes than $C$, \ie is
  \emph{more specific}).
\end{enumerate}
The last point is true because if $B \subseteq C'$, then by
\Cref{lem:derivation-is-galois-connection} it is true that $C \subseteq B' = A$.  Notice
that the description of $A$ as a most-specific description of $B$ is also a
characterization, \ie if $A$ is the most-specific description of $B$ in the above sense,
then $A = B'$.

To mimic this \emph{most-specific description} in description logics, Baader and Distel
introduce the notion of \emph{most-specific concept descriptions}.

\begin{Definition}[Most-Specific Concept Description]
  \label{def:most-specific-concept-description}
  Let $\mathcal{I} = (\Delta^{\mathcal{I}}, \cdot^{\mathcal{I}})$ be an interpretation, and
  let $X \subseteq \Delta^{\mathcal{I}}$.  A \emph{most-specific concept description} of
  $X$ in $\mathcal{I}$ is a concept description $C$ such that
  \begin{enumerate}[i. ]
  \item $X \subseteq C^{\mathcal{I}}$, and
  \item for each concept description $D$ satisfying $X \subseteq D^{\mathcal{I}}$ it is
    true that $C \sqsubseteq D$, \ie $C$ is subsumed by $D$.
  \end{enumerate}
\end{Definition}

If a model-based most-specific concept description $C$ for $X$ in $\mathcal{I}$ exists, it
is unique up to equivalence: if $D$ is another such model-based most-specific concept
description, than $C \sqsubseteq D$ and $D \sqsubseteq C$, by the last condition of the
definition.  Therefore, $C \equiv D$.  Because of this, we can talk about \emph{the}
model-based most-specific concept description of $X$ in $\mathcal{I}$, and shall denote it
with $X^{\mathcal{I}}$, to stress the similarity to the derivation operator from formal
concept analysis.  We shall also write $X^{\mathcal{I}\mathcal{I}}$ instead of
$(X^{\mathcal{I}})^{\mathcal{I}}$ and $C^{\mathcal{I}\mathcal{I}}$ instead of
$(C^{\mathcal{I}})^{\mathcal{I}}$ for syntactic convenience.

The existence of model-based most-specific concept descriptions, however, is not clear per
se, and the choice of the description logic in which we seek for model-based most-specific
concept descriptions is crucial here: if we only consider \ELbot concept descriptions,
then model-based most-specific concept descriptions do not necessarily exist, as is shown
in \Cref{expl:mmscs-may-not-exist-in-ELbot}.  However, if we allow all concept
descriptions in \Cref{def:most-specific-concept-description} to be \ELgfp or \ELgfpbot
concept descriptions, then the existence of model-based most-specific concept descriptions
can be guaranteed.

\begin{Theorem}[Theorem 4.7 of~\cite{Diss-Felix}]
  \label{thm:existence-of-mmscs-in-ELgfpbot}
  Model-based most-specific concept descriptions exist in \ELgfp\ and \ELgfpbot for all
  finite interpretations $\mathcal{I} = (\Delta^{\mathcal{I}}, \cdot^{\mathcal{I}})$ and
  sets $X \subseteq \Delta^{\mathcal{I}}$, and they can be computed effectively.
\end{Theorem}

The computation of model-based most-specific concept descriptions can be achieved using
\emph{\EL description graphs}, least common subsumers and
\emph{simulations}~\cite{DBLP:conf/ijcai/Baader03a,Diss-Felix}.  See~\cite[Section
4.1.2]{Diss-Felix} for details on this.

We have motivated model-based most-specific concept descriptions by most-specific
descriptions in formal contexts, and for this we have made use of the fact that the
derivation operators form a Galois connection.  It is therefore only natural to expect
that model-based most-specific concept descriptions are also part of a Galois connection.
However, we have to notice that we cannot expect to obtain a Galois connection in the
sense of \Cref{sec:galois-connections}, simply because the relation $\sqsubseteq$ is not
antisymmetric, and thus not an order relation: it may be the case that $C \sqsubseteq D$
and $C \sqsubseteq D$, but $C \neq D$.  We can remedy this fact by considering concept
descriptions only \emph{up to equivalence}: instead of a single concept description $C$,
we always consider the set $[C]$ of all concept descriptions which are equivalent to $C$.
Then $[C] \sqsubseteq [D]$ is well-defined for all concept descriptions $C$ and $D$, and
$\sqsubseteq$ indeed yields an order relation this way.  This is only a technical detail,
however, and we shall not make it explicit in our following considerations.

\begin{Lemma}[Lemma 4.1 of~\cite{Diss-Felix}]
  \label{lem:mmsc-and-extension-are-galois-connection}
  Let $\mathcal{I} = (\Delta^{\mathcal{I}}, \cdot^{\mathcal{I}})$ be an interpretation
  over $N_C$ and $N_R$, $X \subseteq \Delta^{\mathcal{I}}$ and $C$ an \ELgfpbot concept
  description over $N_C$ and $N_R$.  Then
  \begin{equation}
    \label{eq:19}
    X \subseteq C^{\mathcal{I}} \iff X^{\mathcal{I}} \sqsubseteq C.
  \end{equation}
  In particular, for $X, Y \subseteq \Delta^{\mathcal{I}}$ and for \ELgfpbot concept
  descriptions $C, D$ over $N_C$ and $N_R$, it is true that
  \begin{enumerate}[i. ]
  \item $X \subseteq Y \implies X^{\mathcal{I}} \sqsubseteq Y^{\mathcal{I}}$,
  \item $C \sqsubseteq D \implies C^{\mathcal{I}} \subseteq D^{\mathcal{I}}$,
  \item $X \subseteq X^{\mathcal{I}\mathcal{I}}$,
  \item $C^{\mathcal{I}\mathcal{I}} \sqsubseteq C$,
  \item $X^{\mathcal{I}} \equiv X^{\mathcal{I}\mathcal{I}\mathcal{I}}$,
  \item $C^{\mathcal{I}} = C^{\mathcal{I}\mathcal{I}\mathcal{I}}$.
  \end{enumerate}
\end{Lemma}
\begin{Proof}
  We only show \eqref{eq:19}, the other claims follow from
  \Cref{lem:properties-of-galois-connections} and the above-made considerations.  If $X
  \subseteq C^{\mathcal{I}}$, then $X^{\mathcal{I}} \sqsubseteq C$ because
  $X^{\mathcal{I}}$ is by definition the most-specific concept description that contains
  $X$ in its extension.  Conversely, if $X^{\mathcal{I}} \sqsubseteq C$, then by
  definition $X^{\mathcal{I}\mathcal{I}} \subseteq C^{\mathcal{I}}$.  But since
  $X^{\mathcal{I}}$ is the model-based most-specific concept description of $X$ in
  $\mathcal{I}$, it contains $X$ in its extension, \ie $X \subseteq
  X^{\mathcal{I}\mathcal{I}}$.  Therefore, $X \subseteq C^{\mathcal{I}}$.
\end{Proof}

Another useful property is the following, rather technical proposition.

\begin{Proposition}[Lemma 4.2 of~\cite{Diss-Felix}]
  \label{prop:double-II-under-I}
  Let $\mathcal{I}$ be an interpretation over $N_C$ and $N_R$, and let $C, D$ be \ELgfpbot
  concept descriptions over $N_C$ and $N_R$ and let $r \in N_R$.  Then
  \begin{enumerate}[i. ]
  \item $(C \sqcap D)^{\mathcal{I}} = (C^{\mathcal{I}\mathcal{I}} \sqcap
    D)^{\mathcal{I}}$, and
  \item $(\exists r. C)^{\mathcal{I}} = (\exists r. C^{\mathcal{I}\mathcal{I}})^{\mathcal{I}}$.
  \end{enumerate}
\end{Proposition}
\begin{Proof}
  For the first claim we use \Cref{lem:mmsc-and-extension-are-galois-connection} and obtain
  \begin{equation*}
    (C \sqcap D)^{\mathcal{I}} = C^{\mathcal{I}} \cap D^{\mathcal{I}} =
    C^{\mathcal{I}\mathcal{I}\mathcal{I}} \cap D^{\mathcal{I}} =
    (C^{\mathcal{I}\mathcal{I}} \sqcap D)^{\mathcal{I}}.
  \end{equation*}
  For the second one we observe that
  \begin{align*}
    (\exists r. C^{\mathcal{I}\mathcal{I}})^{\mathcal{I}}
    &= \set{ x \in \Delta^{\mathcal{I}} \mid \exists y \in \Delta^{\mathcal{I}} \st (x, y)
      \in r^{\mathcal{I}} \wedge y \in C^{\mathcal{I}\mathcal{I}\mathcal{I}} }\\
    &= \set{ x \in \Delta^{\mathcal{I}} \mid \exists y \in \Delta^{\mathcal{I}} \st (x, y)
      \in r^{\mathcal{I}} \wedge y \in C^{\mathcal{I}} } \\
    &= (\exists r. C)^{\mathcal{I}},
  \end{align*}
  again because of $C^{\mathcal{I}} = C^{\mathcal{I}\mathcal{I}\mathcal{I}}$ from
  \Cref{lem:mmsc-and-extension-are-galois-connection}.
\end{Proof}


\subsection{Induced Contexts}
\label{sec:induced-contexts}

We already have seen how formal contexts can be represented as interpretations.  In this
section we shall introduce the approach of Baader and Distel of \emph{induced contexts},
which somehow provides the inverse direction, \ie to represent interpretations as formal
contexts.  The notion of induced contexts also was implicitly used by works of
\textcite{books/math/Prediger00} in her study on \emph{terminological attribute logic}.

\begin{Definition}[Induced Context]
  \label{def:induced-context}
  Let $\mathcal{I} = (\Delta^{\mathcal{I}}, \cdot^{\mathcal{I}})$ be a finite
  interpretation over $N_C$ and $N_R$, and let $M$ be a set of concept descriptions over
  $N_C$ and $N_R$.  The \emph{induced context} of $\mathcal{I}$ and $M$ is the formal
  context $\con K_{\mathcal{I}, M} = (\Delta^{\mathcal{I}}, M, \nabla)$, where for $x \in
  \Delta^{\mathcal{I}}$ and $C \in M$
  \begin{equation*}
    (x, C) \in \nabla \diff x \in C^{\mathcal{I}}.
  \end{equation*}  
\end{Definition}

Induced formal contexts do not necessarily represent all of the interpretation
$\mathcal{I}$; indeed, what is represented of $\mathcal{I}$ heavily depends on the choice
of the set $M$ of concept descriptions.  We later shall see that we can choose this set
$M$ to obtain a close connection between bases of $\con K_{\mathcal{I}, M}$ and bases of
$\mathcal{I}$.

We start our considerations about induced contexts by introducing some auxiliary notions
first.  For a finite set $U \subseteq M$ we define the set
\begin{equation*}
  \bigsqcap U :=
  \begin{cases}
    \top & \text{ if } U = \emptyset, \\
    \bigsqcap_{V \in U} V & \text{otherwise}.
  \end{cases}
\end{equation*}
We call $\bigsqcap U$ the \emph{concept description defined by} $U$.  Furthermore, for a
concept description $C$ we define the \emph{projection} of $C$ onto $M$ as
\begin{equation*}
  \pr_M(C) := \set{ D \in M \mid C \sqsubseteq D }.
\end{equation*}

Concept descriptions defined by subsets of $M$ together with projections capture some kind
of notion of \emph{upper approximation} in terms of $M$: if $C$ is a concept description,
then the most-specific concept description $D$ satisfying $C \sqsubseteq D$ that can be
defined by a subset of $M$ is given by
\begin{equation*}
  D = \bigsqcap \pr_M(C).
\end{equation*}
This looks familiar to our introductory motivation for model-based most-specific concept
descriptions, and indeed there are similarities.  One of them is that $U \mapsto \bigsqcap
U$ and $C \mapsto \pr_M(C)$ satisfy the main condition of an antitone Galois connection.

\begin{Lemma}
  \label{lem:pr-bigsqcap-forms-Galois-connection}
  Let $M$ be a finite set of concept descriptions over $N_C$ and $N_R$.  Then for each $U
  \subseteq M$ and each concept description $C$ over $N_C$ and $N_R$ it is true that
  \begin{equation*}
    C \sqsubseteq \bigsqcap U \iff U \subseteq \pr_M(C).
  \end{equation*}
  In particular, the following statements holds for all $U, V \subseteq M$ and all concept
  descriptions $C, D$ over $N_C$ and $N_R$.
  \begin{enumerate}[i. ]
  \item $C \sqsubseteq D \implies \pr_M(D) \subseteq \pr_M(C)$,
  \item $U \subseteq V \implies \bigsqcap V \sqsubseteq \bigsqcap U$,
  \item $C \sqsubseteq \bigsqcap \pr_M(C)$,
  \item $U \subseteq \pr_M(\bigsqcap U)$.
  \end{enumerate}
\end{Lemma}
\begin{Proof}
  Assume $C \sqsubseteq \bigsqcap U$.  Then $\pr_M(\bigsqcap U) \subseteq \pr_M(C)$, since
  every concept description $D \in M$ satisfying $\bigsqcap U \sqsubseteq D$ also
  satisfies $C \sqsubseteq D$.  Furthermore, $U \subseteq \pr_M(\bigsqcap U)$, since for
  each $F \in U$ it is true that $\bigsqcap U \sqsubseteq F$.  Thus
  \begin{equation*}
    U \subseteq \pr_M(\bigsqcap U) \subseteq \pr_M(C).
  \end{equation*}
  For the converse direction, assume that $U \subseteq \pr_M(C)$.  Then $\bigsqcap
  \pr_M(C) \sqsubseteq \bigsqcap U$.  Since for each $D \in \pr_M(C)$ it is true that $C
  \sqsubseteq D$, we also have $C \sqsubseteq \bigsqcap \pr_M(C)$.  In sum, we obtain
  \begin{equation*}
    C \sqsubseteq \bigsqcap \pr_M(C) \sqsubseteq \bigsqcap U.
  \end{equation*}
\end{Proof}

For certain concept descriptions $C$, the upper approximation provided by $\bigsqcap
\pr_M(C)$ coincides with $C$.  Those concept descriptions are exactly those which are
\emph{expressible in terms of} $M$, \ie there exists a subset $N \subseteq M$ such that $C
\equiv \bigsqcap N$.

\begin{Lemma}[\cite{Diss-Felix}]
  \label{lem:characterizing-expressible-in-terms-of}
  Let $M \cup \set{C}$ be a set of concept descriptions over $N_C$ and $N_R$.  Then $C$ is
  expressible in terms of $M$ if and only if
  \begin{equation*}
    C \equiv \bigsqcap \pr_M(C).
  \end{equation*}
\end{Lemma}
\begin{Proof}
  Clearly, if $C \equiv \bigsqcap \pr_M(C)$, then $C$ is expressible in terms of $M$.
  Conversely, if $C$ is expressible in terms of $M$, then $C \equiv \bigsqcap N$ for some
  $N \subseteq M$.  Then $C \sqsubseteq D$ for all $D \in N$, and therefore $N \subseteq
  \pr_M(C)$.  By \Cref{lem:pr-bigsqcap-forms-Galois-connection}, it is thus true that
  \begin{equation*}
    C \sqsubseteq \bigsqcap \pr_M(C) \sqsubseteq \bigsqcap N \equiv C
  \end{equation*}
  and therefore $C \equiv \bigsqcap \pr_M(C)$.
\end{Proof}

We can now state some connections between the derivation operators of an induced context
on one side, and computing the extension of a concept description as well as model-based
most-specific concept descriptions on the other.  These results are rather technical but
necessary for our further considerations.  We include the proofs of these statements here,
as they are rather simple and may help to better understand the corresponding claims.

\begin{Proposition}[Lemma~4.11 and~4.12 of~\cite{Diss-Felix}]
  \label{prop:connection-I-prime-1}
  Let $\mathcal{I}$ be a finite interpretation and $M$ be a finite set of concept
  descriptions.  Then for every concept description expressible in terms of $M$ it is true
  that
  \begin{equation*}
    C^{\mathcal{I}} = (\pr_M(C))',
  \end{equation*}
  and for $O \subseteq \Delta^{\mathcal{I}}$ it is true that
  \begin{equation*}
    O' = \pr_M(O^{\mathcal{I}}),
  \end{equation*}
  where the derivation is conducted in $\con K_{\mathcal{I}, M}$.
\end{Proposition}
\begin{Proof}
  Since $C$ is expressible in terms of $M$,
  \Cref{lem:characterizing-expressible-in-terms-of} yields $C \equiv \bigsqcap \pr_M(C)$.
  Thus
  \begin{align*}
    x \in C^{\mathcal{I}}
    & \iff x \in (\bigsqcap \pr_M(C))^{\mathcal{I}} \\
    & \iff \forall D \in \pr_M(C) \holds x \in D^{\mathcal{I}} \\
    & \iff x \in (\pr_M(C))',
  \end{align*}
  since $(\pr_M(C))' = \set{ x \in \Delta^{\mathcal{I}} \mid \forall D \in \pr_M(C) \holds
    x \in D^{\mathcal{I}} }$.

  If $O \subseteq \Delta^{\mathcal{I}}$, then
  \begin{align*}
    D \in O'
    & \iff \forall g \in O \holds g \in D^{\mathcal{I}} \\
    & \iff O \subseteq D^{\mathcal{I}} \\
    & \iff O^{\mathcal{I}} \sqsubseteq D \\
    & \iff D \in \pr_M(O^{\mathcal{I}}),
  \end{align*}
  where $O \subseteq D^{\mathcal{I}} \iff O^{\mathcal{I}} \sqsubseteq D$ holds due to
  \Cref{lem:mmsc-and-extension-are-galois-connection}.
\end{Proof}

\begin{Proposition}[Lemma~4.10 and~4.11 of~\cite{Diss-Felix}]
  \label{prop:connection-I-prime-2}
  Let $\mathcal{I}$ be a finite interpretation and let $M$ be a finite set of concept
  descriptions.  Then each $B \subseteq M$ satisfies
  \begin{equation*}
    B' = (\bigsqcap B)^{\mathcal{I}},
  \end{equation*}
  and if $A \subseteq \Delta^{\mathcal{I}}$ is such that $A^{\mathcal{I}}$ is expressible
  in terms of $M$, then
  \begin{equation*}
    \bigsqcap A' \equiv A^{\mathcal{I}},
  \end{equation*}
  where all derivations are conducted in $\con K_{\mathcal{I}, M} = (\Delta^{\mathcal{I}},
  M, \nabla)$.
\end{Proposition}
\begin{Proof}
  Observe that $x \in B'$ if and only if $x \in C^{\mathcal{I}}$ for all $C \in B$.
  Therefore
  \begin{equation*}
    x \in B' \iff \forall C \in B \holds x \in C^{\mathcal{I}} \iff x \in \bigcap_{C \in
      B} C^{\mathcal{I}} = (\bigsqcap B)^{\mathcal{I}},
  \end{equation*}
  and therefore $B' = (\bigsqcap B)^{\mathcal{I}}$.

  If $A \subseteq \Delta^{\mathcal{I}}$ is such that $A^{\mathcal{I}}$ is expressible in
  terms of $M$, then by \Cref{lem:characterizing-expressible-in-terms-of} it is true that
  \begin{equation*}
    A^{\mathcal{I}} \equiv \bigsqcap \pr_M(A^{\mathcal{I}}).
  \end{equation*}
  By \Cref{prop:connection-I-prime-1}, $\pr_M(A^{\mathcal{I}}) = A'$, and thus
  $A^{\mathcal{I}} \equiv \bigsqcap A'$ as required.
\end{Proof}

\begin{Proposition}
  \label{prop:connection-I-prime-3}
  Let $\mathcal{I} = (\Delta^{\mathcal{I}}, \cdot^{\mathcal{I}})$ be a finite
  interpretation and let $M$ be a set of concept descriptions.  Let $A \subseteq
  \Delta^{\mathcal{I}}$ such that $A^{\mathcal{I}}$ is expressible in terms of $M$.  Then
  $A^{\mathcal{I}\mathcal{I}} = A''$, where the derivations are conducted in $\con
  K_{\mathcal{I}, M}$.
\end{Proposition}
\begin{Proof}
  Again, by \Cref{lem:characterizing-expressible-in-terms-of} we have $A^{\mathcal{I}}
  \equiv \bigsqcap \pr_M(A^{\mathcal{I}})$ and thus
  \begin{align*}
    A^{\mathcal{I}\mathcal{I}}
    &= \bigl(\bigsqcap \pr_M(A^{\mathcal{I}})\bigr)^{\mathcal{I}} \\
    &= \pr_M(A^{\mathcal{I}})' \\
    &= A''
  \end{align*}
  by \Cref{prop:connection-I-prime-1} and \Cref{prop:connection-I-prime-2}.
\end{Proof}

We can rephrase some of the above results as follows.  Let $\mathcal{I}$ be a finite
interpretation and let us call a concept description $C$ a \emph{model-based most-specific
  concept description} of $\mathcal{I}$ if it is the model-based most-specific concept
description of some subset of $\Delta^{\mathcal{I}}$.  Note that $C$ is a model-based
most-specific concept description of $\mathcal{I}$ if and only if $C \equiv
C^{\mathcal{I}\mathcal{I}}$.

Let $M$ be a set of concept descriptions such that all model-based most-specific concept
descriptions are expressible in terms of $M$.  If we then identify equivalent model-based
most-specific concept descriptions and order them by $\sqsubseteq$, then the resulting
ordered set is dually isomorphic to the lattice of intents of $\con K_{\mathcal{I}, M}$.
Note that with $\Int(\con K_{\mathcal{I}, M})$ we denote the set of intents of $\con
K_{\mathcal{I}, M}$.

\begin{Corollary}[contains Corollary~4.13 of~\cite{Diss-Felix}]
  \label{cor:mmsc-lattice}
  Let $\mathcal{I}$ be a finite interpretation and let $M$ be a set of concept
  descriptions such that model-based most-specific concept descriptions of $\mathcal{I}$
  are expressible in terms of $M$.  Denote with $\mathcal{M}$ the set of all model-based
  most-specific concept descriptions considered up to equivalence.  Then the mapping
  \begin{equation*}
    \begin{array}{cccc}
      \phi \colon & \Int(\con K_{\mathcal{I}, M}) & \to     & \mathcal{M} \\
      ~           & U                         & \mapsto & \bigsqcap U
    \end{array}
  \end{equation*}
  is an order-isomorphism between $(\Int(\con K_{\mathcal{I}, M}), \subseteq)$ and
  $(\mathcal{M}, \sqsupseteq)$, where
  \begin{equation*}
    \phi^{-1}(C) = \pr_M(C) \quad (C \in \mathcal{M}).
  \end{equation*}
  In particular this means
  \begin{enumerate}[i. ]
  \item\label{item:10} $\bigsqcap U \in \mathcal{M}$ for all $U \in \Int(\con K_{\mathcal{I}, M})$,
  \item\label{item:11} $\pr_M(C) \in \Int(\con K_{\mathcal{I}, M})$ for all $C \in \mathcal{M}$,
  \item\label{item:12} $U \subseteq V$ implies $\bigsqcap U \sqsupseteq \bigsqcap V$ for
    all $U, V \subseteq M$,
  \item\label{item:13} $C \sqsubseteq D$ implies $\pr_M(C) \supseteq \pr_M(D)$ for all $C,
    D \in \mathcal{M}$,
  \item\label{item:14} $\pr_M(\bigsqcap U) = U$ for all $U \in \Int(\con K_{\mathcal{I}, M})$,
  \item\label{item:15} $\bigsqcap \pr_M(C) \equiv C$ for each $C \in \mathcal{M}$.
  \end{enumerate}
  Additionally,
  \begin{equation}
    \label{eq:18}
    \begin{aligned}
      U'' &= \pr_M((\bigsqcap U)^{\mathcal{I}\mathcal{I}}), \\
      C^{\mathcal{I}\mathcal{I}} &= \bigsqcap (\pr_M(C))''
    \end{aligned}
  \end{equation}
  is true for all $U \subseteq M$ and all concept descriptions $C$ expressible in terms of
  $M$, and where the derivations are done in $\con K_{\mathcal{I}, M}$.
\end{Corollary}
\begin{Proof}
  Claims~(\ref{item:12}) and~(\ref{item:13}) are already contained in
  \Cref{lem:pr-bigsqcap-forms-Galois-connection}, and (\ref{item:15}) is just
  \Cref{lem:characterizing-expressible-in-terms-of} again.  We show the other claims step
  by step.

  For~(\ref{item:10}) let $U \in \Int(\con K_{\mathcal{I}, M})$.  Then $U = U''$, and thus
  \begin{equation*}
    \bigsqcap U = \bigsqcap U'' \equiv (U')^{\mathcal{I}} = (\bigsqcap U)^{\mathcal{I}\mathcal{I}}
  \end{equation*}
  by \Cref{prop:connection-I-prime-2}.  Thus $U \in \mathcal{M}$ up to equivalence.

  For~(\ref{item:11}) let $C \in \mathcal{M}$.  Then $C \equiv C^{\mathcal{I}\mathcal{I}}$
  and $C$ is expressible in terms of $M$.  From \Cref{prop:connection-I-prime-1} it
  follows
  \begin{align*}
    \pr_M(C)
    &= \pr_M(C^{\mathcal{I}\mathcal{I}}) \\
    &= (C^{\mathcal{I}})' \\
    &= \pr_M(C)''
  \end{align*}
  and thus $\pr_M(C) \in \Int(\con K_{\mathcal{I}, M})$.

  For~(\ref{item:14}) let again $U \in \Int(\con K_{\mathcal{I}, M})$.  We first observe
  that because of \Cref{lem:pr-bigsqcap-forms-Galois-connection} it is true that $U
  \subseteq \pr_M(\bigsqcap U)$.  Furthermore, for each concept description $D$ it is true
  that
  \begin{align*}
    D \in \pr_M(\bigsqcap U)
    &\iff \bigsqcap U \sqsubseteq D \\
    &\:\implies (\bigsqcap U)^{\mathcal{I}} \subseteq D^{\mathcal{I}} \\
    &\iff U' \subseteq \set{ D }'\\
    &\iff U'' \supseteq \set{ D }'' \ni D \\
    &\iff D \in U'' = U,
  \end{align*}
  using \Cref{prop:connection-I-prime-2} for $(\bigsqcap U)^{\mathcal{I}} = U'$, and the
  definition of $\con K_{\mathcal{I}, M}$ to obtain $D^{\mathcal{I}} = \set{ D }'$.  Thus,
  $\pr_M(\bigsqcap U) \subseteq U$ and equality follows.

  For the equations given in~(\ref{eq:18}) we observe
  \begin{align*}
    \pr_M((\bigsqcap U)^{\mathcal{I}\mathcal{I}})
    &= \pr_M((U')^{\mathcal{I}}) \\
    &= U''
  \intertext{by \Cref{prop:connection-I-prime-2} and \Cref{prop:connection-I-prime-1}, and}
    \bigsqcap (\pr_M(C))''
    &\equiv (\pr_M(C)')^{\mathcal{I}} \\
    &= C^{\mathcal{I}\mathcal{I}},
  \end{align*}
  again because of \Cref{prop:connection-I-prime-2} and \Cref{prop:connection-I-prime-1},
  for every $U \subseteq M$ and every concept description $C$ expressible in terms of $M$.
\end{Proof}

The equivalence $\bigsqcap (\pr_M(C))'' \equiv C^{\mathcal{I}\mathcal{I}}$ does not hold
in general for concept descriptions $C$, as the following trivial example shows.

\begin{Example}
  \label{expl:counterexample}
  Let $N_C = \emptyset$ and $N_R = \set{ \mathsf{r} }$, and let $\mathcal{I} =
  (\Delta^{\mathcal{I}}, \cdot^{\mathcal{I}})$ be an interpretation over $N_C$ and $N_R$
  with $\Delta^{\mathcal{I}} = \set{ x }$ and $r^{\mathcal{I}} = \emptyset$.  Then the
  model-based most-specific concept descriptions of $\mathcal{I}$ are, up to equivalence,
  just $\top$ and $\bot$.  Let $M = \set{ \bot }$.  Then clearly all model-based
  most-specific concept descriptions of $\mathcal{I}$ are expressible in terms of $M$.
  Then
  \begin{equation*}
    \con K_{\mathcal{I}, M} =
    \begin{array}{c|c}
      ~ & \bot \\\midrule
      x & . \\
    \end{array}
  \end{equation*}
  Now consider $C = \exists r. \top$.  Then on the one hand,
  \begin{equation*}
    C^{\mathcal{I}\mathcal{I}} = \emptyset^{\mathcal{I}} = \bot,
  \end{equation*}
  but on the other hand
  \begin{equation*}
    \bigsqcap \pr_M(C)'' = \bigsqcap \emptyset'' = \bigsqcap \emptyset = \top,
  \end{equation*}
  so $C^{\mathcal{I}\mathcal{I}} \neq \bigsqcap \pr_M(C)''$.
\end{Example}

A useful consequence of \Cref{cor:mmsc-lattice} is the following result.

\begin{Lemma}
  \label{lem:double-II-gets-double-prime}
  Let $\mathcal{I}$ be a finite interpretation, and let $U \subseteq M_{\mathcal{I}}$.  Then
  \begin{equation*}
    (\bigsqcap U)^{\mathcal{I}\mathcal{I}} = \bigsqcap U'',
  \end{equation*}
  where the derivations are done in $\con K_{\mathcal{I}}$.
\end{Lemma}
\begin{Proof}
  Clearly $\bigsqcap U$ is expressible in terms of $M_{\mathcal{I}}$.  Thus
  \Cref{cor:mmsc-lattice} yields
  \begin{equation*}
    (\bigsqcap U)^{\mathcal{I}\mathcal{I}} = \bigsqcap(\pr_{M_{\mathcal{I}}}(\bigsqcap U))''.
  \end{equation*}
  By \Cref{prop:connection-I-prime-1} it is true that $\pr_{M_{\mathcal{I}}}(\bigsqcap U)'
  = (\bigsqcap U)^{\mathcal{I}}$, thus
  \begin{align*}
    (\bigsqcap U)^{\mathcal{I}\mathcal{I}}
    &= \bigsqcap ((\bigsqcap U)^{\mathcal{I}})'\\
    &= \bigsqcap U''
  \end{align*}
  where $(\bigsqcap U)^{\mathcal{I}} = U'$ is true due to
  \Cref{prop:connection-I-prime-2}.
\end{Proof}


\section{Computing Bases of Valid GCIs of a Finite Interpretation}
\label{sec:base-all-valid}

Using the notions of model-based most-specific concept descriptions and induced contexts,
we are finally prepared to introduce some of the main results of~\cite{Diss-Felix} on
computing bases of finite interpretations.  The main idea behind these results is to use
ideas and methods from formal concept analysis, either by simulating them in a description
logic setting, or by transforming the initially given interpretation into formal contexts
and applying standard formal concept analysis methods to it.

Recall that for a (finite) formal context $\con K = (G, M, I)$ the set
\begin{equation*}
  \set{ A \to A'' \mid A \subseteq M }
\end{equation*}
is always a base of $\con K$.  This is because every valid implication $(A \to B) \in
\Th(\con K)$ already follows from $A \to A''$, because if $\con K \models (A \to B)$, then
$A' \subseteq B'$, \ie $B \subseteq A''$ and thus $\set{ A \to A ''} \models (A \to B)$.
Having introduced model-based most-specific concept descriptions, we are able to simulate
this result in terms of description logics as follows.

\begin{Lemma}[Lemma 4.3 of~\cite{Diss-Felix}]
  \label{lem:simple-entailment-with-mmsc}
  Let $\mathcal{I} = (\Delta^{\mathcal{I}}, \cdot^{\mathcal{I}})$ be an interpretation,
  and let $C \sqsubseteq D$ be a general concept inclusion that is valid in $\mathcal{I}$.
  Then $C \sqsubseteq C^{\mathcal{I}\mathcal{I}}$ is valid in $\mathcal{I}$ as well, and
  $C \sqsubseteq D$ follows from $C \sqsubseteq C^{\mathcal{I}\mathcal{I}}$.
\end{Lemma}

The following statement is then a simple corollary.

\begin{Corollary}
  \label{cor:Felix-base-B0}
  Let $\mathcal{I} = (\Delta^{\mathcal{I}}, \cdot^{\mathcal{I}})$ be an interpretation
  over $N_C$ and $N_R$.  Then
  \begin{equation}
    \label{eq:20}
    \mathcal{B}_0 := \set{ C \sqsubseteq C^{\mathcal{I}\mathcal{I}} \mid C \in
      \ELgfpbot(N_C, N_R), C \neq \bot }
  \end{equation}
  is a base of $\mathcal{I}$.
\end{Corollary}

Of course, this base is not finite in general, \ie if $N_R \neq \emptyset$.  However,
based on this result, Baader and Distel investigate subsets of $\mathcal{B}_0$ and finally
arrives at a finite base.  The first step into this direction is to show that considering
only \ELbot concept descriptions is enough, in the sense as described in the next theorem.
The main advantage of this result is that we can now use induction over the premises of
general concept inclusions.

\begin{Theorem}[Theorem~5.7 of~\cite{Diss-Felix}]
  \label{thm:Felix-base-B1}
  Let $\mathcal{I} = (\Delta^{\mathcal{I}}, \cdot^{\mathcal{I}})$ be an interpretation
  over $N_C$ and $N_R$.  Then
  \begin{equation}
    \label{eq:21}
    \mathcal{B}_1 := \set{ C \sqsubseteq C^{\mathcal{I}\mathcal{I}} \mid C \in \ELbot(N_C,
      N_R), C \neq \bot }
  \end{equation}
  is a base of $\mathcal{I}$.
\end{Theorem}

The proof of this theorem is quite involved, and again uses the notions \EL description
graphs and simulations between them.  We shall not go into details here, and refer the
reader to~\cite[Section 5.1.1]{Diss-Felix}.

The base $\mathcal{B}_1$ still is not finite in general.  To achieve finiteness, we
consider a particular finite set $M_{\mathcal{I}}$ of concept descriptions which turns out
to be enough, in the sense that we only need to consider general concept inclusions $C
\sqsubseteq C^{\mathcal{I}\mathcal{I}}$ where $C = \bigsqcap U$ for some $U \subseteq
M_{\mathcal{I}}$.  Since $M_{\mathcal{I}}$ is finite, the resulting set of general concept
inclusions is finite and therefore yields a finite base of $\mathcal{I}$.

\begin{Definition}[$M_{\mathcal{I}}$]
  \label{def:M_I}
  Let $\mathcal{I} = (\Delta^{\mathcal{I}}, \cdot^{\mathcal{I}})$ be a finite
  interpretation over $N_C$ and $N_R$.  Then
  \begin{equation*}
    M_{\mathcal{I}} := N_C \cup \set{ \bot } \cup \set{ \exists r. X^{\mathcal{I}} \mid
      r \in N_R, X \subseteq \Delta^{\mathcal{I}}, X \neq \emptyset }.
  \end{equation*}
\end{Definition}

The definition of $M_{\mathcal{I}}$ seems to be incomprehensible at first.  However, since
this set will play a major role for our further considerations, we shall give some
intuition why it is suitable for our purposes the way it is defined.

Note that $M_{\mathcal{I}}$ is finite since $\mathcal{I}$ is finite, and thus there are
only finitely many subsets of $\Delta^{\mathcal{I}}$.  Furthermore notice that
$M_{\mathcal{I}}$ can be computed using the Next-Closure algorithm from
\Cref{thm:next-closure}.  More precisely, we can compute all concept descriptions
$X^{\mathcal{I}}$ by noticing that $X^{\mathcal{I}} \equiv
X^{\mathcal{I}\mathcal{I}\mathcal{I}}$, and we can compute the sets
$X^{\mathcal{I}\mathcal{I}}$ using Next-Closure because the mapping $X \mapsto
X^{\mathcal{I}\mathcal{I}}$ is a closure operator on $\subsets{M_{\mathcal{I}}}$.

Before we show how the set $M_{\mathcal{I}}$ helps in finding finite bases, we note an
important property of it.

\begin{Lemma}[Lemma~5.9 of~\cite{Diss-Felix}]
  \label{lem:mmsc-are-expressible-in-terms-of-M_I}
  Let $\mathcal{I}$ be a finite interpretation and let $C$ be a model-based most-specific
  concept description of $\mathcal{I}$.  Then $C$ is expressible in terms of
  $M_{\mathcal{I}}$.
\end{Lemma}

Let us define
\begin{equation}
  \label{eq:26}
  \mathcal{B}_2 := \set{ \bigsqcap U \sqsubseteq (\bigsqcap U)^{\mathcal{I}\mathcal{I}}
    \mid U \subseteq M_{\mathcal{I}} }.
\end{equation}
Then clearly $\mathcal{B}_2 \models (C \sqsubseteq C^{\mathcal{I}\mathcal{I}})$ for $C \in
N_C$ or $C = \bot$.  For $C = D \sqcap E$, and assuming by induction that $\mathcal{B}_2
\models (D \sqsubseteq D^{\mathcal{I}\mathcal{I}})$ and $\mathcal{B}_2 \models (E
\sqsubseteq E^{\mathcal{I}\mathcal{I}})$, we can find that
\begin{equation*}
  \mathcal{B}_2 \models (D \sqcap E \sqsubseteq D^{\mathcal{I}\mathcal{I}} \sqcap E^{\mathcal{I}\mathcal{I}}).
\end{equation*}
But then $D^{\mathcal{I}\mathcal{I}} \sqcap E^{\mathcal{I}\mathcal{I}}$ is expressible in
terms of $M_{\mathcal{I}}$ (as a conjunction of model-based most-specific concept
descriptions, using \Cref{lem:mmsc-are-expressible-in-terms-of-M_I}), so
\begin{equation*}
  \mathcal{B}_2 \models ((D^{\mathcal{I}\mathcal{I}} \sqcap E^{\mathcal{I}\mathcal{I}})
  \sqsubseteq (D^{\mathcal{I}\mathcal{I}} \sqcap E^{\mathcal{I}\mathcal{I}})^{\mathcal{I}\mathcal{I}}).
\end{equation*}
Using \Cref{prop:double-II-under-I} we obtain $(D^{\mathcal{I}\mathcal{I}} \sqcap
E^{\mathcal{I}\mathcal{I}})^{\mathcal{I}\mathcal{I}} \equiv (D \sqcap
E)^{\mathcal{I}\mathcal{I}}$, so all in all
\begin{equation*}
  \mathcal{B}_2 \models (D \sqcap E \sqsubseteq (D \sqcap E)^{\mathcal{I}\mathcal{I}}).
\end{equation*}
Notice that the main arguments here are \Cref{prop:double-II-under-I} and that all
model-based most-specific concept descriptions are expressible in terms of
$M_{\mathcal{I}}$.

If $C = \exists r. D$, and assuming that $\mathcal{B}_2 \models (D \sqsubseteq
D^{\mathcal{I}\mathcal{I}})$, we first obtain
\begin{equation}
  \label{eq:22}
  \mathcal{B}_2 \models ( \exists r. C \sqsubseteq \exists r. C^{\mathcal{I}\mathcal{I}} ).
\end{equation}
But then $(\exists r. C^{\mathcal{I}\mathcal{I}}) \in M_{\mathcal{I}}$ up to equivalence,
so
\begin{equation*}
  \mathcal{B}_2 \models ( \exists r. C^{\mathcal{I}\mathcal{I}} \sqsubseteq (\exists
  r.C^{\mathcal{I}\mathcal{I}})^{\mathcal{I}\mathcal{I}}).  
\end{equation*}
Using \Cref{prop:double-II-under-I} again we obtain $(\exists
r. C^{\mathcal{I}\mathcal{I}})^{\mathcal{I}\mathcal{I}} \equiv (\exists
r. C)^{\mathcal{I}\mathcal{I}}$, so
\begin{equation*}
  \mathcal{B}_2 \models ( \exists r. C \sqsubseteq (\exists r. C)^{\mathcal{I}\mathcal{I}} ).
\end{equation*}
Notice that the crucial property in that argumentation is that $M_{\mathcal{I}}$ contains
concept descriptions of the form $\exists r. C^{\mathcal{I}\mathcal{I}}$, and that
\Cref{prop:double-II-under-I} has been used again.

The preceding argument then shows the following claim.

\begin{Theorem}[Theorem~5.10 of~\cite{Diss-Felix}]
  \label{thm:Felix-base-B2}
  Let $\mathcal{I}$ be a finite interpretation.  Then $\mathcal{B}_2$ as defined in
  \Cref{eq:26} is a finite base of $\mathcal{I}$.
\end{Theorem}

A practical disadvantage of the finite base $\mathcal{B}_2$ is its size, which may be
exponential in $\abs{ M_{\mathcal{I}} }$, which itself may be exponential in the size of
$\Delta^{\mathcal{I}}$.  To remedy this, we use methods from formal concept analysis to
extract bases from formal contexts.  In particular, recall that the canonical base of a
formal context is minimal in size among all bases of a formal context, and that it can be
computed effectively.  Having this in mind, we further observe that if we consider the
induced formal context $\con K_{\mathcal{I}} := \con K_{\mathcal{I}, M_{\mathcal{I}}}$,
then the set $\mathcal{L} := \set{ A \to A'' \mid A \subseteq M_{\mathcal{I}} }$ is a base
of $\con K_{\mathcal{I}}$, and that
\begin{equation*}
  \mathcal{B}_2 = \bigsqcap \mathcal{L} := \set{ \bigsqcap A \sqsubseteq \bigsqcap A''
    \mid (A \to A'') \in \mathcal{L} }.
\end{equation*}
Recall that $\bigsqcap A'' \equiv (\bigsqcap A)^{\mathcal{I}\mathcal{I}}$ by
\Cref{cor:mmsc-lattice}.

We can generalize this observation as follows: if $\mathcal{L} \subseteq \Th(\con
K_{\mathcal{I}})$ is a base of $\con K_{\mathcal{I}}$ which only contains implications of
the form $U \to U''$, then the set $\bigsqcap \mathcal{L}$ defined as
\begin{equation*}
  \bigsqcap \mathcal{L} := \set{ \bigsqcap U \sqsubseteq (\bigsqcap
    U)^{\mathcal{I}\mathcal{I}} \mid (U \to U'') \in \mathcal{L} }
\end{equation*}
is a base of $\con K_{\mathcal{I}}$.  Note that then $\bigsqcap \mathcal{L}$ is always a
subset of $\mathcal{B}_2$, but $\bigsqcap \mathcal{L}$ may be much smaller than
$\mathcal{B}_2$, for example if $\mathcal{L}$ is irredundant or even minimal.

However, there is a redundancy in $\mathcal{L}$ which cannot be removed this way: if $C, D
\in M_{\mathcal{I}}$ such that $C$ is subsumed by $D$, then the implication $\set{ C } \to
\set{ D }$ will always be true in $\con K_{\mathcal{I}}$.  But this means that this
implication has to be contained implicitly or explicitly in any base of $\con
K_{\mathcal{I}}$.  On the other hand, the resulting GCI $C \sqsubseteq D$ is trivial, and
thus dispensable.

We can alleviate this situation by making use of bases with background knowledge.  The
background knowledge we are interested in would be
\begin{equation}
  \label{eq:28}
  \mathcal{S}_{\mathcal{I}} := \set{ \set{ C } \to \set{ D } \mid C, D \in
    M_{\mathcal{I}}, C \sqsubseteq D }.
\end{equation}
A base of $\con K_{\mathcal{I}}$ with background knowledge $\mathcal{S}_{\mathcal{I}}$ now
does not have to contain the information about the implications in
$\mathcal{S}_{\mathcal{I}}$ anymore, and may thus may be smaller than a base without this
background knowledge.

\begin{Theorem}[Theorem~5.12 of~\cite{Diss-Felix}]
  \label{thm:Felix-base-B3}
  Let $\mathcal{I} = (\Delta^{\mathcal{I}}, \cdot^{\mathcal{I}})$ be a finite
  interpretation, and let $\mathcal{L}$ be a base of $\con K_{\mathcal{I}}$ with
  background knowledge $\mathcal{S}_{\mathcal{I}}$.  Assume that $\mathcal{L}$ only
  contains implications of the form $U \to U''$ for some $U \subseteq M_{\mathcal{I}}$.
  Then $\bigsqcap \mathcal{L}$ is a finite base of $\mathcal{I}$.
\end{Theorem}

We can extend this connection between bases of $\con K_{\mathcal{I}}$ and bases of
$\mathcal{I}$ even more: if $\mathcal{L}$ is the canonical base of $\con K_{\mathcal{I}}$
with background knowledge $\mathcal{S}_{\mathcal{I}}$, then $\bigsqcap \mathcal{L}$ is a
minimal base of $\mathcal{I}$.

\begin{Theorem}[Theorem~5.18 of~\cite{Diss-Felix}]
  \label{thm:Felix-5.18}
  Let $\mathcal{I} = (\Delta^{\mathcal{I}}, \cdot^{\mathcal{I}})$ be a finite
  interpretation, and define
  \begin{equation*}
    \mathcal{B} := \bigsqcap \set{ A \to A'' \mid (A \to A'') \in \Can(\con
      K_{\mathcal{I}}, \mathcal{S}_{\mathcal{I}}) }.
  \end{equation*}
  Then $\mathcal{B}$ is a minimal base of $\mathcal{I}$.
\end{Theorem}

So far, all bases we have obtained were \ELgfpbot-bases, \ie the GCIs contained in these
bases where allowed to contain proper \ELgfpbot concept descriptions.  From a logical
point of view this is not a problem.  However, \ELgfpbot concept descriptions are
inherently harder to read, since they allow for ``local recursion'' within concept
descriptions.  This may be undesired, as those concept descriptions may have to be
inspected by domain experts for their validity, and those experts may not necessarily be
experts in logic as well.

On the other hand, \ELbot concept descriptions are much easier to read, and thus obtaining
\ELbot bases instead of \ELgfpbot bases may be much more desirable.  For this, Baader and
Distel discuss a way to obtain such \ELbot bases from arbitrary \ELgfpbot bases by
\emph{unravelling}.

The crucial observation towards obtaining \ELbot bases from \ELgfpbot bases is that given
a finite interpretation $\mathcal{I}$ and a concept description $C$ it is true that for $d
\in \NN_0$ ``large enough'' that $C^{\mathcal{I}} = C_d^{\mathcal{I}}$.  Recall that $C_d$
denotes the unravelling of $C$ up to depth $d$.

\begin{Lemma}[Lemma~5.5 of~\cite{Diss-Felix}]
  \label{lem:Felix-lemma-5.5}
  Let $\mathcal{I} = (\Delta^{\mathcal{I}}, \cdot^{\mathcal{I}})$ be a finite
  interpretation, and let $C = (A, \mathcal{T})$ be an \ELgfp concept description.
  Then for $d = \abs{ N_D(\mathcal{T}) } \cdot \abs{ \Delta^{\mathcal{I}} } + 1$ it is
  true that $C^{\mathcal{I}} = C_d^{\mathcal{I}}$.
\end{Lemma}

Secondly, it is true that unravelling up to depth $d$ respects the structure of \ELbot
concept descriptions, as formulated in the following lemma.

\begin{Lemma}[Lemma~5.19 of~\cite{Diss-Felix}]
  \label{lem:unravelling-is-homomorphism}
  Let $C, D$ be two \ELgfp concept descriptions.  Then
  \begin{enumerate}[i. ]
  \item $(\exists r. C)_d \equiv \exists r. C_{d-1}$,
  \item $(C \sqcap D)_d \equiv C_d \sqcap D_d$.
  \end{enumerate}
\end{Lemma}

To now unravel an \ELgfpbot base $\mathcal{B}$ of $\mathcal{I}$ the idea is to just
unravel every GCI $(C \sqsubseteq D) \in \mathcal{B}$ ``deep enough'', \ie replacing these
GCIs by $C_d \sqsubseteq D_d$, where $d$ is chosen as in \Cref{lem:Felix-lemma-5.5}.
This, however, may not be enough, as we may not be able anymore to entail GCIs of the form
$(X^{\mathcal{I}})_d \sqsubseteq X^{\mathcal{I}}$ for $X \subseteq \Delta^{\mathcal{I}}$
from the base thus obtained.  To remedy this, some extra GCIs need to be added.

\begin{Theorem}[Theorem~5.21 of~\cite{Diss-Felix}]
  \label{thm:unravelling-ELgfpbot-bases}
  Let $\mathcal{I}$ be a finite interpretation and let $\mathcal{B}$ be a finite \ELgfpbot
  base of $\mathcal{I}$.  Then
  \begin{align*}
    \mathcal{B}_{\mathsf{u}} &:= \set{ C_d \sqsubseteq (C^{\mathcal{I}\mathcal{I}})_d \mid (C
      \sqsubseteq D) \in \mathcal{B} } \cup {} \\
    &\phantom{{}:={}} \set{ (X^{\mathcal{I}})_d \sqsubseteq (X^{\mathcal{I}})_{d+1} \mid X
      \subseteq \Delta^{\mathcal{I}}, X \neq \emptyset }
  \end{align*}
  is a finite \ELbot base of $\mathcal{I}$, where $d \in \NN_0$ is defined as in
  \Cref{lem:Felix-lemma-5.5}.
\end{Theorem}

We shall only give some intuition why this theorem is correct, as we shall discuss its
proof when we generalize it to bases of confident GCIs in \Cref{sec:unrav-elgfpb-bases}.
An important observation is that the set
\begin{equation*}
  \mathcal{X} := \set{ (X^{\mathcal{I}})_d \sqsubseteq (X^{\mathcal{I}})_{d+1} \mid X
    \subseteq \Delta^{\mathcal{I}}, X \neq \emptyset }
\end{equation*}
satisfies for all $X \subseteq \Delta^{\mathcal{I}}$
\begin{enumerate}[i. ]
\item $\mathcal{X} \models ( (X^{\mathcal{I}})_k \sqsubseteq (X^{\mathcal{I}})_{k+1} )$
  for all $k \in \NN_0, k \geq d$, and
\item $\mathcal{X} \models ( (X^{\mathcal{I}})_d \sqsubseteq X^{\mathcal{I}} )$.
\end{enumerate}
The first property can be shown by induction over $k$, and for the second property we
observe that if $\mathcal{J}$ is a finite interpretation such that $\mathcal{J} \models
\mathcal{X}$, then by the first property
\begin{equation*}
  ((X^{\mathcal{I}})_d)^{\mathcal{J}} \subseteq ((X^{\mathcal{I}})_{d+1})^{\mathcal{J}}
  \subseteq ((X^{\mathcal{I}})_{d+2})^{\mathcal{J}} \subseteq \dots
\end{equation*}
Since $\mathcal{J}$ is finite, for $k$ large enough it is true that
$((X^{\mathcal{I}})_k)^{\mathcal{J}} = ((X^{\mathcal{I}})_{k+1})^{\mathcal{J}}$ and thus
\begin{equation*}
  ((X^{\mathcal{I}})_k)^{\mathcal{J}} = (X^{\mathcal{I}})^{\mathcal{J}}.
\end{equation*}
Thus, $\mathcal{J} \models ( (X^{\mathcal{I}})_d \sqsubseteq X^{\mathcal{I}} )$ and
therefore $\mathcal{X} \models ( (X^{\mathcal{I}})_d \sqsubseteq X^{\mathcal{I}} )$,
because $\ELbot$ has the finite model property.

But then if $(C \sqsubseteq D) \in \mathcal{B}$, then $\mathcal{B}_{\mathsf{u}} \models (
(C^{\mathcal{I}\mathcal{I}})_d \sqsubseteq C^{\mathcal{I}\mathcal{I}})$ by the argument
just shown, and $\mathcal{B}_{\mathsf{u}} \models ( C_d \sqsubseteq
(C^{\mathcal{I}\mathcal{I}})_d )$, because this GCI is contained in
$\mathcal{B}_{\mathsf{u}}$.  Thus
\begin{equation*}
  \mathcal{B}_{\mathsf{u}} \models ( C \sqsubseteq C_d \sqsubseteq (C^{\mathcal{I}\mathcal{I}})_d
  \sqsubseteq C^{\mathcal{I}\mathcal{I}}),
\end{equation*}
and \Cref{lem:simple-entailment-with-mmsc} yields $\mathcal{B}_{\mathsf{u}} \models (C
\sqsubseteq D)$.  Thus, $\mathcal{B}_d$ entails all GCIs in $\mathcal{B}$, and since
$\mathcal{B}$ is complete for $\mathcal{I}$, $\mathcal{B}_{\mathsf{u}}$ is complete for
$\mathcal{I}$ as well.

%%% Local Variables: 
%%% mode: latex
%%% TeX-master: "../main"
%%% End: 

%  LocalWords:  Prediger gener conc
