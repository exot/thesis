\chapter{Axiomatizing Confident \EL-GCIs of Finite Interpretations}
\label{cha:axiom-conf-el}

The results obtained by Distel about computing finite bases of finite interpretations are
not only interesting from a theoretical point of view.  Although we have skipped most of
the details, all the relevant results are \emph{effective} in the sense that the bases
obtained can in principle be computed by computers.  Thus, these results may also be
interesting for practical applications.

A possible application of Distel's results is to compute bases from \emph{Linked Open
  Data}~\cite{Linked-Data}, a format for representing data as used by the semantic web.
This data format is mainly constituted of RDF triples and can be thought of as an
edge-labeled graph.  As such, it is very similar to interpretations, and thus Distel's
results are applicable here.

As a first contribution of this thesis we have implemented the major results obtained by
Distel on computing finite bases as described previously, and applied them to a particular
data set of the \emph{Linked Open Data Cloud}, namely to a subset of the DBpedia data
set~\cite{DBpedia}.  In \Cref{sec:computing-bases-from} we describe this experiment in
detail and show what Distel's results yield when applied to this data set.  This
experiment has also been discussed previously
in~\cite{Borchmann:confident-GCIs,DBLP:conf/icdm/BorchmannD11}.

One result obtained from this experiment is that Distel's results are very sensitive to
\emph{errors} in the data.  This is actually not surprising: bases of finite
interpretations only contain general concept inclusion which are valid in the data, and if
there is only one single counterexample to a general concept inclusions it is not
contained in any base.

If those counterexamples are erroneous, however, then this can cause problems.  Not only
that otherwise valid general concept inclusions are not obtained by Distel's results
anymore.  Sporadic erroneous counterexamples may also cause GCIs found while computing
bases to be rather complicated, because those GCIs have to avoid those erroneous
counterexamples.

To remedy, or at least to alleviate this effect of erroneous counterexample for computing
finite bases we shall consider an extension of Distel's results which tries to find bases
of GCIs which are not necessarily valid in the given interpretation, but instead enjoy a
\emph{high confidence} in it.  The notion of \emph{confidence} is borrowed from
data-mining~\cite{arules:agrawal:association-rules}, more precisely from the theory of
\emph{association rules}, and allows to measure how much an association rule is allowed to
ignore counterexamples.  We shall transfer the notion of confidence also to general
concept inclusions and shall then try to find bases for those \emph{GCIs with high
  confidence} instead.

For this we shall make use of results obtained by Luxenburger from his work on
\emph{partial implications}~\cite{diss:Luxenburger,Luxenburger91}, and extensions
thereof~\cite{DBLP:conf/ki/StummeTBPL01}.  Partial implications can be thought of as
implications considered together with their confidence in some particular formal context.
We shall discuss in \Cref{Luxen-base} how these results allow us to obtain bases of all
implications which have \emph{high confidence} in some given formal context.

In \Cref{sec:first-base} we then show how these ideas can be simulated in \ELgfpbot to
find bases for GCIs with high confidence.  Moreover, we shall also show in
\Cref{sec:bases-confident-gcis} how bases of implications with high confidence yield bases
of GCIs with high confidence.  Finally, we shall also discuss a minimality result in
\Cref{sec:minimality-result} which is similar to the one given in \Cref{thm:Felix-5.18},
as well as how to obtain \ELbot bases of \ELgfpbot bases for GCIs with high confidence.

The results thus obtained are again all effective, and we shall discuss some experiments
in \Cref{sec:exper-with-conf} which use the same data-set as the one used in
\Cref{sec:computing-bases-from}.  This allows us to directly compare the approaches of
computing bases of valid GCIs on the one hand, and bases of GCIs with high confidence on
the other.  Moreover, we shall also discuss shortcomings of the approach of considering
GCIs with high confidence, which will eventually lead us to considering extensions of the
attribute exploration algorithm.  These will be discussed in
\Cref{cha:expl-conf,cha:model-expl-conf}.

\section{Computing Bases from DBpedia}
\label{sec:computing-bases-from}

\todo[inline]{Write: talk about example data set from DBpedia's child relation}%
\todo[inline]{Write: show results}%
\todo[inline]{Write: motivate confidence of GCIs}%

\section{Confident GCIs of Finite Interpretations}
\label{sec:confident-gcis}

\todo[inline]{Write: introduce notion of confidence for GCIs of finite interpretations}%

\subsection{Luxenburger's Base}
\label{Luxen-base}

\todo[inline]{Write: confidence of implications}%
\todo[inline]{Write: introduce work of Luxenburger}%

\subsection{A Luxenburger-Style Base of all Confident GCIs}
\label{sec:first-base}

\todo[inline]{Write: adapt ideas from Luxenburger}%
\todo[inline]{Write: use neighborhood-relation}%
\todo[inline]{Write: show how to compute this base from the induced context}

\subsection{Bases of Confident GCIs from Bases of Confident Implications}
\label{sec:bases-confident-gcis}

\todo[inline]{Write: compute confident bases of GCIs from bases of confident implications
  (12-06/4.3) }%

\subsection{A Minimality Result}
\label{sec:minimality-result}

\todo[inline]{Write: completing sets and the minimality result}

\subsection{Unravelling \ELgfpbot Bases into \ELbot Bases}
\label{sec:unrav-elgfpb-bases}

\todo[inline]{Write: introduce unravelling of \ELgfpbot concept descriptions}%
\todo[inline]{Write: show how unravelling can be used to obtain \ELbot-Bases from
  \ELgfpbot ones (12-06/6)}%

\section{Experiments with Confident GCIs}
\label{sec:exper-with-conf}

\todo[inline]{Write: recall initial experiments, and show how the approach performs}%
\todo[inline]{Write: include results from 12-06/5}


%%% Local Variables: 
%%% mode: latex
%%% TeX-master: "../main"
%%% End: 

%  LocalWords:  DBpedia's
