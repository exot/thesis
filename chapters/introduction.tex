\chapter{Introduction}
\label{cha:introduction}

For a machine to interact in an intelligent manner with its outside world it needs to be
equipped with knowledge about it.  The very same is true for humans, and acquiring this
knowledge is a highly complex task which up to today is not completely understood.  To
help within this process, previously acquired knowledge is \emph{represented} in different
ways, \eg as books or films, to help others learning it.

However, the way knowledge is represented for human consumption is almost always not
suitable for machines.  While humans can extract knowledge from natural language, which
may be full of \emph{ambiguity}, machines require precise formulations of knowledge.  The
representation of knowledge in a machine-understandable way is the main focus of the area
of \emph{knowledge representation}~\cite{KRhandbook}.

One of the most successful approaches to knowledge representations are description
logics~\cite{DLhandbook}, a family of logic-based knowledge representations formalisms
with varying expressivity and reasoning complexity.  Description logics allow to represent
knowledge as \emph{knowledge bases} or \emph{ontologies}.  Essentially, these are just
collections of axioms, which may be either \emph{assertional} or \emph{terminological}.
For example, to represent the fact (the \emph{assertion}) that an individual \textsf{tom}
is a cat, one can use the assertional axiom
\begin{equation}
  \label{eq:14}
  \mathsf{Cat}(\mathsf{tom}).
\end{equation}
On the other hand, to state the terminological knowledge that every cat hunts mice, one
can use so-called \emph{general concept inclusions} and write
\begin{equation}
  \label{eq:15}
  \mathsf{Cat} \sqsubseteq \exists \mathsf{hunts}. \mathsf{Mouse}.
\end{equation}
As soon as a knowledge base is available, it is possible to \emph{reason} with it, \ie to
extract knowledge which is implicitly contained in this knowledge base.  For example, from
the two axioms state above, we can infer that the individual \textsf{tom} hunts mice.

Knowledge bases could be used to represent knowledge in a machine-consumable way.
However, the questions arises how to obtain such knowledge bases.  In one way or the
other, knowledge which is available to humans has to be translated into the form of a
description logic knowledge base.  Of course, this could be done by humans, but this
approach would not only be rather time-consuming but also prone to errors.  An automatic
translations would thus be highly welcome.  On the other hand, such an automatic
translation would just mean that machines can consume knowledge in the way humans
represent it, an assumption which is not reasonable.

However, one can still think about approaches which are \emph{semi-automatic} in the sense
that the results obtained from such a translation are preliminary and require further
refinement, or that the translation procedure requires additional \emph{assistance} by
means of human experts.

An approach to achieve such a semi-automatic translation procedure, or \emph{learning
  procedure}, has been made in~\cite{Diss-Felix}.  There, the focus lied in extracting
terminological knowledge of the form as given in \Cref{eq:14} from some given \emph{finite
  interpretation} $\mathcal{I}$.  More precisely, the procedures described
in~\cite{Diss-Felix} would automatically learn all general concept inclusions which are
\emph{valid} in the data set $\mathcal{I}$.  This approach made use of the mathematical
theory of \emph{formal concept analysis}, a subfield of mathematical order theory which
has very closed connections to description logics.

Of course, the general concept inclusions learned this way may not be correct, in the
sense that the general concept inclusion may hold in $\mathcal{I}$, but this data missed
to contain some relevant counterexamples.  One way to remedy this is to use an algorithm
from the field of formal concept analysis which is called \emph{attribute exploration}.
Within this algorithm, an expert (possibly human) is asked extracted general concept
inclusions for there correctness, and if proposed general concept inclusions are not found
then the expert has to provide \emph{counterexamples} to them.  This way, the problem that
the data may be \emph{incomplete} in that it lacks crucial counterexamples can be solved.
This has also been done in~\cite{Diss-Felix}.

However, there are still problems with this approach, especially if it comes to the
\emph{quality} of the data $\mathcal{I}$ from which general concept inclusions are
learned.  The main problem here is that the data may contain \emph{errors}.  These errors
may either cause otherwise valid general concept inclusions not to be found, because these
errors act as \emph{false counterexamples}, or general concept inclusions are found which
are not correct because errors cause positive counterexamples to vanish.  While the latter
approach can in theory be handled by the attribute exploration approach sketched above,
the former cannot, because the approach discussed in~\cite{Diss-Felix} will not even
extract general concept inclusions for which there may be errors in $\mathcal{I}$.

In this work, we want to extent the results obtain in~\cite{Diss-Felix} to this new
setting of where the data from which general concept inclusions are learned may contain
errors.  The main approach for this is to transfer the notion of
\emph{confidence}~\cite{arules:agrawal:association-rules} from the area of data-mining to
general concept inclusions.  Intuitively, this means that general concept inclusions may
have \emph{few} errors in the data, as opposed to having none in the original approach
of~\cite{Diss-Felix}.  The notion of ``few'' is quantified by means of the confidence of
the general concept inclusions in the data.

In the following we shall give a more in-depth overview of the problem sketched above, by
briefly introducing description logics and formal concept analysis in
Sections~\ref{sec:repr-knowl-using} and~\ref{sec:learn-impl-using}, respectively.  These
introductions will be mostly historical and by means of examples.  Thereafter, we discuss
in more detail what it means to learn general concept inclusions from data, and sketch our
approach of utilizing the notion of confidence to handle errors in the data.  This will be
done in Section~\ref{sec:extr-term-knowl}.  Finally, in Section~\ref{sec:related-work}, we
briefly discuss some related work which is important for our approach, and we summarize
our contribution in Section~\ref{sec:contributions}.

\section{Description Logics}
\label{sec:repr-knowl-using}

Description logics~\cite{DLhandbook} are a family of logic-based knowledge representation
formalisms, with a strong emphasis on well-defined semantics and practical reasoning
procedures.  The family of description logics contains various flavors of logical
formalisms, varying in expressiveness and reasoning complexity, allowing users to choose
the expressiveness they need, or the complexity they can afford, in their respective
applications.

The development of description logics~\cite{baader01overview} was motivated by earlier
knowledge representation formalisms like \emph{semantic networks}~\cite{SemanticNetworks}
or \emph{frame}~\cite{Minsky-Frames}, whose semantics was highly ambiguous, and mostly
depended on human interpretation or implementation details.  The need for well-defined and
predictable knowledge representation formalisms then lead to the first logic-based
systems~\cite{journals/cogsci/BrachmanS85}, which were, however,
incomplete~\cite{conf/kr/Schmidt-Schauss89}.

The term \enquote{description} in \enquote{description logics} is motivated by the
intention to use description logics to express knowledge about \emph{concepts
  descriptions}.  For this, description logics provide a number of constructors, which can
then be used to build concept descriptions from atomic \emph{concept names} and binary
\emph{role names}.  Then, two classical \emph{reasoning problems} are \emph{instance
  checking} and \emph{subsumption}: given an element $a$ and a concept description $C$,
the instance checking problem is to ask whether $a$ is an \emph{instance} of $C$, \ie
whether $a$ \emph{satisfies} the concept description $C$.  This can made more precise
using the notion of \emph{interpretations}, which are used more generally to define the
semantics of description logics.  Moreover, the subsumption problem is, given two concept
descriptions $C, D$, to ask whether it is true that $C$ is a \emph{subconcept} of $D$, \ie
whether every instance of $C$ is also an instance of $D$.

The first description logics considered were relatively small fragments of first order
logic, but already for them it could be shown that reasoning is
intractable~\cite{conf/aaai/BrachmanL84,journals/ai/Nebel88}.  One approach to remedy this
was to investigate highly optimized reasoning algorithms, which behave well in practice.
The most prominent class of such algorithms are \emph{tableau algorithms}, first invented
for the description logic
$\ALC$~\cite{journals/ai/Schmidt-SchaussS91,conf/ecai/HollunderNS90} for the subsumption
problem, and thereafter extended to other, even more expressive logics.  After a
connection of $\ALC$ to multimodal logic $\mathsf{K}_{(\mathsf{m})}$ was
discovered~\cite{DBLP:conf/ijcai/Schild91}, it was seen that this tableau algorithm is
actually a re-invention of the tableau algorithms used in modal logics.  The development
of description logics continued to investigate highly expressive description logics, whose
expressiveness exceeds that of $\ALC$, but which still behave well in
practice~\cite{journals/igpl/HorrocksST00}, and for which highly-optimized implementations
exist~\cite{sirin_pellet:practical_2007,Haarslev:2001,DBLP:conf/cade/TsarkovH06}.  This
finally lead to the adoption by the W3C of the \emph{Web Ontology Language OWL}, which is
based on the highly expressive description logic
$\mathcal{S}\mathcal{H}\mathcal{O}\mathcal{I}\mathcal{N}$~\cite{horrocks03fromshiqrdftoowl}.

The focus of description logics research departed from the sole focus on expressive
description logics when it was discovered at the beginning of this millennium that for the
in inexpressive description logic $\EL$ reasoning is
tractable~\cite{DBLP:conf/ijcai/Baader03a,DBLP:conf/ecai/Brandt04}, and stays polynomial
if the expressiveness of $\EL$ is extended
slightly~\cite{DBLP:conf/ijcai/BaaderBL05,BaaderEtAl-OWLED08DC}.  The practical relevance
of these results is given by the fact that large biomedical ontologies can be reformulated
as \emph{description logic knowledge bases} (or \emph{description logic ontologies}) using
just $\EL$ or a slight extension of it.  Examples for this are the \emph{Systematized
  Nomenclature of Medicine--Clinical Terms}, the Gene Ontology~\cite{gene-ontology}, and
large parts of the GALEN ontology~\cite{Rector199475}.

The main constructors of $\EL$ are conjunction ($\sqcap$) and existential restriction
($\exists$), and examples of $\EL$ concept descriptions are
\begin{equation*}
  \mathsf{Cat},\, \mathsf{Cat} \sqcap \mathsf{Mouse},\, \exists \mathsf{hunts}. \mathsf{Mouse}.
\end{equation*}
A description logic knowledge base formulated in $\EL$ usually consist, as any description
logic knowledge base, of two parts, namely an \emph{ABox}, holding assertional knowledge,
and a \emph{TBox}, containing terminological knowledge.  An example knowledge base would
then be
\begin{equation*}
  \mathcal{K} = (\set{ \mathsf{Cat} \sqsubseteq \exists \mathsf{hunts}. \mathsf{Mouse} },
  \set{ \mathsf{Cat}(\mathsf{tom}) }),
\end{equation*}
where the first entry denotes the TBox, and the second one denotes the ABox.

The semantics of knowledge bases is defined using \emph{interpretations} $\mathcal{I}$,
which can be thought of as edge- and vertex-labeled graphs.  The labels of the vertices,
which we shall \emph{elements} or \emph{individuals}, are concept names, and the labels of
the edges are role names.  An interpretation is a model of a knowledge base if all
elements satisfy the axioms contained in this knowledge base.  For example, the
interpretation
\begin{center}
  \begin{tikzpicture}[>=stealth']
    \begin{scope}[every node/.style = { draw, circle }]
      \node (Tom) {\textsf{tom}};
      \node[right=2cm of Tom, inner sep = .1cm] (Jerry) {\textsf{jerry}}; % fix
    \end{scope}
    \path (Tom) edge[->, bend left=30] node[midway, above] {\textsf{hunts}} (Jerry);
    \path (Jerry) edge[->, bend left=30] node[midway, below] {\textsf{hunts}} (Tom);
  \end{tikzpicture}
\end{center}
is a model of the knowledge base $\mathcal{K}$, since the element \textsf{tom} is labeled
with \textsf{Cat}, and every element which is labeled with \textsf{Cat} is connected to
some element labeled with \textsf{Mouse} via an edge labeled with \textsf{hunts}.

\section{Formal Concept Analysis}
\label{sec:learn-impl-using}

\todo[inline]{Write: say how we can learn implications using FCA (bases, exploration)}%

\section{Extracting Terminological Knowledge from Data}
\label{sec:extr-term-knowl}

\todo[inline]{Write: bring together the above two, say what we want to do; mention Felix}%

\section{Related Work}
\label{sec:related-work}

\todo[inline]{Write: mention Felix}%
\todo[inline]{Write: mention Luxenburger}%
\todo[inline]{Write: bottom-up construction}%
\todo[inline]{Write: Sebastian Rudolph}%

\section{Contributions}
\label{sec:contributions}

\todo[inline]{Write: say what's new}

\subsection{Experiments with Extracting Valid GCIs}
\label{sec:exper-with-extr}

\subsection{Extracting GCIs from Erroneous Data}
\label{sec:extracting-gcis-from}

\subsection{Exploration by Confidence}
\label{sec:expl-conf-2}

\subsection{Model-Exploration by Confidence with Completely Specified Counterexamples}
\label{sec:model-expl-conf}

\section{Acknowledgments}
\label{sec:acknowledgements}

\todo[inline]{Write: acknowledgments}

%%% Local Variables: 
%%% mode: latex
%%% TeX-master: "../main"
%%% End: 
